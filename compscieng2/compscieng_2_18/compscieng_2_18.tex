\documentclass[12pt,fleqn]{article}\usepackage{../common}
\begin{document}
Ders 18

[bazi multigrid yorumlari atlandi]

Krylov Matrisleri 

Bu matrislerden $K$ olarak bahsedecegiz ve bu yontem baglaminda 

\[ Ax = b \]

sistemini cozuyor olacagiz. Krylov matrisleri soyle yaratilir

\[ K_j = \left[\begin{array}{rrrrr}
b & Ab & A^2b & .. & A^{j-1}b
\end{array}\right] \]

Krylov altuzayi $\mathscr{K}$ ise ustteki kolonlarin lineer kombinasyonudur
(span), ya da ustteki matrisin kolon uzayidir da denebilir. Bu tur bir
matrisle niye ilgilenirim? Jacobi islemi aslinda bu kolonlarin
kombinasyonlarindan birini her adimda yavas yavas secer, yani aslinda
Krylov altuzayinin bir parcasinda calisir. Daha dogrusu ufak ufak baslar, o
altuzayda yavas yavas buyur.

Jacobi surekli bir kombinasyon secimi yapar, tabii bu secimin en iyi secim
oldugu soylenemez. Secimin en iyisini yapsak daha iyi olmaz mi? 

En iyiyi secmek icin kullanilacak metot eslenik gradyan (conjugate
gradient) olacak. Bu metot $K$ icinde $x_j$'yi secer. 

$\mathscr{K}$ uzayi yaklasiksal cozumumuzu aradigimiz yer tabii ki. Bu arada ustteki
$K$ matrisinin elemanlarini yaratmak cok kolay, matris carpimi yapiyoruz, ve
bir sonraki eleman bir oncekinin $A$ katidir, ve $A$ cogunlukla seyrektir
(sparse), bazen de simetriktir (eslenik gradyan metotu icin $A$ simetrik,
pozitif kesin olmali).

Ama EG metotundan once Arnoldi kavramini gormemiz lazim. 

Uygulamali Matematikte surekli bir seyler ``seceriz'', ve cogunlukla baz
vektorleri seceriz ve birkac ozellik arariz. Aradigimiz ozellikler
oncelikle hizdir, yukarida gordugumuz gibi, matris carpimi var, bu cok
hizli. Bir diger ozellik bagimsizlik. Bir digeri baz vektorlerinin
ortonormal olmasi. Bu son ozellik elde edilebilirse en iyisidir. Ustteki
$K$ pek iyi bir baz degildir. Arnoldi'nin amaci Krylov bazini ortogonalize
etmektir. $b,Ab,..$'yi alip $q_1,q_2,..,q_j$ olusturmaktir. Koda bakalim, 

\begin{minted}[mathescape,linenos,xleftmargin=1cm]{matlab}
q_1 = b / ||b||; % normalize et
for j = 1,..,n-1 % $q_{j+1}$'i hesaplamaya basla
  t = A * q_j; 
  for i = 1,..,j  % $t$ $\mathscr{K}_{j+1}$ uzayinda
    h_{ij} = q_i^T t  % h_{ij}q_i, $t$'nin $q_i$'ye yansimasi
    t = t - h_{ij}q_i % yansimayi cikart
  end % $t$, $q_1,..,q_j$'ye ortogonal
  h_{j+1,j} = ||t|| % $t$'nin buyuklugunu hesapla
  q_{j+1} = t / h_{j+1,j}
end % $q_1,..,q_j$ ortonormal
\end{minted}

Fikir Gram-Schmidt fikriyle cok benzer. 1. satirda ilk vektoru oldugu gibi
aliyoruz, sadece normalize ediyoruz. Sonra 3. satirda bir deneme
amacli bir vektor $t$'ye  bakiyoruz. Bu vektor ilk bastaki $b$'ye ortogonal
olmayacak muhakkak. O zaman 5. satirda bir ic carpim (inner product)
sonrasi, 6. satirda $t$'den cikartiyoruz. 8 ve 9. satirlarda bu
vektoru normalize ediyoruz. 

Eger $A$ simetrik ise, $h_{ij}h_{ij-1}$ carpimini birkac kere cikartmam
yeterlidir. 

Ornek 

\[  
A = 
\left[\begin{array}{rrrr}
1 &&& \\
 & 2 && \\
 && 3 & \\
 &&& 4 
\end{array}\right]
\]

$A$ hem simetrik, onun otesinde kosegen, ayrica oldukca seyrek. 

\[  
b = 
\left[\begin{array}{r}
1 \\ 1\\ 1 \\ 1
\end{array}\right]
\]


\[  
K_4 = 
\left[\begin{array}{rrrr}
1 & 1 & 1 & 1\\
1 & 2 & 4 & 8\\
1 & 3 & 9 & 27\\
1 & 4 & 16 & 64 
\end{array}\right]
\]

Krylov matrisi ustte. Ilk kolonu $b$ ile ayni. 2. kolon icin $A$ ile
carpmisiz. 3. icin bir daha $A$ ile carpmissiz, 4. icin bir daha. 

$K$ eger bir baz ise, temsil ettigi uzay tum $\mathbb{R}^4$'tur. Ustteki
ornekte $j = n = 4$, tum degerleri isledik. Eger $n$ cok buyuk bir sayi ise
mesela $10^5$ gibi, $j << n$ yani sona gelmeden cok once durmak
isteriz. Eslenik gradyan bunu basariyor. 

$K$ formatindaki bir matrise Vondermonde matrisi de denir, bu tur
matrislerde ilk kolon sabit, 3., 4., .. kolonlar ikincinin ustel 
halidir. 

Vondermond matrisleri pek iyi kosullandirilmis (conditioned) matrisler
degildir. Alakali bir soru: iyi, kotu kosullandirilmis matrisi nasil
anlariz? Matris tekil (singular) degil. Determinanti hesaplasak sifir
cikmaz. Ama neredeyse ``tekil olmaya yakin''. Bunun testini nasil yapariz? 

Matris tekil degil, o zaman ozdegerleri hesaplamak akla gelebilir,
oradan $\lambda_{min}, \lambda_{maks}$'i kontrol etmek.. Fakat simetrik
olmayan matrislerin ozdegerlerini hesaplamak hos degildir, ``guvenilir''
hesaplar degildirler. Cok kotu kosullandirilmis ama tum ozdegerleri 1 olan
matrisler olabilir mesela, caprazinda 1'ler olur, caprazin ustunden
katrilyonlar olabilir.. 

Bu isi dogru yapmanin yolu $V^TV$'ye bakmak. Yani genel kural, matris
simetrik degilse, devrigi ile kendisini carp, sonucun ozdegerleri hep
pozitif olur. $V^TV$'nin $i$'inci ozdegeri, $V$'nin $i$'inci ozdegerinin
karesi olacaktir. 

Bu arada $V^TV$ matrisine Gram matrisi denir. 

Eger $Q^TQ$ olsaydi kosullandirma sayisi (condition number), yani en buyuk /
en kucuk ozdeger ne olurdu? $Q^TQ = I$  o zaman caprazda hep 1 var, $1/1 =
1$. 
Bu en iyi kosullandirma sayisidir. 

Simdi su cok onemli formul icin gerekli her bilesene sahibiz. 

\[ AQ = QH \]

$A$ bize verilen -diyelim ki- simetrik matris. $Q$ Arnoldi'den gelen
baz. $H$ ise kodda gorulen carpan degerleri. Yani $QH$ bir nevi
Gram-Schmidt'teki gibi, hatirlarsak Gram-Schmidt $QR$ ile temsil
ediliyordu. $Q$ yine ortonormal, Gram-Schmidt'te $R$ ust kosegen. 

$H$ hesaplanirsa

\[  H = 
\left[\begin{array}{rrrr}
5/2 & \sqrt{ 5/2} && \\
\sqrt{ 5/2} & 5/2 & \sqrt{ 4/5}& \\
 & \sqrt{ 4/5} & 5/2 & \sqrt{ 9/20}\\
 &&  \sqrt{ 9/20} & 5/2 
\end{array}\right]
\]

$H$ simetrik ve uclu kosegen (tridiagonal). Uclu kosegenlik bize ne
soyler?  Tekrarin (recurrence) kisa oldugunu.

\[ AQ = QH \]

formulune donelim, kolonsal olarak ustteki carpimi nasil gosteririz? 

\[ Aq_1  = \frac{ 5}{2}q_1  + \frac{ \sqrt{ 5}}{2} q_2 \]

Tek bir kalemde eger $A$ simetrik ise $H$'in de simetrik oldugunu nasil
gosteririm? $H$'nin formulu lazim, 

\[ H = Q^{-1}AQ \]

$Q^{-1}$ nedir? $Q$'nun ortogonal oldugunu hatirlayalim, o zaman 
$Q^{-1} =
Q^T$. Ustte yerine koyalim,

\[ H = Q^{T}AQ \]

Buna bakarak $H$ kesin simetriktir diyebiliriz, simetrik matrisler aynen
ustteki gibi yaratilir zaten, ortaya bir simetrik matris koyarsin, sagdan
herhangi bir matris, soldan onun devrigi ile carparsin, ve yeni bir
simetrik matris ortaya cikar. 

Yani vardigimiz sonuc Krylov bazinin hizli, basit sekilde ortogonalize
edilebilecegidir. 



\end{document}
