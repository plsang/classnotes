\documentclass[12pt,fleqn]{article}
\setlength{\parindent}{0pt}
\usepackage{graphicx}
\usepackage{cancel}
\usepackage{listings}
\usepackage[latin5]{inputenc}
\usepackage{color}
\setlength{\parskip}{8pt}
\setlength{\parsep}{0pt}
\setlength{\headsep}{0pt}
\setlength{\topskip}{0pt}
\setlength{\topmargin}{0pt}
\setlength{\topsep}{0pt}
\setlength{\partopsep}{0pt}
\setlength{\mathindent}{0cm}

\begin{document}
MIT OCW Cok Degiskenli Calculus - Ders 19

Konumuz vektor alanlari (vector fields) ve cizgi entegralleri (line
integrals). Bundan onceki derslerde cift entegral (double integral)
konusunu isledik, fakat o tur entegraller cizgi entegrallerinden tamamen
farklidir. Bunu aklimizda tutalim.

Vektor Alanlari

Vektor alanlari bir vektordurler aslinda, diyelim ki $\vec{F}$

\[ \vec{F} = M\hat{i} + N\vec{j} \]

fakat $M,N$, $x,y$'nin bir fonksiyonudur. Bu demektir ki kordinat
sistemindeki her $x,y$ kombinasyonu icin degisik bir vektor olacaktir. Bir
misir tarlasinda her noktada misir vardir, vektor alaninda her noktada bir
vektor vardir [hoca bu analojiyi misirlar uzun, yonleri olan seyler oldugu
icin kullaniyor herhalde]. 












\end{document}
