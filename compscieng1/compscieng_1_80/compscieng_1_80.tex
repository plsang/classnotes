\documentclass[12pt,fleqn]{article}
\setlength{\parindent}{0pt}
\usepackage{graphicx}
\usepackage{cancel}
\usepackage{listings}
\usepackage[latin5]{inputenc}
\usepackage{color}
\setlength{\parskip}{8pt}
\setlength{\parsep}{0pt}
\setlength{\headsep}{0pt}
\setlength{\topskip}{0pt}
\setlength{\topmargin}{0pt}
\setlength{\topsep}{0pt}
\setlength{\partopsep}{0pt}
\setlength{\mathindent}{0cm}
\usepackage{latexsym}
\usepackage{showkeys}
\renewcommand*\showkeyslabelformat[1]{(#1)}

\begin{document}
Ek - z Transform 

Bu konuya Strang hocanin dersinde deginiliyor, fakat cok kapsamli olarak
islenmiyor. Bu yaziyi baska bir baslikta yayinlayabilirdik, fakat ait
oldugu yer bu ders, o sebeple buraya koyduk.

z Transform, Laplace Transformunun ayriksal dunyadaki karsiligidir. Bu
sebeple transform edilen surekli fonksiyon $f(t)$ degil, ayriksal, bir
vektor olarak gorulebilecek $x(n)$'dir. z Transform

\[ Z[x(n)] \leadsto X(z) = \sum_{-\infty}^{\infty} x(n)z^{-n}  \]

ki $z$ bir kompleks sayidir. 

Gelisiguzel (casual) sistemlerden gelen verilerde sadece $n>0$ veriye
bakilabilir, o zaman alt sinir sifir olur

\[ X(z) = \sum_{0}^{\infty} x(n)z^{-n}  \]

z Transform ne ise yarar? Laplace Transform diferansiyel denklemlerin
cozulmesine yardim ediyordu. z Transform benzer sekilde farklilik
(difference) denklemlerin cozulmesine yardim eder. Farklilik denklemi
mesela

\[ y(n) = 0.9 y(n-1) + x(n) \]

seklinde olabilir. Daha genel olarak farklilik denklemleri su sekilde
belirtilebilir,

\[ \sum_{k=0}^N a_k y(n-k) = \sum_{l=0}^M b_k x(n-l) 
\ \ \ \label{1}
\]

Iki ustteki ornek genel denklemin $N=1,M=0$ oldugu halidir, katsayilar 
$a_0
= 0.9,a_1=1,b_0=1$. Genel formdan $y(n)$'i disari cekebiliriz, o zaman $k$
sifir yerine $k=1$'den baslar

\[ y(n) + \sum_{k=1}^N a_k y(n-k) = \sum_{l=0}^M b_k x(n-l) \]

Farklilik denkleminin genel haline z Transform uygulariz. 

\[ X(z) = x(0) + x(1)z^{-1} + x(2) z^{-2} + ... 
\ \ \ \label{2}
\]

ise, bu dizin uzerinde zaman kaydirma islemi yapsak, yani $-1$ indeksi $0$
haline gelse, onun gibi tum degerler bir ileri kaysa, $x(-1)$, $x(0)$ olsa,
onun gibi $a$ katsayilari da ileri kaysa

\[ x(-1) + x(0)z^{-1} + x(1) z^{-2} + ...\]

Simdi $z^{-1}$'i disari cekelim

\[ = x(-1) + z^{-1} \bigg[ x(0) + x(1) z^{-1} + ... \bigg] \]

Koseli parantez icine bakarsak, oradaki degerler (2)'deki seriye benzemiyor
mu? O zaman oraya direk $X(z)$ degerini koyabiliriz

\[ = x(-1) + z^{-1}X(z)\]

Bir daha kaydirirsak, 

\[ z^{-2}X(z) + z^{-1}x(-1) + x(-2) \]

elde ederiz, $m$ kadar kaydirirsak

\[ z^{-m}X(z^{-1}) + z^{-m+1}x(-1) + z^{-m+2}x(-2) + ... + x(-m) \]

Eger baslangic sartlari sifir ise, ustteki formulde $x(-1),x(-2),..$
tamamen sifir kabul edilebilir, ve daha basit su formulu elde ederiz. 

\[ Z[x(n-m)] \leadsto z^{-m}X(z^{-1})\]

Simdi bu bilgiyle beraber (1)'in z Transformunu yapalim. 










Kaynaklar

[1] Introduction to DSP and Filter Design, B. A. Shenoi, pg. 41

[2] Digital Signal Processing, Ifeachor, pg. 105

[3] Digital Signal Processing using Matlab, Slicer, pg. 119

\end{document}


