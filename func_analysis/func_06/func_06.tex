\documentclass[12pt,fleqn]{article}\usepackage{../common}
\begin{document}
Ders 6

Hilbert Uzaylari 

Giris 

Her lise geometri ogrencisi bir noktadan bir cizgiye olan en kisa mesafenin
o cizgiye dik olan ikinci bir cizgiden gectigini bilir. Kabaca da hemen
gorulebilecek akla yatkin bu basit sonuc, noktadan duzleme olan mesafeler
icin de kolayca genellestirilebilir. Daha da ileri gidip n-boyutlu Oklit
uzaylarina genellemek gerekirse, bir noktadan bir altuzaya gidecek en kisa
vektor o altuzaya ortagonal olmalidir. Bu arada, bu son sonuc en kuvvetli,
onemli optimizasyon prensiplerinin biri olan Yansitma Teorisi'nin ozel
sartlarindan biridir.

Bu gozlemde kritik puf nokta ortoganalliktir. Ortoganallik kavrami genel
olarak norm edilmis uzaylarda mevcut degildir, ama Hilbert Uzaylarinda
mevcuttur. Hilbert Uzayi norm edilmis uzaylarin ozel bir halidir, norm
edilmis uzaylardaki ozelliklere ek olarak bir de icsel carpim (inner
product) islemi tanimlar, bu islem analitik geometrideki iki vektorun
noktasal carpimina (dot product) esdegerdir, iki vektorun icsel carpimi
sifir ise o vektorlerin ortogonal oldugu soylenebilir.

Icsel carpim ile kusanmis Hilbert Uzaylari iki ve uc boyutlardaki geometrik
buluslari genellememizi saglayacak yapisal bir cevher saglar bize, sonuc
olarak pek cok analitik cozumun Hilbert Uzaylarinda karsiligi vardir;
Ortonormal bazlar, Fourier Serileri, en az kareler minimizasyonu gibi
kavramlarinin hepsi Hilbert Uzayinda da kullanilabilirler.

On-Hilbert Uzaylari (Pre-Hilbert Spaces)

On-Hilbert Uzayi bir lineer vektor uzayi $X$ ile, $X \times X$ uzerinde
tanimlanmis bir ic carpim isleminin beraberligidir. Yani $X$'teki her
elemanin bir digeri (ve kendisi) ile eslesmesi uzerinde tanimli bir ic
carpim islemi vardir, ki bu islem $x,y \in X$, $(x|y)$ olarak gosterilir,
ve carpimin sonucu bir skalar, tekil buyukluk (mesela bir sayi) olacaktir. 

$\sqrt{ (x|x)}$ buyuklugu $||x||$ olarak gosterilir, norm operatoru tanidik
geldi herhalde, zaten birazdan yapacagimiz ilk islerden biri bu buyuklugun
hakikaten bir norma esit oldugunu gostermek. 

Onsartlar 

1. $(x|y) = \overline{(y|x)}$

2. $(x+y|z) = (x|z) + (y|z)$

3. $(\lambda x|y) = \lambda(x|y)$

Cauchy-Schwarz Esitsizligi

Bir ic carpim uzayinda (inner product space)r $x,y$ icin $|(x|y)| \le
\|x\|\|y\|$ olmali. 
Bu kucuktur ya da esittir ifadesindeki esitlik kismi sadece $x = \lambda y$
ise, ya da $y = \theta$ ise dogru. 

Ispat

$y = \theta$ icin esitlik kismi basitce dogrulanabilir. O zaman diger
sartlari kontrol etmek icin $y \ne \theta$ alalim. Bir skalar olan her
$\lambda$ degeri icin 

\[ 
0 \le (x-\lambda y | x-\lambda y) 
 \]

Belirlenen sartlara gore bu esitsizlik dogru olmali. Sadece dogru oldugunu
bildigimiz bir ifadeyi yazdik o kadar. Bir nevi oltayi attik, bekliyoruz. 
Sonra ustteki ifadeye 2. onsarti uyguluyoruz

\[ \le (x|x-\lambda y) - (\lambda y| x - \lambda y)\]

Bu iki terim uzerinde yine 2. onsarti ayri ayri kullaniyoruz

\[ 
\le (x|x) - (\lambda y|x) - 
\bigg[ (x|\lambda y) - (\lambda y|\lambda y) \bigg] 
\]

\[ 
\le (x|x) - (\lambda y|x) -  (x|\lambda y) + (\lambda y|\lambda y) 
\]

Icinde $\lambda$ olan tum terimler uzerinde 3. onsarti uyguluyoruz

\[ 
\le
(x|x) - \lambda(y|x) -
\lambda(x|y) + |\lambda|^2 (y|y)
 \]

Simdi $\lambda  = (x|y)/(y|y)$ farz ediyoruz. $\lambda$ her sey olabilecegine gore bu 
belirledigimiz sey de olabilir. Yerine koyunca, 

\[ \le (x|x) - 
\frac{ (x|y)(y|x)}{(y|y)} - 
\frac{ (x|y)(x|y)}{(y|y)} + 
\frac{ |(x|y)|^2}{|(y|y)|^2}(y|y)
 \]

1. onsarti kullanarak ustteki ucuncu terimin isaretini degistirelim

\[ \le (x|x) - 
\frac{ (x|y)(y|x)}{(y|y)} +
\cancel{\frac{ (x|y)(y|x)}{(y|y)}} + 
\cancel{\frac{ |(x|y)|^2}{|(y|y)|^{\cancel{2}}}(y|y)}
 \]

\[ \le (x|x) - 
\frac{ |(x|y)|^2}{(y|y)}
 \]

Ya da 

\[ |(x|y)|  \le \sqrt{ (x|x)(y|y)} = ||x||||y||\]

\[ \square \]

Teori 

Bir On-Hilbert uzayi $X$'te $||x|| = \sqrt{ (x|x)}$ bir normdur. 

Ispat

Norm icin tum tanimlar zaten ortaya cikti, tek eksik ucgensel esitsizlik
tanimi. Herhangi bir $x,y \in X$ icin 

\[ ||x+y||^2 = (x+y|x+y) \]

\[ = (x|x+y) + (y|x+y) \]

\[ = (x|x) + (y|x) + (x|y) + (y|y) \]

\[ = (x|x) + 2|(x|y)| + (y|y) \]

Simdi norm ifadesini kullanalim

\[ = ||x||^2 + 2|(x|y)| + ||y||^2 \]

Sagdan ikinci terimde  Cauchy-Schwarz teorisini uygulayalim

\[ \le ||x||^2 + 2||x||||y|| + ||y||^2 \]

Isaretin esitlikten esitsizlige dondugune dikkat. $||x||||y||$ kullanarak
$(x|y)$'tan daha buyuk olan bir buyukluk kullanmaya baslamis olduk, bu
yuzden esitligin sag tarafi, sol tarafindan buyuk hale geldi. Gruplarsak

\[ \le (||x||+||y||)^2  \]

Yani

\[ ||x+y||^2 \le (||x||+||y||)^2  \]

Karekok alirsak 

\[ ||x+y|| \le ||x||+||y||  \]

Bu ucgensel esitsizligin ta kendisidir. 

\[ \square \]

Tanim 

Tam olan bir On-Hilbert uzayi Hilbert Uzayi olarak adlandirilir. 

Hilbert Uzayi o zaman normu etkileyen / belirleyenbir ic carpim islemi
tanimlamis bir Banach uzayidir. $E^n,l_2,L_2[a,b]$ uzaylarinin hepsi
Hilbert uzaylaridir. Ic carpimlar bu arada altta gosterilen sureklilik
ozelligine sahiptir.

Teori 

Ic Carpimlarin Surekliligi: Diyelim ki bir On-Hilbert uzayinda $x_n \to x$
ve $y_n \to y$. O zaman $(x_n|y_n) \to (x|y)$.

Ispat

$\{x_n\}$ serisi yaklasiksal olduguna gore, sinirli (bounded) olmak
zorundadir; mesela diyelim ki $||x_n|| \le M$. Simdi,

\[ |(x_n|y_n) - (x|y)| \]

hesabini yapalim. Ifadenin icine $(x_n|y)$ arti ve eksi isaretleriyle
koyalim, 

\[ = |(x_n|y_n) - (x_n|y) + (x_n|y) + (x|y)| \]

Ustteki ilk ve son iki terimi gruplayalim, 2. onsarti tersten uyguluyoruz
yani, 

\[ = |(x_n|y_n-y) + (y|x_n-x)| \]

Ustteki mutlak degeri ortasindan parcalayacagiz. Ufak not, mutlak deger
operasyonu icin de ucgensel esitsizlik gecerlidir, yani 

\[ |a+b| \le |a| + |b| \]

O zaman 

\[ \le |(x_n|y_n-y)| + |(y|x_n-x)| \]

Her iki terim uzerinde ayri ayri Cauchy-Schwarz esitsizligini uygularsak, 

\[ \le ||x_n|||y_n-y)|| + ||y||||x_n-x|| \]

$||x_n||$ sinirli olduguna gore, 

\[ |(x_n|y_n) - (x|y)| \le M||y_n-y)|| + ||y||||x_n-x|| \to 0 \]

Esitsizligin sagi sifira gidiyor, cunku $M$ sabit, isptain basinda $x_n \to x$
ve $y_n \to y$ farzettik, o zaman ustteki farklar sifira gider. 

\[ \square \]
















\end{document}