\documentclass[12pt,fleqn]{article}
\setlength{\parindent}{0pt}
\usepackage{graphicx}
\usepackage{cancel}
\usepackage{listings}
\usepackage[latin5]{inputenc}
\usepackage{color}
\setlength{\parskip}{8pt}
\setlength{\parsep}{0pt}
\setlength{\headsep}{0pt}
\setlength{\topskip}{0pt}
\setlength{\topmargin}{0pt}
\setlength{\topsep}{0pt}
\setlength{\partopsep}{0pt}
\setlength{\mathindent}{0cm}
\usepackage{latexsym}
\usepackage{amsfonts}
\usepackage{showkeys}
\renewcommand*\showkeyslabelformat[1]{(#1)}

\begin{document}
Ders 3

Ornek 

Bir onceki ornegin dogal bir uzantisi n-boyutlu reel kordinat uzayi
olabilir. Bu uzaydaki vektorler n-ogeli icinde n tane reel sayi olan bir
dizidirler, ve vektorler $x = (\xi_1, \xi_2,...,\xi_n)$
formundadirlar. Reel tek sayi $\xi_k$'ye vektorun $k$'inci elemani adi
verilir. Iki vektor, eger tum ogeleri birbirine esit ise, esittir. Sifir
vektoru $\theta = (0,0,...,0)$ seklinde tanimlidir. 

$n$ boyutlu reel kordinat uzayi $R^n$ olarak tanimlanir. Buna tekabul eden
n-ogeli kompleks sayilarin uzayi $C^n$'dir. 

Bu noktada aslinda boyut kavramini devreye sokmak icin biraz erken. Daha
ileriki derslerde boyut kavraminin detayli tanimi yapilacak, ve bu
bahsettigimiz uzaylarin hakikaten n-boyutlu oldugu ispatlanacak. 

Ornek 

Sonsuz sayida eleman, sonsuz ogeli dizi iceren vektorlerlerle ilginc bazi
uzaylar insa edilebiliyor, ki bu uzayda tipik bir vektor vektorler 
$x =
(\xi_1, \xi_2,...,\xi_k,...)$ seklinde oluyor. Diger bir sekliyle $x =
\{\xi_k\} _{k=1}^{\infty}$. 
Toplama ve cikartma onceden oldugu gibi teker teker, sirasi birbirine uyan
ogeler arasinda yapiliyor. Reel sayilardan olusan her turlu sonsuz
dizilerin listesi bir vektor uzayi olusturuyor. Bir dizi $\{\xi_k\}$'ye
sinirli (bounded) denir, eger her $k$ icin.  $|\xi_k| < M$ olacak sekilde
bir $M$ sabiti var ise. Sonsuz ve sinirli (tanimini biraz once yaptik) olan 
her dizi bir vektor uzayi olusturur, cunku iki sinirli dizinin toplami, ya
da dizinin sayisal carpimi yine bir sinirli dizi olacaktir. 









\end{document}
