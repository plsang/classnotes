\documentclass[12pt,fleqn]{article}
\setlength{\parindent}{0pt}
\usepackage{graphicx}
\usepackage{cancel}
\usepackage{listings}
\usepackage[latin5]{inputenc}
\usepackage{color}
\setlength{\parskip}{8pt}
\setlength{\parsep}{0pt}
\setlength{\headsep}{0pt}
\setlength{\topskip}{0pt}
\setlength{\topmargin}{0pt}
\setlength{\topsep}{0pt}
\setlength{\partopsep}{0pt}
\setlength{\mathindent}{0cm}
\usepackage{latexsym}
\usepackage{amsfonts}
\usepackage{showkeys}
\renewcommand*\showkeyslabelformat[1]{(#1)}

\begin{document}
Ders 3

Ornek 

Bir onceki ornegin dogal bir uzantisi n-boyutlu reel kordinat uzayi
olabilir. Bu uzaydaki vektorler n-ogeli icinde n tane reel sayi olan bir
dizidirler, ve vektorler $x = (\xi_1, \xi_2,...,\xi_n)$
formundadirlar. Reel tek sayi $\xi_k$'ye vektorun $k$'inci elemani adi
verilir. Iki vektor, eger tum ogeleri birbirine esit ise, esittir. Sifir
vektoru $\theta = (0,0,...,0)$ seklinde tanimlidir. 












\end{document}
