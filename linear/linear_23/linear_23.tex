\documentclass[12pt,fleqn]{article}\usepackage{../common}

\begin{document}
Ders 23

Bu derste birinci dereceden, sabit katsayili (coefficients) lineer
diferansiyel denklem sistemini cozmeye gorecegiz. Form

\[ \frac{du}{dt} = A u \]

seklinde olacak. Katsayilar $A$ matrisi icinde. Temel fikir: sabit katsayili
lineer diferansiyel denklemlerin cozumu usteldir (exponential). Sonucun
ustel oldugunu bilince bulmamiz gereken ustel degerin ne oldugu, yani $e$
uzerine ne geldigi, onu neyin carptigi, ki bunlari bulmak lineer cebirin
isi olacak.

Bir ornekle baslayalim

\[ \frac{du_1}{dt} = -u_1 + 2u_2 \]

\[ \frac{du_2}{dt} = u_1 - 2u_2 \]

Ustteki sistemin katsayilarini disari cekersek $A$ matrisi su olur

\[ A = 
\left[\begin{array}{cc}
-1 & 2 \\
1 & -2
\end{array}\right]
 \]

Baslangic degerleri soyle olsun

\[ u(0) = 
\left[\begin{array}{c}
1 \\
0
\end{array}\right]
 \]

Eger $u_1$ ve $u_2$'yi bu sistemin temsil ettigi iki kap gibi gorsek
(mesela), her sey $u_1$'in ``icinde'' olarak baslayacakti, sonra zaman
gectikce oradan cikacak, $u_2$'ye dogru akacak. Tum bunlari $A$ matrisinin
ozdeger/vektorlerine bakarak anlayabilmemiz lazim. O zaman ilk isimiz
ozdeger/vektorleri bulmak olmali. 

$A$'ya bakinca ne goruyoruz? Bir kolon digerinin kati, o zaman matris
tekilsel (singular), bu demektir ki bir ozdeger $\lambda = 0$. Diger
ozdeger icin bir numara kullanalim; ozdegerlerin toplami matris izine
(trace) yani caprazdaki sayilarin toplamina esit olduguna gore, ve toplam
$=-3$, o zaman ikinci ozdeger $-3$ olmali. Ozvektorler ise

\[ 
\left[\begin{array}{c}
2 \\ 1
\end{array}\right] 
,
\left[\begin{array}{c}
1 \\ -1
\end{array}\right]
 \]

O zaman cozum iki cozumun toplami olacak

\[ u(t) = c_1 e^{\lambda_1 t}x_1 + c_2 e^{\lambda_2 t}x_2\]

Bu genel cozum. Genel cozum $c_1, c_2$ haricindeki birinci ve ikinci
terimdeki ``pur ustel'' cozumlerden olusuyor. Hakikaten mesela $e^{\lambda_1
  t}x_1$ 
dif denklemi cozuyor degil mi? Kontrol edelim. Denklem

\[ \frac{du}{dt} = Au\]

$u$ icin $e^{\lambda_1t}x_1$ koyalim, turev $t$'ye gore alindigina gore

\[ \lambda_1 e^{\lambda_1t}x_1 = A e^{\lambda_1t}x_1\]

Iki taraftaki ustel degerler iptal olur, geri kalanlar

\[ \lambda_1 x_1 = A x_1\]

Bu da ozdeger/vektor formuludur. Artik $u(t)$'nin son formulunu
yazabiliriz. 

\[ u(t) =
c_1 \cdot 1 \cdot 
\left[\begin{array}{c}
2 \\ 1
\end{array}\right]
+
c_2 e^{-3t} \cdot 
\left[\begin{array}{c}
1 \\ -1
\end{array}\right]
 \]

$t=0$ aninda

\[ u(t) =
c_1 
\left[\begin{array}{c}
2 \\ 1
\end{array}\right]
+
c_2 
\left[\begin{array}{c}
1 \\ -1
\end{array}\right] = 
\left[\begin{array}{c}
1 \\ 0
\end{array}\right] 
 \]

Iki tane formul, iki tane bilinmeyen var. Sonuc

\[ c_1 = \frac{1}{3},\ c_2 = \frac{1}{3}\]

yani

\[ u(t) =
\frac{1}{3}
\left[\begin{array}{c}
2 \\ 1
\end{array}\right]
+
\frac{1}{3}  e^{-3t} 
\left[\begin{array}{c}
1 \\ -1
\end{array}\right] 
 \]

Denklem kararli konuma (steady-state) gelince, yani $t \rightarrow \infty$ icin,
ustel bolum yokolacaktir, ve bastaki terim kalacaktir. 

\[ u(\infty) =
\frac{1}{3}
\left[\begin{array}{c}
2 \\ 1
\end{array}\right]
 \]

Fakat ustteki durum, yani sabit bir istikrarli konuma yaklasmak her zaman
mumkun olmayabilir. Bazen sifira, bazen de sonsuzluga da yaklasabiliriz. 

1. Stabilite 

Ne zaman $u(t) \rightarrow 0$? Ozdegerler negatif ise, cunku o zaman eksi
ustel deger olarak kucultucu etki yapacaklar. Eger $\lambda$ kompleks bir
sayi olsaydi, mesela 

\[ e^{(-3 + 6i)} \]

Bu sayi ne kadar buyuktur? Mutlak degeri (absolute value) nedir? 

\[ | e^{(-3 + 6i)t}| = e^{-3t} \]

cunku 

\[ |e^{6it}| = 1 \]

Niye 1? Cunku kesin deger isareti icindeki $e^{6it} = cos(6t)+isin(6t)$, ve
sagdaki $a+ib$ formudur, hatirlarsak kompleks eksenlerde $a$ ve $b$ ucgenin
iki kenaridir, hipotenus ise, ustte kesin deger olarak betimlenen seydir,
uzunlugu $a^2 + b^2$, yani $cos^2(6t) + sin^2(6t) = 1$. Aslinda $cos$ ve
$sin$ icine ne gelse sonuc degismezdi. 

Yani kompleks kisim konumuz icin onemli degil, cunku nasil olsa 1 olacak,
esas cozumu patlatabilecek (sonsuza goturecek), ya da kucultecek olan sey
reel kisim. Altta Re ile reel bolum demek istiyoruz.

2. Istikrarli konum: $\lambda_1 = 0$ ve oteki $Re \ \lambda < 0$. 

3. Patlama: herhangi bir $Re \ \lambda > 0$ ise. 

* * * 

Su formulun matris formu nedir?

\[ u(0) =
c_1 
\left[\begin{array}{c}
2 \\ 1
\end{array}\right]
+
c_2 
\left[\begin{array}{c}
1 \\ -1
\end{array}\right] 
=
\left[\begin{array}{c}
1 \\ 0
\end{array}\right] 
\]

Soyle

\[ 
\left[\begin{array}{cc}
2 & 1 \\ 1 & -1
\end{array}\right]
\left[\begin{array}{c}
c_1 \\ c_2
\end{array}\right] 
=
\left[\begin{array}{c}
1 \\ 0
\end{array}\right] 
 \]

Soldaki matris, ozvektor matrisi $S$. O zaman $Sc = u_0$ 

* * *

Probleme bakmanin degisik sekillerinden biri, onu ``baglantisiz
(decoupled)'' hale getirmek. Baslangic formulune donersek

\[ \frac{du}{dt} = Au \]

Bu $A$ baglantili (coupled) bir halde. $u = Sv$ kullanirsak

\[ S\frac{dv}{dt} = ASv\]

$S$ ozvektor matrisi. 

\[ \frac{dv}{dt} = S^{-1}ASv = \Lambda v\]

Bu geciste iyi bilinen esitlik $AS=S\Lambda$'nin $S^{-1}AS = \Lambda$
halini kullandik. Bu esitligin detaylari icin Hesapsal Bilim Ders 6'ya
bakabilirsiniz.

Boylece denklemler arasindaki baglantidan kurtulmus olduk. Tum sistem su
hale geldi:

\[ \frac{dv_1}{dt} = \lambda_1 v_1\]

\[ \frac{dv_2}{dt} = \lambda_2 v_2\]

..

Bu sistemi cozmek daha kolay, her biri icin ayri ayri cozumu yazalim,
mesela $v_1$

\[ v_1(t) = e^{\lambda_1 t} v_1(0) \]

Tamami icin

\[ v(t) = e^{\Lambda t} v(0) \]

Eger $u=Sv$ ise o zaman $v=S^{-1}u$, ve $v(0)=S^{-1}u(0)$

\[ S^{-1}u(t) = e^{\Lambda t} S^{-1}u \]

\[ u(t) = Se^{\Lambda t}S^{-1} u(0) \]

Aradigimiz formul ustteki su bolum

\[ e^{At} = Ae^{\Lambda t} S^{-1} \]

Ama bir matrisin ustel deger haline gelmesi ne demektir? Problemimizde elde
ettigimiz cozum bu oncelikle, yani $du/dt = Au$'nun cozumu $e^{At}$. Fakat,
yine soralim, bu demek? 

Guc serilerini (power series) hatirlayalim. $e^{x}$ icin guc serisi nedir? 

\[ e^x = 1 + x + \frac{1}{2}x^2 + \frac{1}{6}x^3 + ... \]

1 yerine $I$ (birim matris), ve $x$ yerine $At$ alirsak

\[ e^{At} = I + At + \frac{(At)^2}{2} + \frac{(At)^3}{6} + ... +
 \frac{(At)^n}{n!} + ..
\]

Bu ``guzel'' bir Taylor serisi. Aslinda matematikte iki tane cok guzel,
temiz Taylor serisi vardir, bir tanesi

\[ e^x = \sum_0^{\infty} \frac{x^n}{n!} \]

Oteki de Geometrik seri

\[ \frac{1}{1-x} = \sum_0^{\infty} x^n\]

Bu da en guzel guc serisidir. Onu da matris formunda kullanabilirdik aslinda

\[ (I-At)^{-1} = I + At + (At)^2 + (At)^3 + ... \]

Bu formul, bu arada, eger $t$ kucuk bir degerse bir matrisin tersini
hesaplamanin iyi yaklasiksal yontemlerinden biri olabilir. Cunku ustu
alinan kucuk $t$ degerleri daha da kuculuyor demektir; ustteki terimlerin
cogunu bir noktadan kesip atariz (yaklasiksallik bu demek), ve sadece 
$I =
At + (At)^2$ hesabi yaparak matris tersini yaklasiksal olarak
hesaplayabiliriz.

Zihin egzersizine devam edelim: $e^{At}$ acilimi $(I-At)^{-1}$ acilimindan
hangisi daha iyi? Ikincisi daha temiz, ama birincisi bir degere yaklasiyor
(converges), cunku gitgide buyuyen $n$ degerleriyle bolum yapiyorum. Demek
ki $t$ ne kadar buyurse buyusun, toplam bir sonlu (finite) sayiya dogru
gidiyor. Fakat ikinci acilim oyle degil. Eger $A$'nin 1'den buyuk ozdegeri
var ise, toplam patlar ($At$ tum ozdegerleri 1'den kucukse durum degisir
tabii, yani $|\lambda(At)| < 1$ ise).

Neyse, konumuza donelim. 

$e^{At}$'nin $Se^{\Lambda t}S^{-1}$ ile baglantili oldugunu nasil
gorebilirim? $e^{At}$'yi anlamak icin $S$ ve $\Lambda$'yi anlamaya calismak
lazim, $S$ zaten ozvektor matrisi, $\Lambda$ ise caprazsal (diagonal) bir
matris, tum degerleri caprazinda, bunlar nispeten temiz formlar, onlari
anlarsak isimiz kolaylasacak.

Peki gecisi nasil yapalim? $e^{At}$ acilimi olan guc serisini kullanarak bu
seriden $S$ ve $\Lambda$ cikmasini saglayabilir miyim acaba? Sunu zaten
biliyoruz: $A = S \Lambda S^{-1}$. Peki $A^2$ nedir?

\[ A^2 = (S \Lambda S^{-1})(S \Lambda S^{-1}) \]

\[ = S \Lambda S^{-1}S \Lambda S^{-1} \]

ortadaki $S$ ve $S^{-1}$ birbirini iptal eder. 

\[ = S \Lambda^2 S^{-1} \]

O zaman 

\[ e^{At} = I + S \Lambda S^{-1} t + \frac{S \Lambda^2 S^{-1}}{2}t^2 + ...\]

Disari tum $S$'leri cikartmak istiyorum. O zaman $I$'yi su sekilde yazarsam
daha iyi olacak (ki icinden $S$'leri cekip cikartabilelim), $I = SS^{-1}$

\[ = SS^{-1} + S \Lambda S^{-1} t + \frac{S \Lambda^2 S^{-1}}{2}t^2 + ...\]

ve

\[ = S (I + \Lambda t + \frac{\Lambda^2}{2}t^2 + ...) S^{-1}\]

Ortada kalanlar $e^{\Lambda t}$ degil mi? O zaman

\[ = S e^{\Lambda t}S^{-1} \]

Boylece gecisi yapmis olduk. Soru: bu formul hep isler mi? Hayir. Ne zaman
isler? Eger $A$ caprazlastiralabilen bir matris ise isler, yani icinden
$\Lambda$ matrisi cekip cikartilabilecek matrisler icin. Onu yapmanin on
sarti ise $A$'nin tersine cevirelebilir olmasi, yani tum ozvektorlerinin
bagimsiz olmasidir.

Peki $e^{\Lambda t}$ nedir? Yani caprazsal bir matris ustel deger olarak
karsimiza cikinca sonuc ne olur? $\Lambda$ nedir?

\[ \Lambda =
\left[\begin{array}{ccc}
\lambda_1 && \\
&&.. \\
&& \lambda_n
\end{array}\right]
 \]

Bunu $e$ uzeri olarak hesaplayinca ne elde ederiz? Sunu elde ederiz:


\[ e^{\Lambda t} =
\left[\begin{array}{ccc}
e^{\lambda_1t} && \\
&&.. \\
&& e^{\lambda_n t}
\end{array}\right]
 \]

Not: Ustteki acilim sezgisel olarak tahmin edilebilecek bir sey olsa da,
ustel olarak bir matris olunca, beklenen her islem yapilamayabiliyor. Mesela 
eger matris

\[ 
\left[\begin{array}{cc}
a & b \\ c & d
\end{array}\right]
 \]

olsaydi o zaman her elemani ustel olarak $e^{a}$, $e^b$ seklinde kullanmak
ve matris icindeki yerine yazmak ise yaramazdi. Ustel matris dunyasinin
kendine has bazi kurallari var.

Ornek

\[ y'' + by' + Ky = 0 \]

2. dereceden bu denklemi 1. dereceden formullerden olusan 2x2 bir
``sisteme'' donusturebiliriz. Ekstra bir denklem ortaya cikaracagiz, eger
$y$ yerine bir vektor formundaki $u$'yu su sekilde kullanirsak

\[ u = 
\left[\begin{array}{c}
y' \\ y
\end{array}\right]
 \]

$u$'nun turevi soyle olur

\[ 
u' = 
\left[\begin{array}{c}
y'' \\ y'
\end{array}\right]
 \]

$u'$ formuna gore 2. derece denklemi su sekilde temsil edebiliriz

\[ 
\left[\begin{array}{cc}
-b & -K \\
1 & 0
\end{array}\right]
\left[\begin{array}{c}
y' \\ y
\end{array}\right]
 \]

Genel olarak, mesela 5. dereceden bir denklemi alip 5x5 boyutlarinda 1. 
derece denklem sistemine gecmek te mumkundur. Bu gecis alttaki matrisin 
ust satirina katsayi degerleri atayacak, ve onun altindan baslayarak 1 
degerleri dolduracaktir. 

\[ 
\left[\begin{array}{rrrrr}
- & - & - & - & - \\
1 &&&& \\
& 1 &&& \\
&& 1 && \\
&&& 1 & 
\end{array}\right]
 \]

\end{document}

