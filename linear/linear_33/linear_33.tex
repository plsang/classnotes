\documentclass[12pt,fleqn]{article}\usepackage{../common}
\begin{document}
Lineer Cebir 33

Bir matrisin ne zaman mukemmel bir tersi (inverse) bulunur? 2 x 2 bir
matris mesela, ve bizim matris tersi dedigimiz sey aslinda ``iki tarafli
ters (two sided inverse)''tir. Mukemmel ters matrisi sagdan da soldan da
carpsak bize birim matrisi verir. 

$$ A A^{-1} = I = A^{-1}A $$

Ustteki mukemmel durumda $m = n = r$, yani satir, kolon sayisi ve kerte
(rank) birbirine esit. Bu duruma tam kerte (full rank) ismi verilir. Bu
durumu kitabin 2. bolumunde isledik. 

Sonra 3. bolumde tam kerte olmayan durumlara baktik, ve kertenin ne
oldugunu bulmayi ogrendik. Bu durumlarin bazilarinda matris tam kolon
kertesine (full column rank) sahip olabiliyordu, yani $n = r$. Bu durumda
kolonlar birbirinden bagimsiz olacaktir, ama belki satirlar
olmayacaktir. Peki bu durumda sifir uzayinda (nullspace) sadece 0 vektoru
vardir, cunku kolonlar bagimsiz, bu durumda kolonlarin hicbir kombinasyonu
bize sifir sonucunu vermez, sadece 0 vektoru ile carpmak bize sifir
sonucunu verebilir. 

Ustteki durumun en az kareler baglaminda onemli oldugunu hatirlayalim, eger
$Ax = b$'nin bir cozumu var ise bu cozum tek cozumdur, yani ``0 ya da 1
cozum vardir''.

Devam edelim, ustteki $r=n$ durumunda $A^TA$ tam kerte olur, 
cunku $n
\times m \cdot m \times n = n \times n$ boyutlarindadir, ve bu sonuc
matrisi tersine cevirilebilir (invertible) bir matristir.

Simdi size $A$'nin tek tarafli tersi (one-sided inverse) oldugunu gostermek
istiyorum.

$$ (A^TA)^{-1} A^T = A_{left}^{-1} = I$$

Yani ustteki ifadenin $A$'nin soldan tersi oldugunu soyledim. Peki $I$'ya
olan esitligi nereden biliyorum? Cunku ustteki ifadeyi alip $A$'yi soldan
carparsam, ve parantezleri degisik yerlere koyarsam,

$$ (A^TA)^{-1} (A^TA) = I$$

$A^TA$ ifadesinin en az karelerde katsayi matrisi olarak bulundugunu
hatirlayalim, tam kolon kerte durumunda $A^TA$'sinin tersi alinabilir, ve
en az kareler bu sekilde isini yapar.  $A_{left}^{-1}$'nin boyutlari $n \times m$. 

Sagdan Ters (Right Inverse)

Tam satir kertesi (full row rank) durumu kolon durumunun devrigi alinmis
hali, bu durumda $A^T$'nin sifir uzayi sifir vektorunu iceriyor, cunku
satirlar bagimsiz, kolonlar degil, ve $r < m = n$. Bu durumda $Ax = b$'nin
cozumu kesinlikle vardir, fakat birden fazla cozum vardir. Sagdan ters suna
benzer, 

$$ A_{right}^{-1} = A^T(AA^T)^{-1} $$

yani 

$$ A A^T(AA^T)^{-1} = I$$

Sagdan ters soldan tersin ayna yansimasi gibi. 

Simdi soldan terse donelim, eger bu tersi soldan degil, sagdan
uygulasaydik, bu durumda birim matris elde edemezdik, peki ne elde ederdik?


$$ A (A^TA)^{-1} A^T = ?$$

Ustteki ifade tanidik geliyor mu? Bu ifade yansitma (projection) matrisi
$P$ degil mi? Hatirlarsak $P$ matrisi $A$'nin kolon uzayina yansitma
yapilmasini sagliyordu. Bir baglamda aslinda $P$'ye imkansiz bir gorev
vermis olduk, o birim matrisine erismeye ugrasiyor fakat bunu her yerde
yapamiyor, yapabildigi yerde 0 diger yerlerde baska degerlere sahip
oluyor. 

Ayni sekilde sag tersi soldan carparsak, o zaman 

$$ A^T (AA^T)^{-1}A = ?$$

Bu matris nedir? Yine yansitma matrisi, bu sefer satir uzayina olan bir
yansitmadir. 








\end{document}
