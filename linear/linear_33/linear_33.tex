\documentclass[12pt,fleqn]{article}\usepackage{../common}
\begin{document}
Lineer Cebir 33

Bir matrisin ne zaman mukemmel bir tersi (inverse) bulunur? $2 \times 2$
boyutlarinda bir matris mesela, ve bizim matris tersi dedigimiz sey aslinda
``iki tarafli ters (two sided inverse)''tir. Mukemmel ters matrisi sagdan
da soldan da carpsak birim matrisi elde ederiz,

$$ A A^{-1} = I = A^{-1}A $$

Mukemmel durumda $m = n = r$, yani kare matris durumudur, yani satir, kolon
sayisi ve kerte (rank) birbirine esittir. Bu duruma tam kerte (full rank)
ismi de verilir. Kitabin 2. bolumunde isledik.

3. bolumde ise tam kerte olmayan durumlara baktik, ve kertenin ne oldugunu
bulmayi ogrendik. Bu durumlarin bazilarinda matris tam kolon kertesine
(full column rank) sahip olabiliyordu, yani $n = r$. O zaman kolonlar
birbirinden bagimsiz oluyordu, ama belki satirlar olmuyordu. Bu durumda
sifir uzayinda (nullspace) sadece 0 vektoru vardir, cunku kolonlar
bagimsiz, bu durumda kolonlarin hicbir kombinasyonu bize sifir sonucunu
vermez, sadece 0 vektoru ile carpmak bize sifir sonucunu verebilir.

Bu durumun en az kareler (least squares) baglaminda onemli oldugunu
hatirlayalim, eger $Ax = b$'nin bir cozumu var ise bu cozum tek cozumdur,
yani ``0 ya da 1 cozum vardir''.

$r=n$ durumunda $A^TA$ tam kerte olur, cunku $n \times m \cdot m \times
n = n \times n$ 
boyutlarindadir, ve bu sonuc matrisi tersine cevirilebilir
(invertible) bir matristir.

Simdi $A$'nin tek tarafli tersi (one-sided inverse) oldugunu gostermek
istiyorum.

$$ (A^TA)^{-1} A^T = A_{left}^{-1} $$

Yani ustteki ifadenin $A$'nin soldan tersi oldugunu soyledim. Peki $I$'ya
olan esitligi nereden biliyorum? Cunku ustteki ifadeyi alip $A$'yi soldan
carparsam, ve parantezleri degisik yerlere koyarsam,

$$  A_{left}^{-1} A  = (A^TA)^{-1} (A^TA) = I$$

$A^TA$ tanidik gelebilir, bu ifade en az kareler ortaminda katsayi matrisi
olarak gorulur. Tam kolon kerte durumunda $A^TA$'sinin tersi alinabilir, ve
en az kareler bu sekilde isini gorebilir.  $A_{left}^{-1}$'nin boyutlari $n
\times m$.

Sagdan Ters (Right Inverse)

Tam satir kertesi (full row rank) durumu kolon durumunun devrigi alinmis
hali, bu durumda $A^T$'nin sifir uzayi sifir vektorunu iceriyor, cunku
satirlar bagimsiz, kolonlar degil, ve $r < m = n$. Bu durumda $Ax = b$'nin
cozumu kesinlikle vardir, fakat birden fazla cozum vardir. Sagdan ters suna
benzer, 

$$ A_{right}^{-1} = A^T(AA^T)^{-1} $$

yani 

$$ A A^T(AA^T)^{-1} = I$$

Sagdan ters soldan tersin ayna yansimasi gibi. 

Simdi soldan terse donelim, eger bu tersi soldan degil, sagdan
uygulasaydik, bu durumda birim matris elde edemezdik, peki ne elde ederdik?


$$ A (A^TA)^{-1} A^T = ?$$

Ustteki ifade tanidik geliyor mu? Bu ifade yansitma (projection) matrisi
$P$ degil mi? Hatirlarsak $P$ matrisi $A$'nin kolon uzayina yansitma
yapilmasini sagliyordu. Bir baglamda aslinda $P$'ye imkansiz bir gorev
vermis olduk, o birim matrisine erismeye ugrasiyor fakat bunu her yerde
yapamiyor, yapabildigi yerde 0 diger yerlerde baska degerlere sahip
oluyor. 

Ayni sekilde sag tersi soldan carparsak, o zaman 

$$ A^T (AA^T)^{-1}A = ?$$

Bu matris nedir? Yine yansitma matrisi, bu sefer satir uzayina olan bir
yansitmadir. 

Ozel hali gorduk (sagdan, soldan). 

Artik genel hali gorebiliriz ki buna sozde hal (pseudo case) ismi
veriliyor. Diyelim ki bir $A$ var, bu $A$'nin kertesi $r < n$, yani satir
uzayinin biraz sifir uzayi var, $r < m$, yani kolon uzayinin biraz sifir
uzayinda birseyler var. Matris terisini ``bozan'' sey nihayetinde bu sifir
uzaylari degil midir? O zaman bu kosulda olabilecek en iyi matris tersi
nedir?

Bir vektor $x$ var ise, $Ax$ bize $A$'nin kolon uzayinda bir sonuc verir,
cunku $Ax$, $A$'nin kolonlarini bir sekilde kombine eder. Simdi diyelim ki
$x$ vektoru $A$'nin kendi satir uzayindan geliyor olsun. Hatta hem $x$ ve
$y$, $A$'nin satir uzayindan geliyor olsunlar, o zaman hem $Ax$ hem de $Ay$
$A$'nin kolon uzayinda olacaktir ve $Ax \ne Ay$ eger $x \ne y$ ise. 

Peki eger $Ax = Ay$ ise? O zaman 

$$ A(x-y) = 0 $$

Peki $x-y$ hangi uzaydadir? Satir uzayindadir. Cunku $x$ satir uzayinda,
$y$ satir uzayinda ise bu iki vektorun farki da satir uzayinda olacaktir. 

Fakat bu durumda $x-y$ hem satir uzayinda hem de sifir uzayinda. Bu vektor
hangi vektordur? Sifir vektorudur. Yani $Ax = Ay$ ise $x = y$ olmaya
mecbur. 

Yani $A$'nin herhangi bir satiriyla $A$'nin kolon uzayina gecis
yapabiliyoruz. $Ax$, ya da $Ay$, vs. O zaman, ve eger kolon uzayindan satir
uzayina geri gelebilirsek, yani $y = A^{+}(Ay)$ ile yani, bunu yapmamizi
saglayan $A^{+}$ ile gosterilen sey bir sozde terstir. Bu ifadenin iyi
tarafi sifir uzaylarindan tamamen uzak duruyor olmamiz. 

Sozde ters $A^{+}$'i nasil buluruz? 

1) SVD ile basla. SVD ile $A = U \Sigma V^T$. Kertesi $r$ olan bir matrisin
$\Sigma$'sinin sadece $r$ kosegen ogesi sifir olmayan deger icerir,
digerleri sifirdir. 

$$ 
\Sigma = 
\left[\begin{array}{cccc}
\sigma_1 & & & \\
 & \ddots & & \\
 &  & \sigma_r & \\
 &  &  & 0 \\
\end{array}\right]
 $$

Sozde tersi nasil olur? Tam tersi dusunelim, bu cok kolay olurdu, sadece
sifir olan kosegen degerlere uygulayamazdik bu degisimi, o zaman

$$ 
\Sigma^{+} = 
\left[\begin{array}{cccc}
1/\sigma_1 & & & \\
 & \ddots & & \\
 &  & 1/\sigma_r & \\
 &  &  & 0 \\
\end{array}\right]
 $$

Dikkat, $\Sigma$ $n \times m$ boyutlarinda, $\Sigma^{+}$ ise $m \times n$
boyutlarinda. 

$\Sigma \Sigma^{+}$ nedir? 


$$ \Sigma \Sigma^{+} =  
\left[\begin{array}{cccc}
1 & & & \\
 & \ddots & & \\
 &  & 1 & \\
 &  &  & 0 \\
\end{array}\right]
$$

Boyut $m \times m$. Carpimi degisik yonden yaparsam? 

$$ \Sigma^{+}\Sigma  =  
\left[\begin{array}{cccc}
1 & & & \\
 & \ddots & & \\
 &  & 1 & \\
 &  &  & 0 \\
\end{array}\right]
$$

Benzer durum, fakat boyut $n \times n$. 

$A = U \Sigma V^T$ formulune donersek, $A^{+}$ nedir? 

$V^{T}$ birim dikey (orthogonal) matristir, ve sozde tersi devrigine
esittir. $\Sigma$'yi gorduk, $U$ icin benzer durum $U^T$. 

$$ A^{+} =  V \Sigma^{+} U^T$$








\end{document}
