\documentclass[12pt,fleqn]{article}\usepackage{../common}
\begin{document}
Ders 4

Bu derste yapacaklarimiz sunlar; bir $AB$ carpiminin tersini (inverse)
nasil alirim? Biliyoruz ki 

\[ AA ^{-1}  = I = A ^{-1} A \]

Soru su ki 

\[ AB... = I \]

noktali kisma ne gelirse sonuc birim (identity) matris olur? Sonra bu
bilgiyi bir baska carpim uzerinde kullanacagiz, bu carpim eliminasyon
matrislerini bir sira carpim olarak gorecek, ve bu sekilde Gaussian
eliminasyon islemine degisik bir bakis getirmis olacagiz. 

$AB$'den birim matrise nasil erisirim? 

\[ ABB ^{-1} A ^{-1} = I \]

Ya da

\[ B ^{-1}A ^{-1} AB  = I\]

Simdi tersini alma isleminin devrigini alma ile nasil isleyecegine
bakalim. Diyelim ki 

\[ AA ^{-1} = I \]

Bunun tersini alirsam ne olur ?

\[ (A ^{-1})^T A^T = I \]

Devrik alinca siralama degisiyor bildigimiz gibi, ve birim matrisin devrigi
yine kendisi. 

Fakat ustteki ifade bir seyi daha soyluyor: $A^T$'yi ne carparsa sonuc $I$
gelir? Cevap $A^T$'nin tersi! Yani $(A ^{-1})^T$ ve  $(A^T) ^{-1}$
ifadelerinin ayni sey oldugunu soylemis oluyoruz. Bu nasil kullanilabilir?
Eger $A^T$'nin tersini hesaplamamiz gerekiyorsa, ve $A$'nin tersini bir
sekilde biliyorsam, onun devrigini almam yeterli. Diger bir deyisle,
tersini alma ve devrigini alma islemleri herhangi bir sirada yapilabilir. 

Eliminasyona gelelim.

Onlar bir matrisi anlamanin dogru yoludur denebilir. $A = LU$
faktorizasyonu bir matrisi en temel parcalarina ayirir. Diyelim ki $A$'dan
basliyorum, hic satir degis tokusu yapmadan sadece eliminasyon yaparak
ilerliyorum, ve $U$'ya erisiyorum, pivotlarimin hicbir sifir degil. Bu iki
matris arasindaki baglanti nedir? $A$ ile $U$ nasil alakali? Simdi
gorecegimiz uzere aradaki baglanti $L$. 

Ornek

\[ \stackrel{A}{
\left[\begin{array}{rr}
2 & 1 \\ 8 & 7
\end{array}\right]
}
 \]

$U$'ya yani bir ust ucgenel matrise (upper triangular matrix) erismek
istiyorum, 1. satirin 4 katini 2. satirdan cikartirim. Bu isleme 2'den 
1 ciktigini sembolize etmek icin $E_{21}$ adini verelim, 

\[ 
\stackrel{E_{21}}{
\left[\begin{array}{rr}
1 & 0 \\ -4 & 1
\end{array}\right]
}
\stackrel{A}{
\left[\begin{array}{rr}
2 & 1 \\ 8 & 7
\end{array}\right]
}
=
\stackrel{U}{
\left[\begin{array}{rr}
2 & 1 \\ 0 & 3
\end{array}\right]
}
 \]

O zaman sunu yazarsak, 

\[ 
\stackrel{A}{
\left[\begin{array}{rr}
2 & 1 \\ 8 & 7
\end{array}\right]
}
=
\stackrel{L}{
\left[\begin{array}{rr}
 &  \\  & 
\end{array}\right]
}
\stackrel{U}{
\left[\begin{array}{rr}
2 & 1 \\ 0 & 3
\end{array}\right]
}
 \]

$L$ diyen yere ne gelmeli? Basit, $E_{21}$'nin tersi gelmeli. $E_{21}A =
U$'yu 
$A=LU $ yapmak icin birinci formulu soldan $E_{21}$'nin tersi ile
carparim, o zaman bana $E_{21}$'nin tersi gerekli. 

Hatirlarsak, eliminasyon matrislerinin tersini almak kolaydir,

\[ 
\stackrel{A}{
\left[\begin{array}{rr}
2 & 1 \\ 8 & 7
\end{array}\right]
}
=
\stackrel{L}{
\left[\begin{array}{rr}
1 & 0 \\ 4 & 1
\end{array}\right]
}
\stackrel{U}{
\left[\begin{array}{rr}
2 & 1 \\ 0 & 3
\end{array}\right]
}
 \]

$L$ alt ucgensel (\textbf{l}ower triangular) demek, $U$'nun kosegeninde
pivotlar var, sol alt kisminda sifirlar. 















\end{document}
