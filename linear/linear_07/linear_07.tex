
\documentclass[12pt,fleqn]{article}\usepackage{../common}
\begin{document}
Lineer Cebir - Ders 7

Vektor uzaylarindan, ozellikle sifir uzayindan (nullspace), ve kolon
uzayindan bahsettik, simdi bu uzaylarin icindeki vektorleri nasil
bulacagimizi, nasil hesaplayacagimizi gorecegiz. Yani onceki derste
gordugumuz tanimlari bu derste algoritmaya donusturecegiz. $Ax=0$'i cozen
algoritma nedir, mesela. Ornek uzerinde gorelim,

$$ 
A = 
\left[\begin{array}{rrrr}
1 & 2 & 2 & 2  \\
2 & 4 & 6 & 8 \\
3 & 6 & 8 & 10
\end{array}\right]
 $$

Ilk bakista gozume carpan 2. kolon 1. kolonun bir kati. Ya da 2. kolon
1. ile ``ayni yonde'', bu iki kolon ``bagimsiz degil''. Tabii bu bilgileri 
cozum sirasinda da algoritmanin bir yan etkisi olarak kesfetmeyi
bekleriz. Satirlara bakiyorum, 1. ve 2. toplami 3. ile ayni, yani 3. satir
bagimsiz degil. Tum bunlar eliminasyonun yan urunleri olarak bulunmalilar. 

Evet ana algoritmamiz eliminasyon, onun dikdortgensel kosula adapte edilmis
hali [pivot ta sifir var ise durmadan cozume devam ediyoruz ?? vs].

Eliminasyon sirasinda yapilan islemler sifir uzayini degistirmez. Degil mi?
Eger mesela bir denklemin (satirin) bir katini bir diger denklemden
cikartiyorsam bu nihai cozumu degistirmez, cunku denklem sistemi
degismemistir. 





















\end{document}
