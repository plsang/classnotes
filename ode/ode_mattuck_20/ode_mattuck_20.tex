\documentclass[12pt,fleqn]{article}
\setlength{\parindent}{0pt}
\usepackage{graphicx}
\usepackage{cancel}
\usepackage{listings}
\usepackage[latin5]{inputenc}
\usepackage{color}
\setlength{\parskip}{8pt}
\setlength{\parsep}{0pt}
\setlength{\headsep}{0pt}
\setlength{\topskip}{0pt}
\setlength{\topmargin}{0pt}
\setlength{\topsep}{0pt}
\setlength{\partopsep}{0pt}
\setlength{\mathindent}{0cm}
\usepackage{latexsym}
\usepackage{showkeys}
\renewcommand*\showkeyslabelformat[1]{(#1)}

\begin{document}
Ders 20

Konumuz Laplace Transformu ile diferansiyel denklem cozmek. Fakat onu
yapmadan once, Laplace Transformunun mumkun oldugundan emin olmamiz
gerekiyor. Bazilariniz dusunebilir, ``ama hocam alttaki formul her zaman
hesaplanaz mi?''

\[ F(s) = \int_0^{\infty} f(t)e^{-st} \ dt \]

Cevap hayir cunku ustteki bir uygunsuz (improper) entegral, ust sinir
sonsuzluga gidiyor ve bildigimiz gibi uygunsuz entegraller her zaman bir
degere yaklasmiyorlar (converge). 





\end{document}
