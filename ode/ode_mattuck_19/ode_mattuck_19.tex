\documentclass[12pt,fleqn]{article}
\setlength{\parindent}{0pt}
\usepackage{graphicx}
\usepackage{cancel}
\usepackage{listings}
\usepackage[latin5]{inputenc}
\usepackage{color}
\setlength{\parskip}{8pt}
\setlength{\parsep}{0pt}
\setlength{\headsep}{0pt}
\setlength{\topskip}{0pt}
\setlength{\topmargin}{0pt}
\setlength{\topsep}{0pt}
\setlength{\partopsep}{0pt}
\setlength{\mathindent}{0cm}
\usepackage{latexsym}
\usepackage{showkeys}
\renewcommand*\showkeyslabelformat[1]{(#1)}

\begin{document}
Ders 19 

Bu dersten baslayarak ve birkac ders boyunca cogu muhendis ve bazi
bilimcinin, onlarin karsilastigi turden tum diferansiyel denklemleri
cozmekte en populer buldugu yontemi gorecegiz. Yontemin ismi Laplace
Transformu. 

Bu yontemi kullanmak icin birkac hafta yeterli, ama o zaman bile metot
etrafinda belli bir gizem bulutu kaliyor, insanlar teknigin nereden
geldigini bir turlu anlayamiyorlar, ve dogal olarak bu onlari rahatsiz
ediyor.  

Laplace Transformunu anlamanin iyi bir yolu onu ustel seri (power series)
olarak gormektir. Bir ustel seri bildigimiz gibi su formdadir

\[ \sum_{0}^{\infty} a_n x^n \]

Bu seriye yapilacak en dogal islem onu toplamaktir. Sonuc genelde $f(x)$
gibi bir genel tanimla gosterilir, biz burada gelenekten biraz kopacagiz,
toplamin $a$ ile iliskisini iyice belli etmek icin $A(x)$ kullanacagiz. 

\[ \sum_{0}^{\infty} a_n x^n = A(x)
\ \ \ \label{1}
\]

Bir degisiklik daha: $a_n$ aslinda bir ayriksal dizin icindeki belli $a$
degerleri, bunu da iyice belli etmek icin bilgisayar notasyonu kullanalim,
$a_n$, $a(n)$ olsun. 

\[ \sum_{0}^{\infty} a(n) x^n = A(x)\]

Bu sekilde bakinca, ustel serinin yaptigi bir ayriksal fonksiyonu $a$'yi
(cunku icinde reel sayilar var, ve bir fonksiyon) belli bir toplam ile
ilintilendirmek. 

\[ a(n) \leadsto A(x) \]

Peki eger $a(n) = 1$ ise, yani fonksiyon hep ayni sabit deger 1'i
veriyorsa, o zaman toplam ne olur? 

\[ 1 \leadsto \frac{1}{1-x}, \ \ |x|<1 \]

cunku $a(n) = 1$ ise ustel seri 

\[ = 1 \cdot x + 1 \cdot x^2 + 1 \cdot x^3 + ... \]

\[ = x + x^2 + x^3 + ... \]

olacaktir, ve bu toplam $1/1-x$'e yaklasir (dikkat: $|x|<1$ oldugu
durumda). 

Baska bir fonksiyona daha bakalim, $1 / n!$ 

\[ \frac{1}{n!} \leadsto e^x \]

Bu (ilginc) bakis acisinda gore, isleme bir ayriksal fonksiyon giriyor,
disariya bir surekli fonksiyon cikiyor. Bu arada dikkat, giren $n$ bazli, cikan $x$
bazli. 

Simdi diyelim ki ayriksal olan toplam islemini surekli hale getirmek
istiyorum. Once

\[ n = 0,1,2,.. \]

yerine surekli bir degisken kullanmaya baslarim, mesela

\[ t: 0 \le t \le \infty \]

ki bu $t$ ustteki araliktaki tum reel degerleri tasiyabilecek. 

Fakat ayriksaldan sureklilige gecince toplam islemini kullanamam, onun
yerine entegral kullanmam gerekir. 

\[ \int_0^{\infty} a(t)x^t \ dt \]

Peki bu neyin fonksiyonu acaba? $t$'nin degil cunku onun ``uzerinden''
entegre ediyorum / yokediyorum (integrate out), entegrasyon sonrasi $t$
kalmiyor. Hayir, ustteki fonksiyon her $x$ icin belli bir deger
hesaplayacagina gore, o $x$'in bir fonksiyonu olmali. 

\[ \int_0^{\infty} a(t)x^t \ dt = A(x)\]

Fakat hala isimiz bitmedi. Hicbir muhendis, matematikci ustteki gibi bir
formu kullanmaz, cunku $x$ baz halde ve bu tur ifadeler turev alirken,
entegrasyon yaparken problem cikartabilir. Daha iyi bir baz $e$ olur, $e$
bazli turev almayi, entegrasyon yapmayi herkes sever [cunku cok kolay]. 

\[ x = e^{ln \ x} \]

\[ x^t = (e^{ln \ x})^t \]

Bir nokta daha: eger $x > 1$ ise ustteki entegralin bir deger yaklasmasi
(converge) cok zordur, cunku entegral ust sinirinda $\infty$ var, bu tur
entegraller dikkatli muamele ister. O zaman su sarti koyarim, $0 < x < 1$. 

Eger  $0 < x < 1$ olacak ise, $ln \ x < 0$ demektir. 

Tabii kimse $ln \ x$'i bir degisken olarak kullanmaz, 

\[ s = ln \ x \]

ve negatif degiskenlerle ugrasmak istemedigimiz icin 

\[ -s = -ln \ x \] 

yapariz, ki hep pozitif olan $-s$ ile is yapabilelim. 

Tum bunlar kozmetik degisiklikler bu arada, sembolik olarak bu islemi
hosumuza giden bir seye dondurmek icin yaptiklarimiz. 

Entegrali tekrar yazalim simdi. Oncelikle kimse fonksiyon olarak $a(t)$
ismini kullanmaz, ona $f$ diyelim. 

\[ \int_0^{\infty} f(t)(e^{-s})^t \ dt = ..\]

Ustunu almak, kurallara gore carpima donustugune gore

\[ \int_0^{\infty} f(t)e^{-st} \ dt = ..\]

Fonksiyon neye esit? $A(x)$, $x$'in fonksiyonu idi, ama simdi $s$
kullaniyoruz, o zaman 

\[ \int_0^{\infty} f(t)e^{-st} \ dt = F(s)\]

Nihai formul ustteki. Ve tekrar vurgulayayim, bu formul ayriksal ustel seri
(1)'in analog versiyonundan ibaret. Iste Laplace Transformu budur. 

















\end{document}
