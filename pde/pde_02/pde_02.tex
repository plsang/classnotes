\documentclass[12pt,fleqn]{article}
\setlength{\parindent}{0pt}
\usepackage{graphicx}
\usepackage{cancel}
\usepackage{listings}
\usepackage[latin5]{inputenc}
\usepackage{color}
\setlength{\parskip}{8pt}
\setlength{\parsep}{0pt}
\setlength{\headsep}{0pt}
\setlength{\topskip}{0pt}
\setlength{\topmargin}{0pt}
\setlength{\topsep}{0pt}
\setlength{\partopsep}{0pt}
\setlength{\mathindent}{0cm}

\begin{document}
PDE - Ders 2

Denklem soyle idi

\[ a(x,y)u_x + b(x,y)u_y + c(x,y)u = f(x,y) \]

Bu denklem homojen degil, cunku denklemin sol tarafi $f \ne 0$, nihai test
tabii ki $u=0$ koyunca $0=0$ cikip cikmayacagi. 

Cozum icin kullandigimiz fikir neydi? Kordinat sistemini transform etmek,
ki 

$\bigg(x,y\bigg) \to \bigg(\xi(x,y), \eta(x,y)\bigg)$

olsun. Bu degisimi yaparken oyle bir degisim ariyoruz ki boylece transform
edilmis PDE'miz cozulmesi kolay bir hale gelsin. 


\end{document}
