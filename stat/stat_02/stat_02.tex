\documentclass[12pt,fleqn]{article}\usepackage{../common}
\begin{document}
Ders 2

Ornek 

Simdi oyle bir $q$ bul ki $P(X < q) = .2$ olsun. Yani $\Phi^{-1}(.2)$'yi
bul. Yine $X \sim N(3,5)$. 

Cevap 

Demek ki tablodan $.2$ degerine tekabul eden esik degerini bulup, ustteki
formul uzerinden geriye tercume etmemiz gerekiyor. Normal tablosunda
$\Phi(-0.8416) = .2$, 

\[ .2 = P(X<q) = P( Z < \frac{ q - \mu}{\sigma}) = \Phi(\frac{ q - \mu}{\sigma})
\]

O zaman 

\[ -0.8416 = \frac{q - \mu}{\sigma} = \frac{ q - 3}{\sqrt{ 5}} \]

\[ q = 3 - 0.8416 \sqrt{ 5} = 1.1181 \]













\end{document}
