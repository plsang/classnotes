\documentclass[12pt,fleqn]{article}\usepackage{../common}
\begin{document}
Banglades'te su kuyusu degisiminin lojistik modeli

Bu analiz Gelman ve Hill'in kitabi *Data Analysis Using Regression and
Multilevel/Hierarchical Models* 5.4'uncu bolumu isliyor.

Verimizde 3,000 haneye gidilerek anketle toplanmis veri var. Veride
hanelerin yakinlarindaki kuyudaki arsenik seviyesi toplanmis, ve
paylasilan verideki tum hanelerin kuyular sagliksiz seviyede arsenik
iceriyor. Verideki diger bilgiler en yakindaki "saglikli" bir kuyuya
yakinlik, ve o hanenin bu saglikli su kuyusuna (bir sene sonra yapilan
kontrole gore) gecip gecmedigi.  Ayrica hanede fikri sorulan kisinin
egitim seviyesi ve bu hanedeki kisilerin herhangi bir sosyal topluluga
(community assocation) ait olup olmadiklari.

Amacimiz su kuyusunun degisimini modellemek. Bu eylem olup / olmama baglaminda
evet / hayir seklinde bir degisken oldugu icin ikili (binary) olarak
temsil edilebilir ve ikili cevaplar / sonuclar lojistik regresyon ile
modellenebilirler.

Veriye bakalim.

\begin{minted}{python}
from pandas import *
from statsmodels.formula.api import logit
from statsmodels.nonparametric import KDE
from patsy import dmatrix, dmatrices
\end{minted}

\begin{minted}{python}
df = read_csv('wells.dat', sep = ' ', header = 0, index_col = 0)
print df.head()
\end{minted}

\begin{verbatim}
   switch  arsenic       dist  assoc  educ
1       1     2.36  16.826000      0     0
2       1     0.71  47.321999      0     0
3       0     2.07  20.966999      0    10
4       1     1.15  21.486000      0    12
5       1     1.10  40.874001      1    14
\end{verbatim}

Model 1: Guvenli su kuyusuna uzaklik

Ilk once modelde kuyu uzakligini kullanalim. 

\begin{minted}{python}
mod1 = logit('switch ~ dist', df = df).fit()
print mod1.summary()
\end{minted}











\end{document}
