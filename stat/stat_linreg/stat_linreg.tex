\documentclass[12pt,fleqn]{article}\usepackage{../common}
\begin{document}
Lineer Regresyon

Bir hedef degiskeninin bir veya daha fazla kaynak degiskenine olan
baglantisini bulmak icin en basit yontemlerden biri bu iliskinin lineer
oldugunu kabul etmektir, ve degiskenlerin carpildigi agirliklari bulmak
icin En Az Kareler (Least Squares) en iyi bilinen yontemlerden biri. En Az
Kareleri daha once pek cok degisik ders notlarinda, yazida turettik. Mesela
{\em Cok Degiskenli Calculus Ders 9}, {\em Lineer Cebir Ders 15}, ya da
Uygulamali Matematik yazilarindan {\em Regresyon, En Az Kareler (Least
  Squares)} yazilarinda. 


\end{document}
