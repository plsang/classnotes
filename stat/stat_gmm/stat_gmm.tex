\documentclass[12pt,fleqn]{article}\usepackage{../common}
\begin{document}
Gaussian Karisimlari ile Deri Rengi Saptamak

Bir projemizde dijital resimlerdeki deri rengi iceren kisimlari cikartmamiz
gerekiyordu; cunku fotografin diger renkleri ile ilgileniyorduk (resimdeki
kisinin uzerindeki kiyafetin renkleri) ve bu sebeple deri renklerini ve o
bolgeleri resimde saptamak gerekti. Bizim de onceden aklimizda kalan bir
tembih vardi, Columbia Universitesi'nde yapay ogrenim dersi veren Tony
Jebara bir derste paylasmisti (bu tur gayri resmi, lakirdi seviyesinde
tiyolar cok faydali oluyor), deri rengi bulmak icin bir projesinde tum deri
renklerini R,G,B olarak grafige basmislar, ve beyaz olsun, zenci olsun, ve
sonuc grafikte deri renklerinin cok ince bir bolgede yanyana durdugunu
gormusler. Ilginc degil mi? 


\begin{minted}[fontsize=\footnotesize]{python}
import pandas as pd, zipfile
with zipfile.ZipFile('skin.zip', 'r') as z:
    d =  pd.read_csv(z.open('skin.csv'),sep=',')
print d[:1]
\end{minted}

\begin{verbatim}
   Unnamed: 0   rgbhex   skin        r         g         b         h     s  \
0           0  #200e08  False  0.12549  0.054902  0.031373  0.041667  0.75   

         v  
0  0.12549  
\end{verbatim}

\end{document}
