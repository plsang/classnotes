\documentclass[12pt,fleqn]{article}\usepackage{../common}
\begin{document}



$t$ t Dagilimi (Student's t) ve Cauchy Dagilimi 

$X$, $v$ derece bagimsizlikta $t$ dagilimina sahiptir, ki bu $X \sim t_v$
diye yazilir eger 

$$ f(x) = 
\frac{ \Gamma(v+1)/2)} {\sqrt{v\pi}\Gamma(v/2)}
\bigg(1 + \frac{ x^2}{v}\bigg)^{-(v+1)/2}
 $$

$t$ dagilimi Normal dagilima benzer ama daha kuyrugu daha kalindir. Aslinda
Normal dagilimi $t$ dagiliminin $v = \infty$ oldugu hale tekabul
eder. Cauchy dagilimi da $t$'nin ozel bir halidir, $v = 1$ halidir. Bu
durumda yogunluk fonksiyonu

$$ f(x)  = \frac{ 1}{\pi(1+ x^2)} $$

Bu formul hakikaten bir yogunluk mudur? Kontrol icin entegralini alalim, 

$$ \int _{ -\infty}^{\infty} f(x) dx = 
\frac{ 1}{\pi} \int _{ -\infty}^{\infty} \frac{ dx}{1 + x^2} 
 $$

Cogunlukla entegre edilen yerde  ``1 arti ya da eksi bir seyin karesi''
turunde  bir ifade gorulurse, yerine gecirme (subsitution) islemi
trigonometrik  olarak  yapilir. 

$$  x = \tan \theta, \theta = \arctan x $$

$$ 1 + x^2 = 1 + \tan^2\theta = \sec^2\theta$$

$$ dx / d\theta = \sec^2\theta $$

O zaman 

$$ =
\frac{ 1}{\pi} \int _{ -\infty}^{\infty} \frac{ dx}{1 + x^2}   =
\frac{ 1}{\pi} \int _{ -\infty}^{\infty}  \frac{ 1}{\sec^2\theta}\sec^2\theta d\theta = 
\frac{ 1}{\pi} \int _{ -\infty}^{\infty}  1 \ d\theta = 
 $$

$$ = 
\frac{ 1}{\pi} \theta | _{ -\infty}^{\infty}   = 
\frac{ 1}{\pi} [\arctan(\infty) - \arctan(-\infty)]
 $$

$$ =
\frac{ 1}{\pi} [\frac{ \pi}{2} - (-\frac{ \pi}{2}) ] = 1
 $$


$\chi^2$ Dagilimi

$X$'in $p$ derece serbestlige sahip bir $\chi^2$ dagilima sahip ise $X \sim
\chi^2_p$ olarak gosterilir, yogunluk 

$$ f(x) = \frac{ 1}{\Gamma(p/2) 2^{p/2}} x^{(p/2) - 1} e^{-x/2 }, \ x > 0 $$

Eger $Z_1, .. , Z_p$ bagimsiz standart Normal rasgele degiskenler ise,
$\sum _{ i=1}^{p} Z_p \sim \chi^2_p$ esitligi dogrudur. 


\end{document}
