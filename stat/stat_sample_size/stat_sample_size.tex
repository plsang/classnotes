\documentclass[12pt,fleqn]{article}\usepackage{../common}
\begin{document}
Orneklem Buyuklugu

Bir arastirmaci $n$ bagimsiz deney baz alinarak elde edilen binom
parametresi $p$'yi tahmin etmek istiyor, fakat kac tane $n$ kullanmasi
gerektigini bilmiyor. Tabii ki daha buyuk $n$ degerleri daha iyi sonuclar
verecektir, her deneyin bir masrafi vardir. Bu iki gereklilik nasil birbiri
ile uzlastirilir?




\end{document}
