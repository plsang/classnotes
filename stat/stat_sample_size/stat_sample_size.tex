\documentclass[12pt,fleqn]{article}\usepackage{../common}
\begin{document}
Orneklem Buyuklugu

Bir arastirmaci $n$ bagimsiz deney baz alinarak elde edilen binom
parametresi $p$'yi tahmin etmek istiyor, fakat kac tane $n$ kullanmasi
gerektigini bilmiyor. Tabii ki daha buyuk $n$ degerleri daha iyi sonuclar
verecektir, ama her deneyin bir masrafi vardir. Bu iki gereklilik nasil
birbiri ile uzlastirilir?

Yeterli olacak en az kesinligi, duyarliligi (precision) bulmak icin Z
transformasyonu kullanilabilir belki. Diyelim ki $p$ icin maksimum olurluk
tahmini olan $X/n$'in en azindan $100(1-\alpha)\%$ olasilikta $p$'nin $d$
kadar yakininda olmasini istiyoruz. O zaman alttaki denklemi tatmin eden en
ufak $n$'i buldugumuz anda problemimizi cozduk demektir, 

$$ P\bigg( -d \le \frac{X}{n} - p \le d \bigg)  = 1-\alpha
\mlabel{1}
$$

Tahmin edici $X/n$'nin kendisi de bir rasgele degiskendir. Bu degisken
normal olarak dagilmistir, cunku $X$ Binom olarak dagilmis ise, bu dagilim
ayri Bernoulli dagilimlarinin toplamina esittir. Toplamin aritmetik
ortalamasi ise Merkezi Limit Kanunu'na gore normal'e yaklasir. O zaman,
standardize etmek icin $X/n$'den beklentiyi cikartip standart sapmaya
bolmek gerekir. Beklenti zaten cikartilmis durumda (sansa bak!),
beklentinin ne oldugunu kontrol edelim tabii, ezbere yapmayalim bu isi,
eger her Bernoulli'yi $X_i$ olarak temsil edersek,

$$ X = X_1 + .. + X_n $$

$$ X/n = 1/n(X_1 + .. + X_n )$$

$$ E[X/n] = E[1/n(X_1 + .. + X_n )]$$

$$  = 1/nE[(X_1 + .. + X_n )]$$

$$  = n(1/n) p = p$$

Varyans icin

$$ Var(X/n) = \frac{1}{n^2}Var(X) = \frac{1}{n^2}np(1-p)=
\frac{1}{n}p(1-p) 
$$

Binom dagilimlar icin $Var(X) = np(1-p)$ oldugunu biliyoruz. Standart sapma
ustteki ifadenin karekoku, yani

$$ Std(X/n) = \sqrt{p(1-p)/n}
$$

Simdi standardize edelim,

$$ P\bigg( 
\frac{-d}{\sqrt{p(1-p)/n}} \le 
\frac{\frac{X}{n} - p }{\sqrt{p(1-p)/n}}\le 
\frac{d}{\sqrt{p(1-p)/n}} 
\bigg)  = 
1-\alpha$$


$$ P\bigg( 
\frac{-d}{\sqrt{p(1-p)/n}} \le 
Z
\frac{d}{\sqrt{p(1-p)/n}} 
\bigg)  = 
1-\alpha$$

Daha onceki z-skoru iceren esitsizlikleri hatirlarsak, ustteki ifade 

$$ \frac{d}{\sqrt{p(1-p)/n}} = z_{\alpha/2} 
$$

O zaman 

$$ \frac{z_{\alpha/2}^2p(1-p)}{d^2} = n $$

Fakat bu bir nihai sonuc olamaz, cunku $n$, $p$'nin bir fonksiyonun haline
geldi ve $p$ bilinmeyen bir deger. Fakat biliyoruz ki $0 \le p \le 1$, ve
$p(1-p) \le \frac{1}{4}$. 

Bunu kontrol edelim, $p(1-p)$ hangi $p$'de maksimize olur? $p$'ye gore
turev aliriz, sifira esitleriz, $(p-p^2)' = 1 - 2p = 0, p=1/2$. Ve hesabi
yaparsak, $1/2(1-1/2)=1/4$. Demek ki $p(1-p)$ degeri $1/4$'ten daha buyuk
olamaz. Buna gore, ustteki formule $p(1-p)$ yerine onun olabilecegi en
buyuk degeri koyarsak, 

$$ \frac{z_{\alpha/2}^21/4}{d^2} = n $$

$$ n = \frac{z_{\alpha/2}^2}{4d^2} $$


Ornek

Buyuk bir sehirde cocuklarin kacta kacinin asisini almis olup olmadigini
anlamak icin bir anket gerceklestirilecek. Anketi duzenleyenler orneklem
orani olan $X/n$'in en az 98\% oranda gercek oran $p$'nin 0.05 yakininda
olmasini istiyorlar. Orneklem ne kadar buyuk olmalidir? 

Burada $100(1-\alpha) = 98$, o zaman $\alpha = 0.02$, demek ki $z_{alpha/2}
= z_{0.02/2} = z_{0.01}$ 
degerine ihtiyacimiz var. Python ile

\begin{minted}[fontsize=\footnotesize]{python}
from scipy.stats.distributions import norm
print norm.ppf(0.99)
\end{minted}

\begin{verbatim}
2.32634787404
\end{verbatim}

Tum hesap icin

$$ n = \frac{(2.33)^2}{4(0.05)^2} = 543$$

Demek ki kabul edilebilir en ufak deger 543. 


\end{document}
