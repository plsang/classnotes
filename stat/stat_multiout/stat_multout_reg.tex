\documentclass[12pt,fleqn]{article}\usepackage{../common}
\begin{document}
Cok Boyutlu Regresyon ile Tavsiye

Tavsiye sistemlerinde eger tercih edilen nesnelerin sayisi cok fazla
degilse ilginc bir metot cok boyutlu cikti regresyonu (multioutput
regression -MR-) kullanmak. Tavsiye sistemleri bilindigi gibi cogunlukla
bircok mumkun secenegi siralayarak en iyi secimleri en uste koymaya
ugrasir, ve yine, cogunlukla bu sistemleri egitirken her secenek icin tek
bir etiket (begenildi / begenilmedi) ile o etiketin bagli oldugu ``kaynak''
verisi arasinda iliski kurulmaya ugrasilir, ve her secenek icin ayri bir
obje egitilir (SVD bir istisna, bu metot ayni anda tum kullanicilar ve tum
secimleri gozonune alabilir).

MR kullanarak cok boyutlu kaynak verisiyle yine cok boyutlu cikti arasinda
{\em ayni anda} bir iliski kurabiliriz, ve daha onemlisi bu iliskiyi esnek
(soft) bir sekilde yapabiliriz. Ayrica MR girdi ve cikti arasinda iliski
kurdugu gibi, ciktilar arasinda da baglantilari bulabilir, eger her secenek
icin ayri ayri tavsiye objeleri kullansaydik bu iliskiyi
ogrenemezdik. Mesela 40 yas ustu Arizona'dan gelen kisiler aksiyon filmi
seviyor, ve ayni kisiler yine komedi filmi seviyor. MR aksiyon ile komedi
ciktilari arasindaki iliskiyi ogrenebilir.

Isler bir MR metotunun ornegini [2] yazisinda bulabiliriz, bu yazida karar
agaci (decision tree) kullaniliyor, \verb!scikit-learn! paketi karar
agaclarinin ciktinin cok olabilmesini destekliyor. [2]'deki ornegin amaci
bir cemberi ogrenmek mesela; cember soyle olusturulmus, kaynak tek bir
boyut - sirali sayilar, cikti ise iki boyutlu, $cos(x)$ digeri $sin(x)$. Bu
dogal olarak bir cember ortaya cikartacaktir. [2]'de goruldugu gibi MR bu
cemberi ogreniyor.

Ayni teknigi mesela Movielens 1M verisinde genre ogrenmek icin
kullanabiliriz. Kaynak musterinin kisisel bilgileri, hedef ise cok boyutlu
0/1 olarak kodlanmis {\em tum genre'ler} olacak. Eger bir kisi hem komedi
hem aksiyon sevmis ise matrisin o kisiye tekabul eden satirinda aksiyon ve
komedi kolonu 1 olacak mesela. Bu sekilde tum musteriler kodlanacak, ve
kaynak olarak kisisel veri alinacak. 

Bu veriyi {\em Pivotlama} yazisinin dizininde bulabilirsiniz, zip dosyasini
acmaniz yeterli. 

\begin{minted}[fontsize=\footnotesize]{python}
import pandas as pd
cols = ['user_id', 'movie_id', 'rating', 'timestamp']
ratings =  pd.read_csv('../stat_pandas_ratings/ratings.dat', sep='::',
           header=None,names=cols)
cols = ['movie_id', 'title', 'genres']
movies =  pd.read_csv('../stat_pandas_ratings/movies.dat',sep='::',
          header=None,names=cols)
cols = ['user_id', 'gender', 'age', 'occupation', 'zip']
users = pd.read_csv('../stat_pandas_ratings/users.dat', sep='::', 
        header=None,names=cols)
\end{minted}

\begin{minted}[fontsize=\footnotesize]{python}
genre_iter = (set(x.split('|')) for x in movies.genres)
genres = sorted(set.union(*genre_iter))
dummies = pd.DataFrame(np.zeros((len(movies), len(genres))), columns=genres)
for i, gen in enumerate(movies.genres):
   dummies.ix[i, gen.split('|')] = 1
movies_windic = movies.join(dummies.add_prefix('Genre_'))
movies_windic = movies_windic.drop(['title','genres'],axis=1)
\end{minted}

\begin{minted}[fontsize=\footnotesize]{python}
joined = ratings.merge(movies_windic, left_on='movie_id',right_on='movie_id')
\end{minted}

\begin{minted}[fontsize=\footnotesize]{python}
y = joined.groupby('user_id').sum()
y = y.drop(['movie_id','rating','timestamp'],axis=1)
y[y > 0.0] = 1.0
\end{minted}

Ciktiyi kodlarken Movielens verisinde bir kisi bir filme herhangi bir not
vermis ise o filmin genre'sini ``tercih edilir'' olarak 1 ile
isaretledik. Tabii ki o film icin not dusuk olabilir (ve bu durumu bir
filtre kosulu olarak kullanabilirdik), ama bu, o kisinin o belirli filmi
begenmedigini gosterir, o filmi sectiyse musteri buyuk bir ihtimalle o
genre'yi seviyordur. Bu hipotezi test edebilirsiniz, \verb!y[y > 0.0]!
yerine \verb!y[y > 0.3]!  secilirse basari orani dusuyor. Yani bize faydali
olacak degerli verileri ikinci filtre ile kaybediyoruz.

\begin{minted}[fontsize=\footnotesize]{python}
from sklearn.feature_extraction import DictVectorizer
def one_hot_dataframe(data, cols):
    vec = DictVectorizer()
    mkdict = lambda row: dict((col, row[col]) for col in cols)
    tmp = vec.fit_transform(data[cols].to_dict(outtype='records')).toarray()
    vecData = pd.DataFrame(tmp)
    vecData.columns = vec.get_feature_names()
    vecData.index = data.index
    data = data.drop(cols, axis=1)
    data = data.join(vecData)
    return data

X = users.copy()
X['occupation2'] = users['occupation'].map(lambda x: str(x))
X['zip2'] = users['zip'].map(lambda x: str(x)[0])
X['zip3'] = users['zip'].map(lambda x: str(x)[:2])
X = one_hot_dataframe(X,['occupation2','gender','zip2','zip3'])
X = X.drop(['occupation','zip'],axis=1)
X = X.set_index('user_id')
X = X.ix[y.index]
X = X.reindex(y.index)
print X.shape, y.shape
\end{minted}

\begin{verbatim}
(6040, 134) (6040, 18)
\end{verbatim}

\begin{minted}[fontsize=\footnotesize]{python}
from sklearn.metrics import roc_curve, auc
from sklearn.metrics import roc_auc_score
from sklearn.cross_validation import train_test_split
from sklearn.tree import DecisionTreeRegressor
from sklearn.ensemble import RandomForestRegressor
from sklearn.linear_model import Lasso, Ridge, LinearRegression

x_train, x_test, y_train, y_test = train_test_split(X, y, test_size=1000)
clf = RandomForestRegressor(max_depth=3,n_estimators=5)

clf.fit(x_train,y_train)
y_pred = clf.predict(x_test)

fpr, tpr, thresholds = roc_curve(np.ravel(y_test), np.ravel(y_pred))
roc_auc = auc(fpr, tpr)
print 'Tree AUC', roc_auc

imps = pd.Series(list(clf.feature_importances_),index=X.columns)
imps = imps.order(ascending=False).head(20)
print 'important features'
print np.array(imps.index)
\end{minted}

\begin{verbatim}
Tree AUC 0.844777794213
important features
['gender=F' 'age' 'gender=M' 'occupation2=10' 'zip3=46' 'zip2=3' 'zip3=37'
 'zip3=65' 'zip3=31' 'zip3=33' 'zip3=41' 'zip3=39' 'zip3=67' 'zip3=01'
 'zip3=20' 'zip3=24' 'zip3=95' 'zip3=08' 'zip3=09' 'zip3=03']
\end{verbatim}

Test etmek icin AUC olcusunu kullandik. Bir puruz vardi, AUC hem tahmin hem
de test etiketlerini tek boyutlarda alir, bizim ciktimiz cok boyutlu
idi. Ne yapmali?  Biz de her iki matrisi (test etiketleri, tahminler)
``duzlestirdik'', tek boyut haline getirdik. Nasil olsa her iki matris ayni
sekilde duzlestirildigi ve boyutlari zaten ayni oldugu icin sonuc iki
vektor de ayni boyutta oldu. Ve bu iki vektor uzerinde AUC hesabi
yaptik. Sonuc ustte goruldugu gibi \%84.

Ayrica \verb!RandomForestRegressor! objesi regresyonun onemli buldugu
kaynak ogelerini onem sirasina gore gosterebilir. Ustte cinsiyet (gender),
yas (age) genre seciminde onemli gorunuyor. Bu akla yatkin. Ayrica
meslekler (occupation) arasinda 10 kodlu olan onemliymis, bu kodun
ilkogretim ogrencisine tekabul ettigini gorduk. Ilginc.

Karsilastirmak icin basit bir alternatif kodladik, tum begenilerin
ortalamasini aldik, ve bu senaryoada tahmin olarak herkes icin surekli ayni
tahmini yaptik. Buradan gelen sonuc \%83. 

\begin{minted}[fontsize=\footnotesize]{python}
y_naive = np.array(y_train).mean(axis=0)
y_naive = pd.DataFrame([y_naive], index=range(y_test.shape[0]), 
                       columns=list(y.columns))
fpr, tpr, thresholds = roc_curve(np.ravel(y_test), np.ravel(y_naive))
roc_auc = auc(fpr, tpr)
print 'Naive AUC', roc_auc
\end{minted}

\begin{verbatim}
Naive AUC 0.833752370725
\end{verbatim}

Evet, bu cok basit bir tahmin
icin yuksek gelebilir, fakat bir Netflix Veri Bilimcisi'nin soyledigi gibi
``populeriteyi yenmek zordur''. Neyse ki ustteki MR metotu populeriteyi
yendi. Ayrica bir sey daha yapti, eger tavsiye sistemi icin populerite
kullansaydik herkes icin surekli ayni tavsiyeyi vermemiz gerekecekti. MR
bazli ``kisiellestirilmis'' metot herkes icin ayri tavsiyeler
verebilecektir.


\url{http://upload.wikimedia.org/wikipedia/commons/2/24/ZIP_Code_zones.svg}

\url{http://scikit-learn.org/stable/auto_examples/tree/plot_tree_regression_multioutput.html}


\end{document}
