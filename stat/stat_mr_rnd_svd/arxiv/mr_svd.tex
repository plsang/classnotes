\documentclass[11pt]{article} 

\usepackage[latin5]{inputenc}
\usepackage{amsmath}
\usepackage{examplep}
\usepackage{graphicx}
\usepackage[ruled]{algorithm2e}
\SetKw{Emit}{emit}
\usepackage[bookmarks=false]{hyperref}

\begin{document}
\title{SVD Factorization for Tall-and-Fat Matrices on Map/Reduce
  Architectures} \author{Burak Bayraml�} \date{October 15, 2013}

\maketitle

\begin{abstract}
  We demonstrate an implementation for an approximate rank-k SVD
  factorization, combining well-known randomized projection techniques with
  previously implemented map/reduce solutions in order to compute steps of
  the random projection based SVD procedure, such QR and SVD. We structure
  the problem in a way that it reduces to Cholesky and SVD factorizations
  on $k \times k$ matrices computed on a single machine, greatly easing the
  computability of the problem.
\end{abstract}

\section{Introduction} \label{intro}

\cite{gleich} presents many excellent techniques for utilizing map/reduce
architectures to compute QR and SVD for the so-called tall-and-skinny
matrices. QR factorization is turned into an $A^TA$ computation problem to
be computed in parallel using map/reduce, and its key element the Cholesky
decomposition, can be performed on a single machine. Since

$$ A^TA = (QR)^T(QR) = R^TQ^TQR = R^TR $$

and because Cholesky factorization of an $n \times n$ symmetric positive
definite matrix is

$$ A = LL^T $$

where $L$ is an $n \times n$ lower triangular matrix, and R is upper
triangular, we can conclude if we factorize $A$ into $L$ and $L^T$, this
implies $LL^T = RR^T$, we have a method of calculating $R$ of QR using
Cholesky factorization on $A^TA$. The key observation here is $A^TA$
computation results an $n \times n$ matrix and if $A$ is ``skinny'' then
$n$ is relatively small (in the thousands), then Cholesky decomposition can
be executed on a small $n \times n$ matrix on a single computer utilizing
an already available function in a scientific computing library. $Q$ is
computed simply as $Q = AR^{-1}$. This again is relatively cheap because R
is $n \times n$, the inverse is computed locallly, matrix multiplication
with $A$ can be performed through map/reduce.

SVD is an additional step. SVD decomposition is 

$$ A = U \Sigma V^T $$

If we expand it with $A = QR$

$$ QR =  U \Sigma V^T $$

$$ R =  Q^T U \Sigma V^T $$

Let's call $\tilde{U} = Q^T U$

$$ R =  \tilde{U} \Sigma V^T $$

This means if we run a local SVD on $R$ (we just calculated above with
Cholesky) which is an $n \times n$ matrix, we will have calculated
$\tilde{U}$, the real $\Sigma$, and real $V^T$. 

Now we have a map/reduce way of calculating QR and SVD on $m \times n$
matrices where $n$ is small.

\subsection{Approximate rank-k SVD}

Switching gears, we look at another method for calculating SVD. The
motivation is computing SVD if $n$ is large, creating a ``fat'' matrix
which might have columns in the billions would require reducing the
dimensionality of the problem. According to \cite{halko}, one way to
achieve is through random projection. First we draw an $n \times k$
Gaussian random matrix $\Omega$. Then we calculate

$$ Y = A \Omega $$

We perform QR decomposition on $Y$

$$ Y = QR $$

Then form $k \times n$ matrix

$$ B = Q^T A \label{bt} $$

Then we can calculate SVD on this small matrix

$$ B = \hat{U} \Sigma V^T $$

Then form the matrix 

$$ U = Q \hat{U} $$

The main idea is based on

$$ A = QQ^T A $$

if replace $Q$ which comes from random projection $Y$, 

$$ A \approx \tilde{Q}\tilde{Q}^T A $$

$Q$ and $R$ of the projection are close to that of $A$. In the
multiplication above $R$ is called $B$ where $B = \tilde{Q}^T A $, and,

$$ A \approx \tilde{Q}B $$

then, as in \cite{gleich}, we can take SVD of $B$ and apply the same
transition rules to obtain an approximate $U$ of $A$.

This approximation works because of the fact that projecting points to a
random subspace preserves distances between points, or in detail,
projecting the n-point subset onto a random subspace of $O(\log
n/\epsilon^2)$ dimensions only changes the interpoint distances by $(1 \pm
\epsilon)$ with positive probability \cite{gupta}. It is also said that $Y$
is a good representation of the span of $A$.

\subsection{Combining Both Methods}

Our idea was using approximate k-rank SVD calculation steps where $k << n$,
and using map/reduce based QR and SVD methods to implement those steps. By
utilizing random projection, we would be able to work in a smaller
dimension which would translate to local Cholesky, and SVD calls on $k
\times k$ matrices that can be performed in a speedy manner. Below we
outline each map/reduce job.

\vspace{1em}

\begin{algorithm}[H]
  \DontPrintSemicolon  
  \SetKwInOut{Input}{input}
  \SetKwInOut{Output}{output}
  \SetKwProg{map}{function}{}{}
  \Input{A}
  \Output{Y}
  \map{MAP{(key, value)}}{
    Tokenize $value$ and pick out id value pairs\;
    result $\leftarrow$ zeros(1,$k$)\;
    \For{each $j^{th}$ $token$ $\in value$}{
      Initialize seed with $j$\;
      r $\leftarrow$ generate $k$ random numbers\;
      result $\leftarrow$ result + $r \cdot token[j]$
    }
    \Emit{key, result}
  }
  \SetKwProg{reduce}{function}{}{}
  \reduce{REDUCE{(key, value)}}{
    noop\;}
  \caption{Random Projection Job}
\end{algorithm} 

\vspace{1em}

Each value of $A$ will arrive to the algorithm as a key and value pair. Key
is line number or other identifier per row of $A$. Value is a collection of
id value pairs where id is column id this time, and value is the value for
that column. Sparsity is handled through this format, if an id for a column
does not appear in a row of A, it is assumed to be zero. The resulting $Y$
matrix has dimensions $m \times k$.

\vspace{1em}

\begin{algorithm}[H]
  \DontPrintSemicolon  
  \SetKwInOut{Input}{input}
  \SetKwInOut{Output}{output}
  \SetKwProg{map}{function}{}{}
  \Input{Y}
  \Output{R}
  \map{MAP{(key $k$, val $a$)}}{
    \For{$i,row$ in enumerate($a^Ta$)}{
      \Emit{i, row}
    }
  }
  \SetKwProg{reduce}{function}{}{}
  \reduce{REDUCE{(key, value)}}{
    \Emit (k,sum($<v_j^k>$)\;}
  \SetKwProg{reduce}{function}{}{}
  \reduce{FINAL LOCAL REDUCE {(key, value)}}{
    result $\leftarrow$ Cholesky($A_{sum}$)\;
    \Emit (result)\;}
  \caption{$A^TA$ Cholesky Job}
\end{algorithm} 

\vspace{1em}

The FINAL\_LOCAL\_REDUCE step is a function provided in most map/reduce
frameworks, it is a central point that collects the output of all reducers,
naturally a single machine which makes it ideal to execute the final
Cholesky call on by now a very small ($k \times k$) matrix. The output is
$R$.

\vspace{1em}

\begin{algorithm}[H]
  \DontPrintSemicolon  
  \SetKwInOut{Input}{input}
  \SetKwInOut{Output}{output}
  \SetKwProg{map}{function}{}{}
  \SetKwProg{initialize}{function}{}{}
  \Input{Y,R}
  \Output{Q}
  \initialize{INIT{()}}{ 
    $R_{inv} = R^{-1}$ 
  }
  \map{MAP{(key, value)}}{
    \For{$row$ in $Y$}{
      \Emit{(key, $row \cdot R_{inv}$)}
    }
  }
  \caption{$Q$ Job}
\end{algorithm} 

\vspace{1em}

There is no reducer in the Q Job, it is a very simple procedure, it merely
computes multiplication between row of $Y$ and a local matrix $R$. Matrix
$R$ is very small, $k \times k$, hence it can be kept locally in every
node. The $INIT$ function is used to store the inverse of $R$ locally, once
the mapper is initialized, it will always use the same $R^{-1}$ for every
multiplication. 

\vspace{1em}

\begin{algorithm}[H]
  \DontPrintSemicolon  
  \SetKwInOut{Input}{input}
  \SetKwInOut{Output}{output}
  \Input{AQ}
  \Output{$B^T$}
  \SetKwProg{map}{function}{}{}
  \map{MAP {(key, value)}}{    
    $left=row$ from $A$ \;
    $right=row$ from $Q$ \;
    \For{nonzero $j^{th}$ $cell$ in $left$}{
      \Emit{$j$, $left[j] \cdot right$ }
    }
  }
  \SetKwProg{reduce}{function}{}{}
  \reduce{REDUCE {(key, value)}}{
    result $\leftarrow$ zeros(1,$k$) \;
    \For{$row$ in $value$}{
      result $\leftarrow$ result + $row$
     }
     \Emit{key, result}
  }
  \caption{$A^TQ$ Job}
\end{algorithm} 

\vspace{1em}

The job above takes an $AQ$ matrix which is assumed to be a join between
$A$ and $Q$, per row, based on key. We split the row and deduce the $A$
part and the $Q$ part, then iterate cells of $A$ one by one, which is
assumed to be sparse, and multiply the entire row of $Q$. Then for each
$j^{th}$ non-zero cell of $A$, we multiply this value with the row from $Q$
and emit the multiplication result with key $j$.

This job's formula in \ref{bt} is described $Q^TA$. For implementation
purposes we changed this formula into

$$ B^T = A^TQ $$

because as output we needed to have a $n \times k$ matrix instead of a $k
\times n$ one, which would allow us to use map/reduce SVD that translates
into a local Cholesky and SVD on $k \times k$ matrices. Since we take SVD
of $B^T$ instead of $B$, that changes the output as well, 

$$ B = U\Sigma V^T $$

becomes

$$ B^T = V\Sigma U^T $$

In other words, in order to obtain $U$ of $B$, we need to take
$(U_{BT}^T)^T$ from the SVD of $B^T$. This is how $A^TA$ Cholesky Job is
called, this time with $B^T$ as its input data.

\vspace{1em}

\begin{algorithm}[H]
  \DontPrintSemicolon  
  \SetKwInOut{Input}{input}
  \SetKwInOut{Output}{output}
  \SetKwProg{map}{function}{}{}
  \SetKwProg{initialize}{function}{}{}
  \Input{Q,R}
  \Output{U}
  \initialize{INIT{()}}{ 
    $\tilde{U}$ = svd of $R$ 
  }
  \map{MAP{(key, value)}}{
    \For{$row$ in $Q$}{
      \Emit{(key, $row \cdot \tilde{U}$)}
    }
  }
  \caption{$Q\tilde{U}$ Job}
\end{algorithm} 

\vspace{1em}

The order of execution for everything is as follows: 

\vspace{1em}

\begin{algorithm}[H]
  \DontPrintSemicolon  
  Y = Random Projection Job (A)\;
  $R_{Y}$ = $A^TA$ Cholesky Job(Y)\;
  $Q_{Y}$ = Q Job\;
  $B^T$ = $A^TQ$ Job\;
  $R_{BT}$ = $A^TA$ Cholesky Job($B^T$)\;
  $U$ = $Q\tilde{U}$ Job($R_{BT},Q$)\;
  \caption{Map/Reduce SVD}
\end{algorithm} 


\subsection{Discussion}

We performed our experiments on the Netflix dataset which has about 100
million from over 480,000 customers on 17770 movies. The implementation was
programmed on Sasha distributed framework \cite{bayramli1}, and SVD
calculation on the full dataset with $k=7$ on two notebook computers,
utilizing in total 6 cores took 20 minutes. Scipy SVD calculation on the
same dataset is much faster, however, map/reduce environment is a streaming
environment which needs to process in record-by-record basis. Fortunately
exactly for the same reasons a map/reduce algorithm can scale horizontally,
being able to process records in the billions proportional to the number of
nodes in a cluster whereas single machine algorithm would not. All code
relevant for this paper can be found here \cite{bayramli2}.

There are only two passes necessary on the full dataset, and three passes
on $m$ rows but with reduced $k$ dimensions this time.  Perhaps
predictably, the procedure spends most of its time at $A^TQ$ Job. This step
performs not only a join between $A$ and $Q$, it also emits $k$ cells per
non-zero value of $A$'s rows, then creates partial sums these $k$ vectors
creating $n \times k$ result. If for simplicity we assume $k$ non-zero
cells in each $A$ row, the complexity of this step would be $O(mk)$.

\begin{thebibliography}{1}

\bibitem{gleich}
Gleich, Benson, Demmel, \emph{Direct QR factorizations for tall-and-skinny
  matrices in MapReduce architectures}, {\tt arXiv:1301.1071 [cs.DC]}, 2013

\bibitem{halko}
N.~Halko, \emph{Randomized methods for computing low-rank approximations of
  matrices}, University of Colorado, Boulder, 2010

\bibitem{gupta}
S.~Dangupta, A.~Gupta \emph{An Elementary Proof of a Theorem of Johnson and
  Lindenstrauss}, Wiley Periodicals, 2002

\bibitem{kurucz}
M.~Kurucz, A. A.~Bencz�r, K.~Csalog�ny, \emph{Methods for large scale SVD with
missing values}, ACM, 2007

\bibitem{bayramli1} B.~Bayramli, \emph{Sasha Framework}, 
\url{git@github.com:burakbayramli/sasha.git}
Github, 2013

\bibitem{bayramli2} B.~Bayramli, \emph{Map/Reduce Code for Netflix SVD Analysis},
  \url{https://github.com/burakbayramli/classnotes/tree/master/stat/stat_hadoop_rnd_svd/sasha},
  Github, 2013


\end{thebibliography}

\end{document}
