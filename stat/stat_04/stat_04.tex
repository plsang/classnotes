\documentclass[12pt,fleqn]{article}\usepackage{../common}
\begin{document}
Ders 4

Beklenti (Expectation) 

Bu deger dagilim $f(x)$'in tek sayilik bir ozetidir. Yani beklenti hesabina
bir taraftan bir dagilim fonksiyonu girer, diger taraftan tek bir sayi
disari cikar. Surekli dagilim fonksiyonlari icin $E(X)$

\[  E(X) = \int x f(x) dx\]

ayriksal durumda

\[ E(X) = \sum_x xf(x) \]

olarak hesaplanir. Hesabin, her $x$ degerini onun olasiligi ile carpip
topladigina dikkat. Bu tur bir hesap dogal olarak tum $x$'lerin
ortalamasini verecektir, ve dolayli olarak dagilimin ortalamasini
hesaplayacaktir. Ortalama $\mu_x$ olarak ta gosterilebilir.

Notasyonel basitlik icin ustteki toplam / entegral yerine 

\[ = \int x \ dF(x) \]

diyecegiz, bu notasyonel bir kullanim sadece, unutmayalim, reel analizde
$\int x \ dF(x)$'in ozel bir anlami var (hoca tam diferansiyel $dF$'den
bahsediyor). 

Beklentinin taniminin kapsamli / eksiksiz olmasi icin $E(X)$'in
``mevcudiyeti'' icin de bir sart tanimlamak gerekir, bu sart soyle olsun, 

\[ \int_x |x|dF_X(x) < \infty \]

ise beklenti mevcut demektir. Tersi sozkonusu ise beklenti mevcut
degildir. 


Ornek 

$X \sim Unif(-1,3)$ olsun. $E(X) = \int xdF(x) = \int x f_X(x)dx = \frac{
  1}{4} \int _{ -1}^{3} x dx = 1$. 

Ornek 

Cauchy dagiliminin $f_X(x) = \{ \pi (1+x^2) \} ^{-1}$ oldugunu soylemistik. Simdi 
beklentiyi hesaplayalim. Parcali entegral teknigi lazim, $u=x$, 
$dv =
1/1+x^2$ deriz, ve o zaman $v = \tan ^{-1}(x)$ olur, bkz {\em Ters
  Trigonometrik Formuller} yazimiz. Demek ki 

\[ \int |x|dF(x) = \frac{ 2}{\pi} \int _{ 0}^{\infty}\frac{x dx}{1+x^2}  \]

2 nereden cikti? Cunku $|x|$ kullaniyoruz, o zaman sinir degerlerinde
sadece sifirin sagina bakip sonucu ikiyle carpmak yeterli, 2 sayisi buradan
geliyor. Bir sabit oldugu icin $\pi$ ile beraber disari cikiyor. Simdi 

\[ \int udv = uv - \int vdu \]
 
uzerinden

\[ = [x \tan ^{-1}(x) ] _{ 0}^{\infty} - \int _{ 0}^{\infty} \tan ^{-1}(x)
dx  = \infty\]

Yani ustteki hesap sonsuzluga gider. O zaman ustteki tanimimiza gore Cauchy
dagiliminin beklentisi yoktur. 











\end{document}
