\documentclass[12pt,fleqn]{article}\usepackage{../common}
\begin{document}

\begin{minted}[fontsize=\footnotesize]{python}
occup_map = \
{ 0:  "other" or not specified,1:  "academic/educator",
  2:  "artist",3:  "clerical/admin",
  4:  "college/grad student",5:  "customer service",
  6:  "doctor/health care",7:  "executive/managerial",
  8:  "farmer",9:  "homemaker",
  10:  "K-12 student", 11:  "lawyer",
  12:  "programmer",13:  "retired",
  14:  "sales/marketing",15:  "scientist",
  16:  "self-employed",17:  "technician/engineer",
  18:  "tradesman/craftsman",19:  "unemployed",
  20:  "writer"}

zip_map = \
{ 0: 'Northeast', 1: 'NY Area', 2: 'DC', 3: 'Florida', 4: 'Michigan/Ohio', 
  5: 'North', 6: 'Illinois', 7: 'Texas / Arkansas', 8: 'Nevada / Utah', 
  9: 'California / Alaska'}
\end{minted}


\begin{minted}[fontsize=\footnotesize]{python}
import pandas as pd
cols = ['user_id', 'gender', 'age', 'occupation', 'zip']
users = pd.read_csv('../stat_pandas_ratings/users.dat', sep='::', 
        header=None,names=cols)

from sklearn.feature_extraction import DictVectorizer
def one_hot_dataframe(data, cols):
    vec = DictVectorizer()
    mkdict = lambda row: dict((col, row[col]) for col in cols)
    tmp = vec.fit_transform(data[cols].to_dict(outtype='records')).toarray()
    vecData = pd.DataFrame(tmp)
    vecData.columns = vec.get_feature_names()
    vecData.index = data.index
    data = data.drop(cols, axis=1)
    data = data.join(vecData)
    return data

df = users.copy()
df['occupation'] = df.apply(lambda x: occup_map[x['occupation']], axis=1)
df['zip2'] = users['zip'].map(lambda x: int(str(x)[0]))
df['zip2'] = df.apply(lambda x: zip_map[x['zip2']], axis=1)
df = one_hot_dataframe(df,['occupation','gender','zip2'])
df = df.drop(['zip'],axis=1)
df = df.set_index('user_id')
\end{minted}

\begin{minted}[fontsize=\footnotesize]{python}
cols = ['user_id', 'movie_id', 'rating', 'timestamp']
ratings =  pd.read_csv('../stat_pandas_ratings/ratings.dat', sep='::',
           header=None,names=cols)
cols = ['movie_id', 'title', 'genres']
movies =  pd.read_csv('../stat_pandas_ratings/movies.dat',sep='::',
          header=None,names=cols)

genre_iter = (set(x.split('|')) for x in movies.genres)
genres = sorted(set.union(*genre_iter))
dummies = pd.DataFrame(np.zeros((len(movies), len(genres))), columns=genres)
for i, gen in enumerate(movies.genres):
   dummies.ix[i, gen.split('|')] = 1
movies_windic = movies.join(dummies.add_prefix('Genre_'))
movies_windic = movies_windic.drop(['title','genres'],axis=1)
joined = ratings.merge(movies_windic, left_on='movie_id',right_on='movie_id')
genres = joined.groupby('user_id').sum()
genres = genres.drop(['movie_id','rating','timestamp'],axis=1)
X = pd.merge(df, genres, left_index=True, right_index=True,how='left')
print X.shape
\end{minted}

\begin{verbatim}
(6040, 52)
\end{verbatim}
\begin{minted}[fontsize=\footnotesize]{python}
fout = open('/tmp/featmap.txt','wb')
for i,col in enumerate(X.columns):
    if  'age'==col: fout.write('%d\t%s\tq\n' % (i,col))
    else: fout.write('%d\t%s\ti\n' % (i,col))    
fout.close()
\end{minted}

\begin{minted}[fontsize=\footnotesize]{python}
print list(X.columns)
from sklearn.preprocessing import normalize
import scipy.sparse.linalg as slin
import scipy.sparse as sps, numpy.random as rand
X2 = sps.csr_matrix(X.fillna(0))
X2 = normalize(X2, norm='l2', axis=0).tocsr()
X2 = normalize(X2, norm='l2', axis=1).tocsr()    
rand.seed(1000)
Omega = sps.csr_matrix(rand.randn(X2.shape[1],4))
u = X2.dot(Omega)
\end{minted}

\begin{verbatim}
['age', 'gender=F', 'gender=M', 'occupation=K-12 student', 'occupation=academic/educator', 'occupation=artist', 'occupation=clerical/admin', 'occupation=college/grad student', 'occupation=customer service', 'occupation=doctor/health care', 'occupation=executive/managerial', 'occupation=farmer', 'occupation=homemaker', 'occupation=lawyer', 'occupation=other', 'occupation=programmer', 'occupation=retired', 'occupation=sales/marketing', 'occupation=scientist', 'occupation=self-employed', 'occupation=technician/engineer', 'occupation=tradesman/craftsman', 'occupation=unemployed', 'occupation=writer', 'zip2=California / Alaska', 'zip2=DC', 'zip2=Florida', 'zip2=Illinois', 'zip2=Michigan/Ohio', 'zip2=NY Area', 'zip2=Nevada / Utah', 'zip2=North', 'zip2=Northeast', 'zip2=Texas / Arkansas', 'Genre_Action', 'Genre_Adventure', 'Genre_Animation', "Genre_Children's", 'Genre_Comedy', 'Genre_Crime', 'Genre_Documentary', 'Genre_Drama', 'Genre_Fantasy', 'Genre_Film-Noir', 'Genre_Horror', 'Genre_Musical', 'Genre_Mystery', 'Genre_Romance', 'Genre_Sci-Fi', 'Genre_Thriller', 'Genre_War', 'Genre_Western']
\end{verbatim}

\begin{minted}[fontsize=\footnotesize]{python}
n_clusters=5
from sklearn.cluster import KMeans
print u.shape
clf = KMeans(n_clusters=n_clusters)
clf.fit(u)    
df2 = X.reset_index()
df2['cluster'] = clf.labels_
df2.to_csv('/tmp/customers_clustered.csv',sep=';',index=None)
\end{minted}

\begin{verbatim}
(6040, 4)
\end{verbatim}

\begin{minted}[fontsize=\footnotesize]{python}
from sklearn.metrics import roc_curve, auc
from sklearn.cross_validation import train_test_split
import os.path,os,logging,datetime,pytz,re, sys
sys.path.append('%s/Downloads/xgboost/wrapper/' % os.environ['HOME'])
import xgboost as xgb

df3 = pd.read_csv('/tmp/customers_clustered.csv',sep=';',index_col='user_id')
X3 = df3.drop('cluster',axis=1)
print X3.shape

aucs = []
for i in range(n_clusters):
    y = (df3['cluster'] == i).astype(float)    
    X_train, X_test, y_train, y_test = train_test_split(X3, y, test_size=0.10, random_state=0)
    xg_train = xgb.DMatrix(X_train, label=y_train)
    xg_test = xgb.DMatrix(X_test, label=y_test)    
    watchlist = [ (xg_train,'train'), (xg_test, 'test') ]    
    param = {}; 
    num_round = 10
    param['silent'] = 1
    param['eval_metric'] = 'auc'
    param['max_depth'] = 4
    bst = xgb.train(param, xg_train, num_round, watchlist )
    pred = bst.predict(xg_test)
    fpr, tpr, thresholds = roc_curve(y_test, pred)
    roc_auc = auc(fpr, tpr)
    print 'cluster', i, roc_auc, np.sum(y)
    aucs.append(roc_auc)
    bst.dump_model('/tmp/tree-%d.txt' % i,'/tmp/featmap.txt')
print 'mean auc', np.array(aucs).mean()
\end{minted}

\begin{verbatim}
(6040, 52)
cluster 0 0.983958333333 1007.0
cluster 1 0.95119010405 1461.0
cluster 2 0.935070470561 1115.0
cluster 3 0.948211029163 1348.0
cluster 4 0.978581267218 1109.0
mean auc 0.959402240865
\end{verbatim}

\begin{minted}[fontsize=\footnotesize]{python}
import clusters
clusters.describe(n_clusters)
\end{minted}

\begin{minted}[fontsize=\footnotesize]{python}
import pandas as pd
dff = pd.read_csv('/tmp/customers_clustered.csv',sep=';',index_col='user_id')
print len(dff)
\end{minted}

\begin{verbatim}
6040
\end{verbatim}

\begin{minted}[fontsize=\footnotesize]{python}
print len( dff[\
(dff['occupation=college/grad student'] == 1) & \
(dff['occupation=technician/engineer'] == 0) & \
(dff['zip2=DC'] == 0) ] )
\end{minted}

\begin{verbatim}
700
\end{verbatim}

\begin{minted}[fontsize=\footnotesize]{python}
print len( dff[\
(dff['Genre_Documentary'] == 0) & \
(dff['zip2=NY Area'] == 1) ] )
\end{minted}

\begin{verbatim}
415
\end{verbatim}


\url{http://upload.wikimedia.org/wikipedia/commons/2/24/ZIP_Code_zones.svg}

\url{http://www.zipboundary.com/zipcode_faqs.html}

\end{document}
