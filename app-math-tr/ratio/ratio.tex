\documentclass[12pt,fleqn]{article}
\setlength{\parindent}{0pt}
\usepackage{graphicx}
\usepackage{cancel}
\usepackage{listings}
\usepackage[latin5]{inputenc}
\usepackage{color}
\setlength{\parskip}{8pt}
\setlength{\parsep}{0pt}
\setlength{\headsep}{0pt}
\setlength{\topskip}{0pt}
\setlength{\topmargin}{0pt}
\setlength{\topsep}{0pt}
\setlength{\partopsep}{0pt}
\setlength{\mathindent}{0cm}

\begin{document}
$dy/dx$ bir kesir olarak gorulebilir mi? 

Mesela Thomas Calculus 11th Baski, bolum 3.8 sf 225'te turev $dy/dx$'in bir
kesir, bir oran olmadigi soylenir. Fakat bu sekilde gorulemez mi? Cunku $dy
= f'(x)dx$ formulunde $dx$ icin gercek sayilar verip sonucu (diferansiyeli)
$dy$ olarak hesaplayabiliyoruz. Eger bu formulu tekrar duzenlersek $dy/dx$
for kesir olarak gorulebilirdi.. belki.

Fakat bu teorik olarak tamamiyle, her zaman islemiyor. Yani pratikte bazen
bu sekilde gorebiliyoruz, fakat isin en temelinde durum boyle degil.

Tarihsel olarak, Calculus'u kesfeden matematikci Leibniz bu notasyonu ileri
surdugunde $dy/dx$'i bir kesir olarak dusunmustu, bu degerin temsil ettigi
buyukluk ``$x$'deki sonsuz ufak (infinitesimal) degisimin $y$'de yarattigi
sonsuz ufak degisime orani'' olarak dusunuluyordu.

Fakat Calculus'un sonsuz ufakliklar mantigini reel sayilar cercevesinde
kullanmak teorik olarak pek cok problemi beraberinde getiriyor. Bunlardan
biri, sonsuz ufakligin reel sayilarin oldugu bir cercevede var
olamamasidir! Reel sayilar onemli bir onsarti yerine getirirler, bu sartin
ismi Arsimet Sarti'dir. Bu sarta gore, ne kadar kucuk olursa olsun herhangi
bir pozitif tam sayi $\epsilon > 0$, ne kadar buyuk olursa olsun reel bir
sayi $M>0$ baglaminda, $n\epsilon > M$ sartini dogrulayacak bir dogal sayi
$n$ her zaman mevcuttur. Fakat sonsuz ufak bir $\xi$ o kadar ufak olmalidir
ki onu kendisine ne kadar eklersek ekleyelim, hicbir zaman 1'e erisemeyiz,
ki bu durum Arsimet Sartina aykiri olur [..]

Bu problemlerden kurtulmak icin takip eden 200 sene icinde Calculus ta
temelinden baslayarak sifirdan tekrar yazilmistir, ve simdi gorduklerimiz
bu sifirdan insanin sonuclari (mesela limit kavrami bunun bir sonucu). Bu
tekrar yazim sayesinde / yuzunden turevler artik bir oran degil, bir limit.

\[ \lim_{h \to 0} \frac{f(x+h) - f(x)}{h}\]

Bu ``oranin limiti''ni ``limitlerin orani'' olarak yazamayacagimiz icin
(cunku hem bolum, hem bolen sifira gidiyorlar), o zaman turev bir oran
degildir.

Fakat Leibniz'in notasyonu o tur kullanimi ozendiriyor sanki, oraya dogru
bir cekim yaratiyor, hatta bazen notasyonu o sekilde gormenin ise yaradigi
bile oluyor, yani cogu zaman bu notasyon sanki {\em kesirmis gibi}
davraniyor. Zincirleme Kanunu mesela 

\[ \frac{dy}{dx} = \frac{dy}{du}\frac{du}{dx} \]

Turevleri kesir olarak goruldugu bir durumda ustteki ifade hakikaten dogal
duruyor. Ya da Tersi Fonksiyon (Inverse Function) teorisi

\[ \frac{dx}{dy} = \frac{1}{\frac{dy}{dx}} \]

sonucu da eger turevleri kesir olarak dusunuldugu bir ortamda dogal
gelecektir. Iste bu sebeple, yani notasyonun cok guzel ve ozendirdiginin
cogunlukla dogru seyler olmasi sebebiyle artik bir limiti eden temsil
notasyonu kullanmaya devam ediyoruz, her ne kadar artik gercekten bir
kesiri temsil etmiyor olsa bile. Hatta su ilginc tarihi anektodu ekleyelim,
bu notasyon o kadar iyidir, Newton'un notasyonu tek tirnak, mesela $y'$,
isaretinden o kadar ileridir ki Ingiltere'deki matematik ve bilim kara
Avrupa'sinin yuzyillarca gerisinde kalmistir, cunku Newton ve Leibniz
arasinda Calculus'u kimin kesfettigi konusunda bir catisma yasandi (bugunku
konsensus ikisinin de Calculus'u ayni anda kesfettigi uzerine), ve bu
catisma ortaminda Ingiliz bilim cevresinin Avrupa'daki ilerlemeleri,
Leibniz notasyonunu dislayip Newton'u takip etmeleri sonucunu getirdi.. ve
bu sebeple pek cok alanda ilerleyemeyip, takilip kaldilar [..]. 

Sonuca gelirsek, $dy/dx$'i bir kesir gibi yaziyor, ve pek cok hesapta onu
sanki bir kesirmis gibi kullaniyor, kullanabiliyor olsak bile, $dy/dx$ bir
kesir degildir, sadece filmde (bazen) o rolu oynar. 



\end{document}
