\documentclass[12pt,fleqn]{article}\usepackage{../common}
\begin{document}
Uzakliklar, Norm, Benzerlik

Literaturdeki anlatim norm ve uzaklik konusu etrafinda biraz kafa
karisikligi yaratabiliyor, bu yazida biraz aciklik getirmeye
calisalim. Norm bir buyukluk olcusudur. Vektor uzaylari ile olan alakasini
gormek icin {\em Fonksiyonel Analiz} notlarina bakilabilir. Buyukluk derken
bir $x$ vektorunun buyuklugunden bahsediyoruz, ki bu cogunlukla $||x||$
gibi bir kullanimda gorulur, eger altsimge yok ise, o zaman 2 kabul edilir,
yani $||x||_2$. Bu ifade bir L2 norm'unu ifade eder. $||x||_1$ varsa L1
norm'u olurdu.

L1,L2 normalari, ya da genel olarak $p$ uzerinden $L_p$ normlari soyle gosterilir

$$ ||x||_p = (\sum_i |x_i|^p)^{1/p} $$

ki $x_i$, $x$ vektoru icindeki ogelerdir. Eger $p=2$ ise, L2 norm

$$ ||x||_2 = \bigg(\sum_i |x_i|^2 \bigg)^{1/2} $$

Ustel olarak $1/2$'nin karekok demek oldugunu hatirlayalim, yani 

$$ ||x||_2 = \sqrt{\sum |x_i|^2} $$

Bu norm ayrica Oklitsel (Euclidian) norm olarak ta bilinir, tabii ki bunun
Oklitsel uzaklik ile yakin baglantisi var (iki vektoru birbirinden cikartip
Oklit normunu alirsak Oklit uzakligini hesaplamis oluruz).

Eger $p=1$ olsaydi, yani L1 norm, o zaman ustel olarak $1/1$ olur, yani
hicbir ustel / koksel islem yapilmasina gerek yoktur, iptal olurlar,

$$ ||x||_1 = \sum |x_i|^2 $$


Ornek

$$ 
a = \left[\begin{array}{r}
3 \\ -2 \\ 1
\end{array}\right]
 $$

$$ ||a|| = \sqrt{3^2+(-2)^2+1^2} = 3.742 $$

Ornekte altsimge yok, demek ki L2 norm. 

Ek Notasyon, Islemler

L1 normu icin yapilan islemi dusunelim, vektor ogeleri kendileri ile
carpiliyor ve sonuclar toplaniyor. Bu islem

$||x||_1 = x^Tx$

olarak ta gosterilemez mi? Ya da $x \cdot x$ olarak ki bu noktasal carpimdir.

Bazen de yapay ogrenim literaturunde $||x||^2$ sekilde bir kullanim
gorebiliyorsunuz. Burada neler oluyor? Altsimge yok, demek ki L2
norm. Sonra L2 normun karesi alinmis, fakat L2 normu tanimina gore bir
karekok almiyor muydu? Evet, fakat o zaman kare islemi karekoku iptal eder,
demek ki L2 normunun karesini almak bizi L1 normuna dondurur! Eh bu normu
da $x^Tx$ olarak hesaplayabildigimize gore hemen o notasyona gecebiliriz,
demek ki $||x||^2 = x^Tx = x \cdot x$. 

Ikisel Vektorlerde Benzerlik

Diger ilginc bir kullanim ikisel degerler iceren iki vektor arasinda
cakisan 1 degerlerinin toplamini bulmak. Mesela 

\begin{minted}[fontsize=\footnotesize]{python}
a = np.array([1,0,0,1,0,0,1,1])
b = np.array([0,0,1,1,0,1,1,0])
\end{minted}

Bu iki vektor arasindaki 1 uyusumunu bulmak icin noktasal carpim yeterli,
cunku 1 ve 0, 0 ve 1, 0 ve 0 carpimi sifir verir, ama 1 carpi 1 = 1
sonucunu verir. O zaman L1 norm bize ikisel iki vektor arasinda kabaca bir
benzerlik fikri verebilir.

\begin{minted}[fontsize=\footnotesize]{python}
print np.dot(a,b)
\end{minted}

\begin{verbatim}
2
\end{verbatim}

\end{document}
