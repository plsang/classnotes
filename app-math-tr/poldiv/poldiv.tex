\documentclass[12pt,fleqn]{article}
\setlength{\parindent}{0pt}
\usepackage{graphicx}
\usepackage{cancel}
\usepackage{listings}
\usepackage[latin5]{inputenc}
\usepackage{color}
\setlength{\parskip}{8pt}
\setlength{\parsep}{0pt}
\setlength{\headsep}{0pt}
\setlength{\topskip}{0pt}
\setlength{\topmargin}{0pt}
\setlength{\topsep}{0pt}
\setlength{\partopsep}{0pt}
\setlength{\mathindent}{0cm}
\usepackage{latexsym}
\usepackage{showkeys}
\renewcommand*\showkeyslabelformat[1]{(#1)}
\usepackage{polynom}

\begin{document}
Polinom Bolmek 

Dogal sayilari bolmek icin kullandigimiz bolme yonteminin (long division)
benzerini polinomlar icin de kullanabiliriz. Dogal sayilar icin bir ornek
mesela 146 / 4 diyelim, once bolumde 3 olacagini tahmin ederiz, 3 x 4 = 12,
bolunende 14 var, 14-12=2. Sonra bu 2'yi alip bolumde kalan 6 ile
birlestiririz, 26 yapariz, daha dogrusu 2*10 + 6 deriz. Devam ederiz. 

Polinom bolerken benzer bir durum var, mesela $6x^3-16x^2+17x-6$ polinomunu
$3x-2$ ile bolelim. Bu bolme islemi su sekilde gosterilir, bolunen yukarida
degil asagida. Notasyon biraz degisik ama onemli degil. 

\polylongdiv[stage=1]{6x^3-16x^2+17x-6}{3x-2}

Once $6x^3$'u $3x$ ile boluyoruz. Sonuc $2x^2$. Onu alip cizgi isaretinin
ustune yaziyoruz. 

\polylongdiv[stage=2]{6x^3-16x^2+17x-6}{3x-2}

Sonra ilginc bir hareket, $2x^2$'i alip hem $3x$ ile, hem de $-2$ ile
carpiyoruz, sonucu bolunen polinomun altina yaziyoruz, 

\polylongdiv[stage=3]{6x^3-16x^2+17x-6}{3x-2}

ve cikartma islemi yapiyoruz.

\polylongdiv[stage=4]{6x^3-16x^2+17x-6}{3x-2}

Ve islem bu sekilde devam ediyor. 

\polylongdiv{6x^3-16x^2+17x-6}{3x-2}

Bolme islemi tamamen sifira gitmeyebilir. Mesela $3x^3 - 2x^2 + 4x  - 3$
ile $x^2+3x+3$ bolunurse

\polylongdiv{3x^3 - 2x^2 + 4x  - 3}{x^2+3x+3}

Geriye $28x+30$ kalacaktir. Geri kalan oldugu zaman da bu aslinda ise
yarayan bir sonuctur, artik eski polinomu su sekilde ifade edebiliriz

\[=  (3x-11) + \frac{28x+30}{x^2+3x+3} \]

Bu polinomda bolum 1. derece, bolen 2. derecedir, fakat orijinal polinom
3. dereceden idi. Yani derece sayisinda bir dusus yasandi, yani bir
basitlestirme elde edildi. 








\end{document}
