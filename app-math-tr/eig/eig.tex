\documentclass[12pt,fleqn]{article}\usepackage{../common}
\begin{document}
Ozvektorler, Ozdegerleri Elle Hesaplamak (Eigenvectors, Eigenvalues)

Ozdegerler ve ozvektorler her matrise gore ozel vektorlerdir, ki matris bu
ozel vektorleri transform ettiginde / islediginde sonuc yine ozvektorun
kendisidir, daha dogrusu onun bir sabit (ozdeger) ile carpilmis
halidir. Yani

\[ Ax = \lambda x \]

Tek tarafa gecirelim

\[ Ax - \lambda x = 0 \]

Bu noktada $x$'leri disari cekmek isterdik, fakat bunu yapamayiz, cunku o
zaman iceride $A - \lambda$ kalir ve bu olmaz, cunku $A$ bir matris,
$\lambda$ bir tek sayi. Ama $Ix = x$'ten hareketle

\[ Ax - \lambda I x = 0 \]

diyebiliriz. Simdi disari cekersek

\[ (A - \lambda I) x = 0 \]

Bu ifadenin dogru olmasi icin parantez icindeki ifade / matris tekil
(singular) olmalidir. Bunun icin ise parantez icinin determinanti sifir
olmalidir. Yani

\[ |A - \lambda I| = 0 \]

Ornek 

\[ 
A = 
\left[\begin{array}{rr}
1 & 4 \\ 3 & 5
\end{array}\right]
 \]

\[ 
A - \lambda I = 
\left[\begin{array}{rr}
1 & 4 \\ 3 & 5
\end{array}\right] - 
\lambda
\left[\begin{array}{rr}
1 & 0 \\ 0 & 1
\end{array}\right] 
 \]

\[ 
= 
\left[\begin{array}{rr}
1 - \lambda & 4 \\ 3 & 5-\lambda
\end{array}\right]
 \]

\[ det(A - \lambda I) = (1-\lambda)(5-\lambda) - 4 \cdot 3 \]

Ustteki denkleme karakteristik denklem (characteristic equation) denir. 

\[ = -7 -6\lambda + \lambda^2 \]

Kokleri $\lambda_1 = 7$, $\lambda_2 = -1$.

Her ozdegere tekabul eden ozvektoru bulmak istiyorsak, cikartma islemini
yapalim

\[ 
A - \lambda I = 
\left[\begin{array}{rr}
1-7 & 4 \\ 3 & 5-7
\end{array}\right] = 
\left[\begin{array}{rr}
-6 & 4 \\ 3 & -2
\end{array}\right]
 \]

Su formule donersek

\[ (A - \lambda I) x = 0 \]

Cikartma sonrasi elimize gecen matrisi carpacak oyle bir $x$ vektoru
ariyoruz ki bu vektorle carpinca elimize sifir (vektoru) gecsin. Yani bu
aradigimiz $x$ vektoru $(A - \lambda I)$'nin sifir uzayinda (nullspace). 

2 x 2 boyutundaki boyle ufak bir ornek icin $x$'i aslinda tahmin
edebiliriz. Oyle iki sayi bulalim ki, 1. ve 2. kolonu onlarla ayri ayri
carpip topladigimizda sonuc sifir olsun. Her iki kolonun tepesinde -6 ve 4
goruyorum, sadece bu iki sayinin sifira toplanmasi icin acaba -6'yi 2 ile
4'u 3 ile carpip toplasam olur mu? Kolondaki diger sayilara bakiyoruz, 3 ve
-2 icin de bu ise yariyor. Demek ki ozvektorlerden biri 

\[ x_1 = 
\left[\begin{array}{r}
2 \\ 3
\end{array}\right]
 \]

Diger ise ayni teknigi kullanarak

\[ x_2 = 
\left[\begin{array}{r}
-2 \\ 1
\end{array}\right]
 \]


\end{document}
