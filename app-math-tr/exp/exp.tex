\documentclass[12pt,fleqn]{article}\usepackage{../common}
\begin{document}
``Hafizasiz'' Dagilim, Ustel (Exponential) Dagilim

Ustel dagilimin hafizasiz oldugu soylenir. Bunun ne anlama geldigini
anlatmaya ugrasalim. Diyelim ki rasgele degisken $X$ bir aletin omrunu
temsil ediyor, yani bir $p(x)$ fonksiyonuna bir zaman ``sordugumuz'' zaman
bize dondurulen olasilik, o aletin $x$ zamani kadar daha islemesinin
olasiligi. Eger $p(2) = 0.2$ ise, aletin 2 yil daha yasamasinin olasiligi
0.2. 

Bu hafizasizligi, olasilik matematigi ile nasil temsil ederiz?

\[ P( X>s+t \ | X>t ) =  P(X>s) , \ \forall s, \ t \ge 0 \]

Yani oyle bir dagilim var ki elimizde, $X>t$ bilgisi veriliyor, ama (kalan)
zamani hala $P(X>s)$ olasiligi veriyor. Yani $t$ kadar zaman gectigi 
bilgisi hicbir seyi degistirmiyor. Ne kadar zaman gecmis olursa olsun,
direk $s$ ile gidip ayni olasilik hesabini yapiyoruz. 

Sartsal (conditional) formulunu uygularsak ustteki soyle olur

\[  \frac{P( X>s+t,  X>t )}{P(X>t)} = P(X>s)  \]

ya da

\[  P( X>s+t,  X>t ) = P(X>s)P(X>t) \]

Bu son denklemin tatmin olmasi icin $X$ ne sekilde dagilmis olmalidir?
Ustteki denklem sadece $X$ dagilim fonksiyonu ustel (exponential) olursa
mumkundur, cunku sadece o zaman

\[ e^{-\lambda(s+t)}  = e^{-\lambda s} e^{-\lambda t}\]

gibi bir iliski kurulabilir. 

Ornek

Diyelim ki bir bankadaki bekleme zamani ortalam 10 dakika ve ustel olarak
dagilmis. Bir musterinin i) bu bankada 15 dakika beklemesinin ihtimali
nedir? ii) Bu musterinin 10 dakika bekledikten sonra toplam olarak 15
dakika (ya da daha fazla) beklemesinin olasiligi nedir? 

Cevap

i) Eger $X$ musterinin bankada bekledigi zamani temsil ediyorsa

\[ P(X>15) = e^{-15 \cdot 1/10} = e^{-3/2} \approx 0.223 \]

ii) Sorunun bu kismi musteri 10 dakika gecirdikten sonra 5 dakika daha
gecirmesinin olasiligini soruyor. Fakat ustel dagilim ``hafizasiz'' oldugu
icin kalan zamani alip yine direk ayni fonksiyona geciyoruz, 

\[ P(X>5> = e^{-5 \cdot 1/10} = e^{-1/2} \approx 0.60\]

Kaynak

Introduction to Probability Models, Sheldon Ross, 8th Edition, sf. 273

\end{document}
