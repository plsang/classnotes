% Created 22.06.2004 Burak Bayramli
%Uses ecctd01.cls style file
\documentclass{ecctd01} % 
\usepackage{amsmath}
\usepackage[latin5]{inputenc}
\title{
Regression, Least Squares
}

\author{Burak Bayramli}

\begin{document}
\maketitle

\begin{abstract}
The purpose of this paper is to communicate the author's knowledge on
machine learning field, a subject on which he desires to complete a
doctorate degree. In this text, we will go over the purposes,
motivation, some history and methods of machine learning while trying
to make the case for the statistical perspective and graphical model
approach. 
\end{abstract}


\section{Introduction}
Machine Learning lets us derive short representations for raw data
when we do not have a clear mathematical model for the domain we are
interested in. Once we  acquire a model using ML methods, we can also
use ML applications like classification, regression, clustering,
anomaly detection, feature selection, all tools which fall under
the umbrella of ML. 

Throughout this paper, vector $\mathbf{x}$ will be used to define
input data for one dimension, and $\mathbf{X}$ for many 
dimensions. All elements of vector $\mathbf{x}$ should be assumed
as data points, measurements for a single attribute, and should be
assumed i.i.d unless stated otherwise.

\section{Regression Based Methods}
\subsection{The Model}
A suitable starting point for a machine learning introduction is linear
regression. Given data $\mathbf{x}$, we decide to model target
function f(x) as a line whose slope and starting point are yet unknown. 

\begin{equation}
  \label{eq:reg1}
  f(x;\theta) = \sum_{i=1}^N \theta_{1}x_{i} + \theta_{0}
\end{equation}
where $\theta_{0}$ and $\theta_{1}$ represent unknown variables. We
have to search through many possible $f(x;\theta)$'s to find an 
optimal $\theta$. 

\begin{figure}[htpb]
  \vspace{3.0cm}  
  \special{psfile=linreg.eps vscale=60 hscale=60}
  \caption{\label{linreg} Linear Regression.}
\end{figure}

First we convert (\ref{eq:reg1}) into matrix form so that linear
algebra notation is used. Through linear algebra, we will later be
able to expand the equation into many dimensions easily.
\begin{equation}
  \label{eq:reg2}
  f(x;\theta) = \sum_{i=1}^N 
  \left[ \begin{array}{cc}
      1 & x_{i}
  \end{array} \right]
  \left[ \begin{array}{c}
      \theta_{0} \\
      \theta_{1}
  \end{array} \right]
\end{equation}

 \begin{equation}
  \label{eq:reg3}
  f(\mathbf{x};\theta) = 
  \left[ \begin{array}{cc}
      1 & x_{1} \\
      \vdots \\
      1 & x_{N} \\      
  \end{array} \right]
  \left[ \begin{array}{c}
      \theta_{0} \\
      \theta_{1}
  \end{array} \right]
\end{equation}
Notice the sum in (\ref{eq:reg2}) disappeared in (\ref{eq:reg3}) and
is replaced by added dimensions in the first matrix. Further introduction
of linear algebra notation simplifies the equation even further: 

\begin{equation}
  \label{eq:reg4}
  f(\mathbf{X};\theta) = \mathbf{X}\theta
\end{equation}

Now for our search, we need to compare the calculated $f(x;\theta)$'s
against a given $\mathbf{y}$. This comparison is done utilizing a loss
function which defines our success/fail criteria. Let's say we choose
sum of square distances between calculated $f(x;\theta)$'s and given
$\mathbf{y}$ as our loss function. Based on this loss function, we can
calculate overall risk as sum of square distances.

\begin{equation}
  \label{eq:reg5}
  R(\theta) = \frac{1}{2N}|\mathbf{y} - \mathbf{X}\theta|^2
\end{equation}

Note: We are multiplying with $\frac{1}{2}$ so that it cancels out
with $^2$ after derivation. It turns out multiplying with a scalar
does not change the location of the minima.

In ML literature, the function $R(\theta)$ is called empirical
risk. We will aim to minimize R, which at the same time means we will  
be searching for the best $\theta$. 

\begin{figure}[htpb]
  \vspace{3.0cm}  
  \special{psfile=risk.eps vscale=70 hscale=70}
  \caption{\label{risk} Risk Function in 2D.}
\end{figure}

The point where risk is lowest is where the rate of change at any
point is zero, and the gradient of $R(\theta)$ equals to
zero. Gradient contains all partial derivatives of $R(\theta)$ in
vector form. Taking the gradient of a function produces a vector
which always points to the direction of largest increase. At 
the point where the increase is zero, it means we reached a flat
region where we have no increase nor decrease.

In cases when taking the gradient results in an algebraically messy
formula, approximation  methods are used to iteratively {\em move} in the
direction of fastest increase, trying to reach the region where
gradient equals zero. In the current case, the algebra will come out clean,
as we shall see below.

\begin{eqnarray*}
  \label{eq:reg6}
  \bigtriangledown_{\theta} R\left(\theta\right) = \left[ \begin{array}{c}
      0 \\
      0 
    \end{array} \right] &  \nonumber \\
  \bigtriangledown_{\theta}\left( \frac{1}{2N}||\mathbf{y} -
  \mathbf{X}\theta||^2\right) = 0 \nonumber \\
  \frac{1}{2N}  \bigtriangledown_{\theta} \left(
  \left(\mathbf{y} -  \mathbf{X}\theta \right) ^T
  \left( \mathbf{y} -  \mathbf{X}\theta \right)
  \right) = 0 \nonumber \\
  \frac{1}{2N}  \bigtriangledown_{\theta} \left(
  \left(\mathbf{y}^T -  \left(\mathbf{X}\theta\right)^T \right)
  \left( \mathbf{y} -  \mathbf{X}\theta \right)
  \right) = 0 \nonumber \\
  \frac{1}{2N}  \bigtriangledown_{\theta} \left(
  \mathbf{y}^T\mathbf{y} -
  \mathbf{y}^T\mathbf{X}\theta -
  \left( \mathbf{X}\theta \right)^T\mathbf{y} +
  \theta^T\mathbf{X}^T\mathbf{X}\theta
  \right) = 0 \nonumber 
\end{eqnarray*}

The second and third term are really the same because for vector
multiplication, $\mathbf{u}^T \mathbf{v} = \mathbf{v}^T \mathbf{u}$.  

\begin{eqnarray}
  \label{eq:reg7}
  \frac{1}{2N}  \bigtriangledown_{\theta} \left(
  \mathbf{y}^T\mathbf{y} -2
  \mathbf{y}^T\mathbf{X}\theta +
  \theta^T\mathbf{X}^T\mathbf{X}\theta
  \right) = 0 \nonumber \\
  \frac{1}{2N} \left(
  -2\mathbf{y}^T\mathbf{X}
  +2\theta\mathbf{X}^T\mathbf{X} \right) = 0 \nonumber \\
  \frac{1}{N} \left(
  -\mathbf{y}^T\mathbf{X}
  +\theta\mathbf{X}^T\mathbf{X} \right) = 0 \nonumber \\
  \theta\mathbf{X}^T\mathbf{X} =
  \mathbf{y}^T\mathbf{X} \nonumber \\
  \theta = \left(\mathbf{X}^T\mathbf{X}\right)^{-1}
  \mathbf{y}^T\mathbf{X}
\end{eqnarray}

Once $\theta$ is calculated, it can be plugged in $f(x;\theta)$
to predict future data.

\subsection{Multiple Dimensions }
Linear regression can be easily extended to multiple dimensions. 

\begin{displaymath}
  \mathbf{X} = 
  \left[ \begin{array}{cccc}
      1 & x_{1} & \hdots & x_{1}(D) \\
      \vdots & \vdots & \vdots & \vdots \\
      1 & x_{N} & \hdots & x_{N}(D) \\
  \end{array} \right] 
\end{displaymath}

\begin{displaymath}
  \label{eq:reg8}  
  R\left(\theta\right) = \frac{1}{2N}
  \left|
  \left[ \begin{array}{c}
      y_{1} \\
      \vdots \\
      y_{N}
  \end{array} \right]
  -
  \mathbf{X}
  \left[ \begin{array}{c}
      \theta_{0} \\
      \theta_{1} \\
      \vdots \\
      \theta_{D} 
  \end{array} \right]
  \right|^2 
\end{displaymath}
This risk calculation is for N data points and D
dimensions. Calculation for $\theta$ can still be performed through
Equation \eqref{eq:reg7}. The result will be a N-dimensional
hyperplane fit to our training data.  

\subsection{Basis Functions}
We also may choose to model the training data through so-called basis
functions. With this method, a polynomial, sinusiodal, or radial basis
function is selected, then every data point $\mathbf{x}_{i\in N}$ is
fed through the basis function calculating a new $\mathbf{x}_{i}$. Then
regression becomes an act of finding the optimal combination of N
basis functions. Radial basis functions are particularly popular, as
they allow learning in many dimensions.  

Calculating a radial basis function means inserting a ``bump'' for
every data point. The bump is shown as

\begin{displaymath}
  \phi_{i}(\mathbf{x}) = exp\left(
  -\frac{1}{2\sigma}\left|\mathbf{x} - \mathbf{x}_{i}\right|^2
  \right)
\end{displaymath}

where $\mathbf{x}$ can be in one or many dimensions. $\mathbf{x}$ here
represents one data point in many dimensions unlike previously, when
it represented all data points in one vector.

The width of the bump is controlled through $\sigma$ which is held
constant throughout the calculations. Risk then becomes  
\begin{equation}
  \label{eq:reg10}
  R(\theta) = \frac{1}{2N}|\mathbf{y} - Q\theta|^2
\end{equation}

where Q is 
\begin{displaymath}
  \label{eq:reg9}
  \left[ \begin{array}{cccc}
      1 & \phi_{0}(x_{1}) & \hdots & \phi_{D}(x_{1}) \\
      \vdots & \vdots & \vdots & \vdots \\
      1 & \phi_{0}(x_{N}) & \hdots & \phi_{D}(x_{N}) \\
  \end{array} \right]
\end{displaymath}

We can still use \eqref{eq:reg7} to perform the regression. 

\subsection{Problems With Regression}
A critical issue with regression is the absence of a firm mathematical
foundation for choosing a modeling method for each
problem. As we have seen, the loss function was pulled out of a
hat. The loss function could easily have been based on the cube of 
distances instead of squares. Certain approaches tried to address this
issue by dividing the training data in two parts, one part used for
training, the other to measure the performance of the training, hence
evaluating each model. Ideally however, we would like to use one
modeling technique for diverse set of problems.

Using regression for classification also poses certain problems. In
figure~\ref{clasreg1}, we see that ideal classification would be the
dashed line. 

\begin{figure}[htpb]
\vspace{4.0cm}
\special{psfile=clasreg1.eps voffset=-320 hoffset=5 vscale=90 hscale=100}
\caption{\label{clasreg1} Classification as Regression}
\end{figure}

However, the points that are furthest to the right and left are being
penalized the most by the algorithm, so the regression naturally wants
to pull the line towards the furthest points. The result is the solid
line shown. An engineering solution for this is using a squashing
function triying to smooth out the regression line. This reverses the
penalization effect for the larger points. A popular squashing
function is the sigmoid function $g(z) = (1 + exp(-z))^{-1}$ and sign
function used for building perceptrons. 

\subsection{Neural Networks}

With the introduction of squashing functions regression becomes more
usable. On top of this, introducing several layers of regression each
utilizing different squashing function lets us handle more complex
classification problems as seen in figure~\ref{nn1}. This approach is
also called a Neural Network.

\begin{figure}[htpb]
\vspace{3.0cm}
\special{psfile=nn1.eps voffset=-320 hoffset=5 vscale=42 hscale=45}
\caption{\label{nn1} Different Layers for a Neural Network.}
\end{figure}

With the introduction of squashing function in the network, finding
the minima using the gradient becomes algebraically hard to
derive. Therefore, we use an approximation technique such as gradient
descent. For a multi-layered neural network, gradient descent is
applied layer by layer, and idea which forms the basis of the BACKPROP
algorithm. BACKPROP has been used successfully for many problems
over the years.

Despite the added power of neural networks, certain limitation still
remain. For one, NNs are difficult to analyze. As we add a new layer
to the network, the consequences of this action on the network cannot be
determined. There are also problems that are still impossible to solve
using NNs, such as learning the equation $x^2+y^2=1$, which is not a
function but a {\em relation}. There is also the infamous Projectile
Cannon problem which is a case of distal learning, and is also
impossible to learn using a regression based approach. Problems such
as these are much easily addressed using a statistical approach, such
as Bayesian Networks. 


\section{Statistical Approach}

\subsection{Introduction}

Statistical approach allows us to capture more about the training
data. In two dimensions, when we do regression, we learn only a slope. With 
statistical approach when we fit a Gaussian distribution, we learn
the center of data distribution $\mu$ (mean), a width $\sigma$
(variance), and a rotation of the ellipse $\Sigma$ (covariance).  

Statistical approach also allows us to use all the tools of probabily theory and
statistics. This means the foundation of the model is rigourous, and the
applications will have to revert less to randomly applied solutions. This will
later help in analysing and improving our model as well. 

\subsection{The Method}

With probability models, instead of learning an exact value for a
variable, we may choose to learn a distribution on the same variable. A $y$
would be replaced by $p(y)$. In comparison, using $y$ instead of $p(y)$ is akin
to using a distribution with all its weight on one single point.

Introducing more unknown variables into distribution results in a joint
distribution. In discrete case, the joint distribution table size is a finite
quantity.

Dependence and independence between random variables play a factor in
joint distribution table size. If two events, therefore their random variables
are related, it is more accurate for our model to reflect this dependence. If
for simplication's sake however, all variables were made dependent, the result
will be an exponential blow up of the joint distribution table. As an example, a
distribution over D unknown binary random variables all dependent on
eachother results a table of size $2^D$. In comparison, if we make all
variables as independent, the size drops to $2D$. 

Perhaps we can start with $2D$ and slowly introduce
dependencies. Dependency between variables are represented with
$p(y|x)$ which reads ``the probability of $y$ given x''. The joint
distribution is $p(x,y) = p(y|x)p(x)$ and is a well known equality in
probability theory. Since writing down all dependencies algebraicly
might be confusing, graph notation is also used. The joint
distribution $p(x,y,z)$ seen in figure~\ref{graph2} will be equal to
product of all probabilities, $p(x)p(y|x)p(z|x)$.  

\begin{figure}[htpb]
\vspace{3.0cm}
\special{psfile=graph2.eps voffset=-320 hoffset=5 vscale=60 hscale=60}
\caption{\label{graph2} Graphical Model.}
\end{figure}

Notice $p(x,y,z)$ does not equal $p(x)p(y)p(z)$ according to this model. That
would only be the case if all variables were independent. 

When dealing with probability distributions, we can also represent model
variables as random variables with other variables depending on
them. As an example, let's take the problem of classifying points in
figure~\ref{gaussdata}. 

\begin{figure}[htpb]
\vspace{3.0cm}
\special{psfile=gaussdata.eps voffset=-320 hoffset=5 vscale=80 hscale=80}
\caption{\label{gaussdata} Sample Data.}
\end{figure}

In the example, $x \in \mathbf{R}^2$ is input and $y \in \{0,1\}$ is
the output. Since y is binary, we can model $p(y)$ as a Bernoulli
distribution, 

\begin{equation}
  p(y_{i}|\alpha)=\alpha^y_{i}(1-\alpha)^{1-y_{i}}
\end{equation}
where $i=1...N$. Using a distribution instead of a table
allows us to manipulate the distribution algebraically, plug it in to
methods such as Maximum Likelihood, take its derivative and so on. 

For x's, we will pick a Gaussian distribution
\begin{equation}
  \label{eq:p_x_y}
  p(x_{i}|y_{i},\mu,\Sigma)=N(x_{i}|\mu_{y},\Sigma_{y})  
\end{equation}

The graphical model for the data will look like in
figure~\ref{gauss_repl_plate}. 

\begin{figure}[htpb]
\vspace{3.0cm}
\special{psfile=gauss_repl_plate.eps voffset=-320 hoffset=5 vscale=55 hscale=55}
\caption{\label{gauss_repl_plate} Model.}
\end{figure}

We see that $\mu,\alpha,\Sigma$ are also made random variables. Looking at the
graph, the algebraic joint distribution $p(x,y,\mu,\alpha,\Sigma)$ can be shown
as 

\begin{equation}
  \label{eq:graphjoint1}
  p(y|\alpha)p(x|y,\mu,\Sigma)p(\Sigma)p(\mu)p(\alpha)
\end{equation}
For our classification needs, we need to turn Equation
\eqref{eq:graphjoint1} into a version that answers
$p(x,y|\alpha,\mu,\Sigma)$. From pure probability theory, we proceed 
algebraically

\begin{equation}
  \label{eq:graphjoint2}
  p(x,y|\alpha,\mu,\Sigma) =
  \frac{p(x,y,\alpha,\mu,\Sigma)}
       {p(\alpha)p(\mu)p(\Sigma)} 
\end{equation}

Now we replace the nominator in \eqref{eq:graphjoint2} with the joint
distribution taken from the graphical model in \eqref{eq:graphjoint1}.

\begin{eqnarray}
  \label{eq:graphjoint3}
  p(x,y|\alpha,\mu,\Sigma) &=&
  \frac{p(y|x)p(x|y,\mu,\Sigma)p(\alpha)p(\mu)p(\Sigma)}
       {p(\alpha)p(\mu)p(\Sigma)} \nonumber \\
  &=& p(y|x)p(x|y,\mu,\Sigma) 
\end{eqnarray}


\subsection{Maximum Likelihood}

In order to find the parameter which are most likely to fit traning data, we
will use a method called Maximum Likelihood. This requires taking the product of
\eqref{eq:graphjoint3} for N data points. The best estimation for the unknown
parameters are at the point when this final super joint probability is at its
heightest value. 

\begin{equation}
  \label{eq:likelihood1}
Likelihood = \prod_{i=1}^N p(y|x)p(x|y,\mu,\Sigma)
\end{equation}


The point when likelihood is heightest is when the derivative of likelihood
equals to zero.


In order to simply the derivative later, we take the log of terms
inside \eqref{eq:likelihood1}. This will turn the product into a
sum. We can do this because the log of the formula in
\eqref{eq:likelihood1} has the same rate of change, therefore the
final result will not be effected.

\begin{equation}
  \label{eq:likelihood2}
  l = \sum_{i=1}^N log (p(y_{i}|x_{i})) + \sum_{i=1}^N log (p(x_{i}|y_{i},\mu,\Sigma))
\end{equation}

We can break apart the last term into two sums. We need to marginalize over $y$
in two chunks, one for each possible value of $y$.  

\begin{eqnarray}
  \label{eq:likelihood3}
  l = \sum_{i=1}^N \log(p(y_{i}|\alpha)) + 
  \sum_{y_{i}\in 0} \log(p(x_{i}|\mu_{0},\Sigma_{0})) + \nonumber \\
  \sum_{y_{i}\in 1} \log(p(x_{i}|\mu_{1},\Sigma_{1}))
\end{eqnarray}

Now, if we take the gradient of the likelihood according to $\alpha, \Sigma,
\mu$ (seperately) and set these equations equal to zero, we get equations for
most likely parameter values. Let's do $\frac{\partial l}{\partial
  \alpha}$ as an example. Because we are taking derivative with respect to
$\alpha$, all the terms of \eqref{eq:likelihood3} go away except for
the first term. Then we have

\begin{eqnarray}
  \label{eq:likelihood4}
  \frac{\partial l}{\partial \alpha} &=&
  \frac{\partial }{\partial \alpha}
  \sum_{i=1}^N log (p(y_{i}|\alpha))  \nonumber \\
  0 &=&
  \frac{\partial}{\partial \alpha}
  \sum_{i=1}^N log\left(\alpha^y_{i}(1-\alpha)^{1-y_{i}}\right) \nonumber \\
  0 &=&
  \frac{\partial}{\partial \alpha}
  \sum_{i=1}^N y_{i}log(\alpha) + (1-y_{i})log(1-\alpha) \nonumber
\end{eqnarray}

From the problem domain, we know that some $y_{i}$'s are 0, some are
1. Zero values would cancel out $y_{i}log(\alpha)$, one values would
cancel out the $(1-y_{i})log(1-\alpha)$. So, in order to drop
$y_{i}$'s from the equation, we change the equation to  

\begin{eqnarray}
  \label{eq:likelihood5}
  0 &=&
  \frac{\partial}{\partial \alpha}\
  \left(
  \sum_{i\in class1} \log(\alpha) +
  \sum_{i\in class0} \log(1-\alpha)
  \right)
  \nonumber \\
  0 &=&
  \sum_{i\in class1}\frac{1}{\alpha} -
  \sum_{i\in class0}\frac{1}{1-\alpha} \nonumber
\end{eqnarray}

Now, let $N_{0}$ be the quantity of $y_{i}$'s in class 0, $N_{1}$ the
quantity of $y_{i}$'s in class 1. 

\begin{eqnarray}
  \label{eq:likelihood6}
  0 = \frac{N_{1}}{\alpha}-\frac{N_{0}}{1-\alpha} \nonumber \\
  \alpha = \frac{N_{1}}{N_{1}+N_{0}} = \frac{N_{1}}{N}\nonumber
\end{eqnarray}

This result supports the intuition that dividing the number of $y_{i}$'s
such that $y_{i}=1$ in the training set with the total number of
traning points results in the value for $\alpha$. We could guessed this,
but it's nice to see verification for it through Maximum Likelihood
mathematically.  

The rest of the unknowns, namely $\mu_{0}, \mu_{1}, \Sigma_{0},
\Sigma_{1}$ can be derived using the same idea. Taking the derivative
of \eqref{eq:likelihood3} with respect the unknowns one by one, and 
equating the results to zero will give us formulas for each
unknown. The derivation will not be shown here due to lack of
space. The final result will give all the unknown parameters of
Equation \eqref{eq:p_x_y}. Using \eqref{eq:p_x_y} in conjunction with
$p(y)$ we derived earlier, we can finally use \eqref{eq:graphjoint3}
for classification. We should note here that in ML literature, $p(y)$
is called {\em prior guess}.

\subsection{Classification}

For classification we need one final equation, namely $p(y|x)$ which
is known as posterior in ML terminology. With the posterior known, when
we are given a new $x$ we have not seen before, we can plug it into
such as $p(y|x=new value)$ and calculate the likelihood of the new $y$. 

Let's derive the posterior $p(y|x)$. From probability theory,

\begin{equation}
  p(y|x) = \frac{p(x,y)}{p(x)} 
\end{equation}

We need $p(x)$ for this to work. Let's take $p(x,y)$ and marginalize
over $y$. 

\begin{eqnarray}
  p(y|x) &=& \frac{p(x,y)}{\sum_{y}p(x,y)}  \\
  &=& \frac{p(x,y)}{p(x,y=0) + p(x,y=1)}
\end{eqnarray}

Now the act of classification becomes calculating $p(y=1|x)$, and
$p(y=0|x)$ seperately for two possible values of $y$. Whichever
probability value has the higher value, will be our classification
answer for the new data point $x$. 

\section{Acknowledgements}

All figures in this paper with the exception of figure~\ref{clasreg1}
are taken from Prof. Tony Jebara's Machine Learning lecture notes at
{\tt http://www.cs.columbia.edu/~jebara/4771}. A list of Machine
Learning literature I own is presented below. 

\begin{thebibliography}{9}
\itemsep=0ex % save space!
  
\bibitem{Bishop} C. Bishop {\em Neural Networks for Pattern
  Recognition}, Oxford University Press, 1995.
\bibitem{Duda} R. O. Duda, P. E. Hart \& D. G. Stork {\em Pattern
  Classification (2nd ed)}, Wiley, 2000.
\bibitem{BishopJordan} M. Jordan, C. Bishop, {\em Introduction to
  Graphical Models}, not yet published.
\bibitem{Mitchell} T. Mitchell, {\em Machine Learning}, McGraw-Hill, 1997.

\end{thebibliography}

\end{document}
