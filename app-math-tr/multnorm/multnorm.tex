\documentclass[12pt,fleqn]{article}\usepackage{../common}
\begin{document}
Cok Degiskenli Normal Numaralari  (Multivariate Normal Tricks)

Cok degiskenli normal dagilimlarla is yaparken, mesela Gaussian karisimlari
kullanirken, bazi numaralari bilmek faydali olabiliyor. Bunlardan birincisi
$(x-\mu)^T\Sigma^{-1}(x-\mu)$ hesabini yapmaktir, diger log-toplam-exp
numarasi (logsumexp trick) diye bilinen hesaptir.

Birinciden baslayalim, daha kisalastirmak icin $y=x-\mu$ diyelim, yani
$y^T\Sigma^{-1}y$ olsun. Simdi bu formulde bir ters alma (inversion)
isleminin oldugunu goruyoruz. Fakat bu islem oldukca pahali bir islem
olarak bilinir, hele hele boyutlarin yukseldigi durumlardan (binler,
onbinler), kovaryansi temsil eden $\Sigma$, $n \times n$ olacaktir. Acaba
tersini almayi baska bir sekilde gerceklestiremez miyiz?

$\Sigma$ matrisi bir kovaryans matrisi oldugu icin simetrik, pozitif yari
kesin bir matristir. Bu tur matrislerin Cholesky ayristirmasinin oldugunu
biliyoruz ve bu islem cok hizli yapilabiliyor. O zaman 

$$ \Sigma = LL^T $$

ki $L$ matrisi alt-ucgensel (lower triangular) bir matristir,

$$ \Sigma^{-1} = (LL^T)^{-1} $$

$$ = L^{-T}L^{-1} $$

Bunu temel alarak iki taraftan $y$'leri geri koyalim,

$$ y^T\Sigma^{-1}y= y^TL^{-T}L^{-1}y $$

Bilindigi gibi lineer cebirde istedigimiz yere parantez koyabiliriz,

$$ = (y^TL^{-T})L^{-1}y $$

Parantezden bir seyin devrigi gibi temsil edersek, parantez icindekilerin
sirasi degisir ve tek tek devrigi alinir,

$$ = (L^{-1}y)^TL^{-1}y $$

$$  = |L^{-1}y|^2 $$

Ustteki ifadede $|\cdot|$ icindeki kisim $Ax=b$ durumundaki $x$'in en az
kareler cozumu olan $A^{-1}b$'ye benzemiyor mu? Evet. Bu durumda her
standart sayisal kutuphanede mevcut bir cagri ile $L^{-1}y$ hesabini
yapabiliriz, bu cagrilar perde arkasinda ters alma isleminden kacinarak bir
suru optimizasyon yaparak sonuca erismektedirler. Ustune ustluk $L$
durumunda bu cok daha hizli olacaktir, cunku $L$ alt-ucgensel oldugu icin
cozum geriye deger gecirmek (backsubstitution) ile aninda bulunabilir. 

Demek ki $y^T\Sigma^{-1}y$ hesabi icin once $\Sigma$ uzerinde Cholesky
aliyoruz, sonra $L^{-1}y$ cozduruyoruz. Elde edilen degerin noktasal
carpimini alinca $\Sigma$'nin tersini elde etmis olacagiz. Ornek, once uzun yoldan,

\begin{minted}[fontsize=\footnotesize]{python}
import numpy.linalg as lin
Sigma = np.array([[10., 2.],[2., 5.]])
y = np.array([[1.],[2.]])
print np.dot(np.dot(y.T,lin.inv(Sigma)),y)
\end{minted}

\begin{verbatim}
[[ 0.80434783]]
\end{verbatim}

Simdi Cholesky ve en az kareler uzerinden

\begin{minted}[fontsize=\footnotesize]{python}
L = lin.cholesky(Sigma)
x = lin.lstsq(L,y)[0]
print np.dot(x.T,x)
\end{minted}

\begin{verbatim}
[[ 0.80434783]]
\end{verbatim}

Ayni sonuca eristik.

log-toplam-exp

Bu numaranin ilk kismi nisbeten basit. Bazi yapay ogrenim algoritmalari icin
olasilik degerlerinin birbiriyle carpilmasi gerekiyor, mesela 

$$ r = p_1 \cdot p_2 \dots p_n $$

Olasiliklar 1'den kucuk oldugu icin 1'den kucuk degerlerin carpimi asiri
kuculebilir, ve k���kl�gun tasmasi (underflow) ortaya cikabilir. Eger
carpim yerine $\log$ alirsak, carpimlar toplama donusur, sonra sonucu
$\exp$ ile tersine ceviririz, ve $\log$'u alinan degerler cok kuculmez,
carpma yernie toplama islemi kullanildigi icin de nihai deger de kucukluge
dogru tasmaz.

$$ \log r = \log p_1 + \log p_2 + \dots + \log p_n $$

$$ r = \exp(\log p_1 + \log p_2 + \dots + \log p_n )$$

Bir diger durum icinde $exp$ ifadesi tasiyan bir olasilik degerinin cok
kucuk degerler tasiyabilmesidir. Mesela cok degiskenli Gaussian karisimlari
icin alttaki gibi bir hesap surekli yapilir, 

$$ = \sum_i w_i
\frac{ 1}{(2\pi)^{k/2} \det(\Sigma)^{1/2}} \exp 
\bigg\{ 
-\frac{ 1}{2}(x-\mu)^T\Sigma^{-1}(x-\mu)
\bigg\}
 $$

ki $0 \le w_i \le 1$ seklinde bir agirlik degeridir. Ustteki formulun
cogunlukla $\log$'u alinir, ve, mesela bir ornek uzerinde gorursek (ve
agirliklari bir kenara birakirsak), 

$$ \log(e^{-1000} + e^{-1001}) $$ 

gibi hesaplar olabilir. Ustteki degerler tamamen uyduruk denemez,
uygulamalarda pek cok kez karsimiza cikan degerler bunlar. Her neyse, eger
ustteki ifadeyi kodla hesaplarsak, 

\begin{minted}[fontsize=\footnotesize]{python}
print np.log(np.exp(-1000) + np.exp(-1001))
\end{minted}

\begin{verbatim}
-inf
\end{verbatim}

Bu durumdan kurtulmak icin bir numara sudur; $\exp$ ifadeleri arasinda en
buyuk olanini disari cekersiniz, ve $\log$'lar carpimi toplam yapar, 

$$ \log(e^{-1000}(e^{0} + e^{-1} ))$$

$$ -1000 + \log(1 + e^{-1})$$

Bunu hesaplarsak, 

\begin{minted}[fontsize=\footnotesize]{python}
print -1000 + np.log(1+np.exp(-1))
\end{minted}

\begin{verbatim}
-999.686738312
\end{verbatim}

Bu numaranin yaptigi nedir? Maksimumu disari cekerek en az bir degerin
kucuklugu tasmamasini garantilemis oluyoruz. Ayrica, bu sekilde, geri kalan
terimlerde de asiri ufalanlar terimler kalma sansi azaliyor. 

{\em Numerical Recipes, 3rd Edition}

\url{http://makarandtapaswi.wordpress.com/2012/07/18/log-sum-exp-trick/}

\end{document}
