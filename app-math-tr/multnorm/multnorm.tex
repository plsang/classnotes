\documentclass[12pt,fleqn]{article}\usepackage{../common}
\begin{document}
Cok Degiskenli Normal Numaralari  (Multivariate Normal Tricks)

Cok degiskenli normal dagilimlarla is yaparken, mesela Gaussian karisimlari
kullanirken, bazi numaralari bilmek faydali olabiliyor. Bunlardan birincisi
$(x-\mu)^T\Sigma^{-1}(x-\mu)$ hesabini yapmaktir, diger logtoplamexp
numarasi (logsumexp trick) diye bilinen hesaptir.

Birinciden baslayalim, daha kisalastirmak icin $y=x-\mu$ diyelim, yani
$y^T\Sigma^{-1}y$ olsun. Simdi bu formulde bir ters alma (inversion)
isleminin oldugunu goruyoruz. Fakat bu islem oldukca pahali bir islem
olarak bilinir, hele hele boyutlarin yukseldigi durumlardan (binler,
onbinler), kovaryansi temsil eden $\Sigma$, $n \times n$ olacaktir. Acaba
tersini almayi baska bir sekilde gerceklestiremez miyiz?




\end{document}
