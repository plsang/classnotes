\documentclass[12pt,fleqn]{article}\usepackage{../common}
\begin{document}
Taylor Serisi

Formul 

$$ f(x) = \sum _{n=0}^{\infty} \frac{f^{n}(0)}{n!} x^n  $$

Bu formule nasil ulasirim? Su sekildeki bir seri olsun

$$ f(x) = a_0 + a_1x + a_2x^2 +a_3x^3 + ...   $$

Usttekinin turevini alalim.  Tabii ki sabit $a_0$ yokolacak, $x$'in
onundeki katsayi kalacak, vs. Sonuc

$$ f' = a_1 + 2a_2x + 3a_3 x^2 + .. $$

Birkac kez daha

$$ f'' =  2a_2 + 3 \cdot 2 a_3 x + .. $$

$$ f''' = 3 \cdot 2 a_3 + 4 \cdot 3 \cdot 2 a_4 x+ ... $$

Eger son formule sifir verirsem, 3. terimi cimbizla cekip alabilirim

$$ f'''(0) = 3 \cdot 2 a_3 $$

Cunku geri kalan her sey sifir olup yokoldu, geriye sabitler kaldi. O zaman
$a_3$'u elde etmek istiyorsam,

$$ \frac{f'''(0)}{3 \cdot 2 \cdot 1}  = a_3$$

Bir kalip ortaya cikmistir herhalde, genel olarak 

$$ a_n = \frac{f^n(0)}{n!} $$

$$ n! = n(n-1)(n-2)...1 $$

Bu katsayilar Taylor formulunde $x_n$ onune gelecek katsayilardir. 

O zaman 

$$ f(x) = f(0) + f'(0)x + f'' \frac{x^2}{2!} + ...$$

Daha genel olarak 0 yerine $a$ alirsak, $a$ yakinindaki fonksiyonun
acilimini temsil edebiliriz

$$ f(x) = f(a) + f'(a)(x-a) + f'' \frac{(x-a)^2}{2!} + ...$$

Alternatif Turetim

Taylor serilerinin arkasindaki fikir, surekli ve sonsuz defa turevi
alinabilen turden bir fonksiyon $f(x)$'i bir $x_0$ noktasinin (burada $a$
sembolu de kullanilabilir) ``cevresinde'', yakin bolgesinde yaklasiksal
olarak temsil edebilmektir.

Turetmek icin

Calculus'un Temel Teorisi der ki:

$$ \int_a^x f' \left({t}\right) \ \mathrm d t = 
f \left({x}\right) - f \left({a} \right)
$$

Bu formulu tekrar duzenlersek, alttakini elde ederiz:

$$ f \left({x}\right) = f \left({a}\right) + \int_a^x f'(t) \ \mathrm d t $$

Bunun uzerinde Parcali Entegral yontemini uygulariz. Parcali Entegral
teknigi genel olarak soyledir:

$$ \int_a^b u \ dv = u \ v - \int_a^b v \ du $$

Simdi iki ustteki formulun entegral icindeki kismini parcali entegrale
uyacak sekilde bolusturelim

$u = f' \left({t}\right)$ ve $dv = dt$

O zaman acilim

$$ f \left({a}\right) + x f' \left({x}\right) - a f' \left({a}\right) - \int_a^x t f'' \left({t}\right) \ \mathrm d t $$

Alttaki formulu kullanarak

$$ \int_a^x x f'' \left({t}\right) \ \mathrm d t = x f' (x)-x f' (a) $$

iki ustteki formulu su hale getiririz

$$ f \left({a}\right) + \int_a^x x f'' \left({t}\right) \ \mathrm d t + x f' \left({a}\right) - a f' \left({a}\right)-\int_a^x \, t f'' \left({t}\right) \ \mathrm d t $$

Bazi ortak terimleri disari cekersek

$$ f \left({a}\right) + (x-a) f' \left({a}\right) + \int_a^x (x-t) f'' \left({t}\right) \ \mathrm d t $$

Ayni teknigi bir daha uygulayinca

$$ f \left({x}\right) = f \left({a}\right)+(x-a) f' \left({a}\right)+ \frac 1 2 (x-a)^2f'' \left({a}\right) + \frac 1 2 \int_a^x (x-t)^2 f''' \left({t}\right) \ \mathrm d t$$

Tum bunlari daha genel olarak kurallastirmamiz gerekirse, tumevarim
(induction) teknigini kullanalim, varsayiyoruz ki Taylor'un Teorisi bir $n$
icin gecerli ve

$$ f(x) = f(x) + \frac{f'(a)}{1!}(x - a) + ... 
\frac{f^{(n)}(a)}{n!}(x - a)^n + 
\int_a^x \frac{f^{(n+1)} (t)}{n!} (x - t)^n \ \mathrm d t
$$

Sonuncu entegrali parcali entegral teknigi ile tekrar yazmamiz
mumkundur. $(x-t)^n$'in anti-turevi (anti-derivative)
$\frac{-(x-t)^{n+1}}{n+1}$ ile verilir, o zaman

$$  \int_a^x \frac{f^{(n+1)} \left({t}\right)}{n!} \left({x - t}\right)^n \ 
\mathrm d t$$

$$ =  - \left[ \frac{f^{(n+1)} \left({t}\right)}{(n+1)n!} \left({x - t}\right)^{n+1} \right]_a^x + \int_a^x \frac{f^{(n+2)} \left({t}\right)}{(n+1)n!} \left({x - t}\right)^{n+1} \ \mathrm d t $$

$$ = \frac{f^{(n+1)} (a)}{(n+1)!} (x - a)^{n+1} + \int_a^x \frac{f^{(n+2)} \left({t}\right)} {(n+1)!} \left({x - t}\right)^{n+1} \ \mathrm d t $$

Son entegral hemen cozulebilir

$$ R_n = \frac{f^{(n+1)}(\xi)}{(n+1)!} (x-a)^{n+1} $$

Alternatif Form

Hesapsal Bilim derslerinde bu serinin alternatif bir formu daha cok
karsimiza cikabilir. $f$'i $t$ yakininda ufak bir $h$ adimi atildigini
farzederek Taylor serileri uzerinden $f(t+h)$'i gelistirmek suretiyle
temsil edebiliriz. Eger $x = t+h$ ve $a = t$ alirsak alttaki orijinal
Taylor serisini

$$ f(x) = f (a)+(x-a) f'(a) + \frac1 2 (x-a)^2f''(x) + ...$$

donusturebiliriz. Baslayalim,

$$ x = t + h \Rightarrow h = x-t $$

$$ t = a $$

Once $a=t$ gecisini yapalim

$$ f(t+h) = f(t) +  f'(t)(x-t) + f''(t)\frac{(x-t)^2}{2!} + ...$$

Simdi $h = x-t$ gecisi

$$ f(t+h) = f(t) +  f'(t)h + f''(t)\frac{(h)^2}{2!} + ...$$

Boylece

$$ f(t+h) = f (t)+h f'(t) + \frac 1 2 h^2 f''(t) + ...$$

Bu tanimin, birinci turevin formuluyle olan alakasini gormek icin

$$ f'(x) \approx \frac {f(x+h) - f(x)}{h} $$

ifadesini hatirlamak yararli olabilir, yaklasiksal isareti $\approx$
kullanildi, cunku bu ifade sadece $h \to 0$ iken dogrudur (turevlerin limit
olarak tanimindan hareketle). Biraz cebirsel manipulasyon yaparsak

$$ f(x+h) - f(x) \approx f'(x)h  $$

$$ f(x+h)  \approx f'(x)h + f(x) $$

En son formulun Taylor serisi 1. derece acilimiyla ayni oldugu goruluyor. 

Iki Boyutlu f(x,y) Fonksiyonunun Taylor Acilimi

Bir $f(x,y)$ fonksiyonunun Taylor acilimini yapmak icin, daha onceden
gordugumuz tek boyutlu fonksiyon acilimindan faydalanabiliriz. 

Once iki boyutlu fonksiyonu tek boyutlu olarak gostermek gerekir. Tek
boyutta isleyen bir fonksiyon $F$ dusunelim ve bu $F$, arka planda iki
boyutlu $f(x,y)$'i kullaniyor olsun

Eger 

$$ f(x_0 +\Delta x, y_o + \Delta y) $$

fonksiyonun acilimini elde etmek istiyorsak, onu

$$ F(t) = f(x_0 + t\Delta x, y_o + t\Delta y) $$

uzerinden $t=1$ oldugu durumda hayal edebiliriz. $x,y$ parametrize 
oldugu icin  $f(x(t),y(t))$, yani

$$ x(t) = x_0 + t\Delta x $$

$$ y(t) = y_0 + t\Delta y $$

$F(t)$ baglaminda $x_o, y_o, \Delta x, \Delta y$ sabit olarak kabul edilecekler. Simdi bildigimiz
tek boyutlu Taylor acilimini bu fonksiyon uzerinde, bir $t_0$ noktasi yakininda 
yaparsak,

$$ F(t) = F(t_0) + F'(t_0)(t-t_0) + \frac{1}{2}F''(t_0)(t-t_0)^2 + ... $$

Eger $t=1,t_0=0$ dersek

$$ F(1) = F(0) + F'(0) + \frac{1}{2}F''(0) + ... $$

olurdu. Bu iki degeri, yani $t=1,t_0=0$'i kullanmamizin sebepleri $t=1$ ile
mesela $x_0 + t\Delta x$'in $x_0 + \Delta x$ olmasi, diger yandan $t=0$ ile
ustteki formulde $t$'nin yokolmasi, basit bir tek boyutlu acilim elde
etmek.

Simdi bize gereken $F',F''$ ifadelerini $x,y$ baglaminda elde edelim, ki bu
diferansiyeller $F$'in $t$'ye gore birinci ve ikinci diferansiyelleri. Ama
$F$'in icinde $x,y$ oldugu icin acilimin Zincirleme Kanunu ile yapilmasi
lazim.

$$ \frac{dF}{dt} = \frac{\partial F}{\partial x}\frac{dx(t)}{dt} +
\frac{\partial F}{\partial y}\frac{dy(t)}{dt} 
 $$

Ayrica

$$ \frac{d}{dt}x(t) = \Delta x $$

$$ \frac{d}{dt}y(t) = \Delta y $$

olduguna gore, tam diferansiyel daha da basitlesir

$$ \frac{dF}{dt} = \frac{\partial F}{\partial x}\Delta x +
\frac{\partial F}{\partial y}\Delta y
 $$

Simdi bu ifadenin bir tam diferansiyelini alacagiz. Ama ondan once sunu
anlayalim ki ustteki ifade icinde mesela birinci terim de aslinda bir
fonksiyon, ve asil hali

$$ \frac{dF}{dt} = \frac{\partial F(x(t),y(t))}{\partial x}\Delta x + ...
 $$

seklinde. O zaman, bu terim uzerinde tam diferansiyel islemini bir daha
uyguladigimizda, Zincirleme Kanunu yine isleyecek, mesela ustte $dx(t)/dt$'nin 
bir daha disari cikmasi gerekecek. O zaman 

$$ \frac{d^2F}{dt} =
\bigg(
\frac{\partial ^2 F}{\partial x^2}\frac{dx}{dt} + 
\frac{\partial ^2 F}{\partial x \partial y}\frac{dy}{dt} + 
\bigg) \Delta x +
\bigg(
\frac{\partial ^2 F}{\partial y \partial x}\frac{dy}{dt} + 
\frac{\partial ^2 F}{\partial y^2}\frac{dx}{dt} + 
\bigg) \Delta y 
$$

$$ =
\bigg(
\frac{\partial ^2 F}{\partial x^2}\Delta x + 
\frac{\partial ^2 F}{\partial x \partial y}\Delta y 
\bigg) \Delta x +
\bigg(
\frac{\partial ^2 F}{\partial y \partial x}\Delta x + 
\frac{\partial ^2 F}{\partial y^2}\Delta y 
\bigg) \Delta y 
 $$

$$ =
\bigg(
\frac{\partial ^2 F}{\partial x^2}\Delta x^2 + 
\frac{\partial ^2 F}{\partial x \partial y}\Delta y \Delta x
\bigg) +
\bigg(
\frac{\partial ^2 F}{\partial y \partial x}\Delta x \Delta y + 
\frac{\partial ^2 F}{\partial y^2}\Delta y^2
\bigg) 
 $$

Calculus'tan biliyoruz ki 

$$ 
\frac{\partial ^2 F}{\partial x \partial y} = 
\frac{\partial ^2 F}{\partial y \partial x} 
 $$

Daha kisa notasyonla

$$ f_{xy} = f_{yx} $$

Yani kismi turevin alinma sirasi farketmiyor. O zaman, ve her seyi daha
kisa notasyonla bir daha yazarsak

$$ =
(f_{xx}\Delta x^2 + f_{xy}\Delta y \Delta x ) +
(f_{xy}\Delta x \Delta y + f_{yy}\Delta y^2 )
 $$

$$
\frac{d^2F}{dt}  =
(f_{xx}\Delta x^2 + 2f_{xy}\Delta y \Delta x + f_{yy}\Delta y^2 )
 $$

Artik elimizde $F$ ve $F'$ var, bunlari 

$$ F(1) = F(0) + F'(0) + \frac{1}{2}F''(0) + ... $$

icine yerlestirebiliriz. En son su kaldi, $F(0)$ nedir? $F$'in $t=0$ oldugu
anda degeridir, 

$$ F(t) = f(x_0 + t\Delta x, y_o + t\Delta y) $$

$$ F(0) = f(x_0 + 0 \cdot \Delta x, y_o + 0 \cdot \Delta y) $$

$$ = f(x_0 , y_o) $$

Benzer sekilde, tum turevler de $t=0$ noktasinda kullanilacaktir, o zaman
onlar da

$$ F'(0) = f_x(x_0,y_0) \Delta x + f_y(x_0,y_0) \Delta y $$

$$ F''(0) =  
f_{xx}(x_0,y_0)\Delta x^2 + 2f_{xy}(x_0,y_0)\Delta y \Delta x + 
f_{yy}(x_0,y_0)\Delta y^2 
$$

seklinde olurlar. Tamam. Simdi ana formulde yerlerine koyalim,

$$ 
\begin{array}{lll}
F(1) &=& f(x_0 , y_o) +  \\ \\ 
&& f_x(x_0,y_0) \Delta x + f_y(x_0,y_0) \Delta y +   \\ \\
&& \frac{1}{2} 
[ 
f_{xx}(x_0,y_0)\Delta x^2 + 
2f_{xy}(x_0,y_0)\Delta y \Delta x +
f_{yy}(x_0,y_0)\Delta y^2 
] + ... 
\end{array}
 $$


Ve

$$  F(1) = f(x_0 +\Delta x, y_o + \Delta y) $$ 

olduguna gore, Taylor 2D acilimimiz tamamlanmis demektir. 

Kaynak

MIT OCW 18.01 Ders 38

http://www.proofwiki.org/wiki/Taylor's\_Theorem

http://www.math.ubc.ca/~feldman/m200/taylor2dSlides.pdf

http://math.uc.edu/~halpern/Calc.4/Handouts/Taylorseries.pdf

\end{document}



