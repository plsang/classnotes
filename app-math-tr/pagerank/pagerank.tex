\documentclass[12pt,fleqn]{article}\usepackage{../common}
\begin{document}
Google Nasil Isler? 

Ozdeger/Vektor Hesabinda Ust Metot (Power Method)

Diyelim ki bir $A$ matrisinin, ki $A \in \mathbb{R}^{n \times n}$,
ozdegerleri $\lambda_1,...,\lambda_n$ ve ozvektorleri $v_1,..,v_n$ olarak
verilmis. Bu demektir ki her $i=1,..,n$ icin $Av_i = \lambda_i v_i$.

Farzedelim ki bu matrisin tum ozvektorleri bir ``ozbaz (eigenbasis)''
olusturuyor ve bu baz ile $\mathbb{R}^n$'deki herhangi bir vektoru temsil
edebiliyoruz. Yine farzedelim ki $|\lambda_1| > |\lambda_2| > .. >
|\lambda_n| $. Biz bu yazida $\lambda_1$'e 
baskin (dominant) ozdeger diyecegiz.

Simdi herhangi bir $v_0 \in \mathbb{R}^n$'i alalim. Usttekiler isiginda
$\mu_1,..,\mu_n$ olarak katsayilar olmalidir, ki 

$$ v_o = \mu_1v_1 + .. + \mu_nv_n 
$$

cunku ozvektorler bir baz olusturuyorlar. Simdi her iki tarafi soldan $A$
ile carpalim, ayrica $Av_i = \lambda_iv_i$ esitliginden hareketle ikinci
bir esitligi de en sagda belirtelim,

$$ A v_o = \mu_1 A v_1 + .. + \mu_n A v_n =
A \lambda_1v_1 + ... + A \lambda_nv_n
$$

Simdi ustteki ifadeyi $A$ ile bir daha, hatta birkac defa carpalim, diyelim
toplam $m$ kere,

$$ A^m v_o = \mu_1 A^m v_1 + .. + \mu_n A^m v_n =
A^m \lambda_1v_1 + ... + A^m \lambda_nv_n
$$













Kaynaklar

\url{http://www.math.mcgill.ca/feys/documents/tutnotesR18.pdf}

Murphy, K., CS340: Machine Learning Lecture Notes, \url{www.ugrad.cs.ubc.ca/~cs340}


\end{document}
