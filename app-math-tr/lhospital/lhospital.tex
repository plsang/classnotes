\documentclass[12pt,fleqn]{article}
\setlength{\parindent}{0pt}
\usepackage{graphicx}
\usepackage{cancel}
\usepackage{listings}
\usepackage[latin5]{inputenc}
\usepackage{color}
\setlength{\parskip}{8pt}
\setlength{\parsep}{0pt}
\setlength{\headsep}{0pt}
\setlength{\topskip}{0pt}
\setlength{\topmargin}{0pt}
\setlength{\topsep}{0pt}
\setlength{\partopsep}{0pt}
\setlength{\mathindent}{0cm}
\usepackage{latexsym}
\usepackage{showkeys}
\renewcommand*\showkeyslabelformat[1]{(#1)}

\begin{document}
L'Hospital (l'H�pital) Kurali

\[ \lim_{x \to a} \frac{f(x)}{g(x)}\]

bazen hesaplanamaz, cunku hem $f(a)$ hem $g(a)$ sifira esittir. Bu durum
$0/0$ gibi acaip bir durum ortaya cikarir, ki boyle bir seyi hesaplamak mumkun
degildir. $0/0$'in diger bir adi ``hesaplanamayan form (indeterminate
form)''. Fakat L'Hospital 1. Senaryo kuralina gore, 

\[ \lim_{x \to a} \frac{f(x)}{g(x)} = 
\frac{f'(a)}{g'(a)} 
\]

esitligi kullanilabilir. 

Ispat

$f'(a)$ ve $g'(a)$'dan geriye dogru gidelim, ki bu tanimlarin kendisi de
birer limit zaten. 

\[ 
\frac{\displaystyle f'(a)}{g'(a)} = 
\frac
{\lim_{x \to a}\frac{\displaystyle f(x) - f(a)}{\displaystyle x-a}}
{\lim_{x \to a}\frac{\displaystyle g(x) - g(a)}{\displaystyle x-a}}
\]

\[ =
\lim_{x \to a } \frac
{\frac{\displaystyle f(x) - f(a)}{\displaystyle x-a}}
{\frac{\displaystyle g(x) - g(a)}{\displaystyle x-a}}
\]

\[ =
\lim_{x \to a } \frac
{f(x) - f(a) }{g(x)-g(a)}
\ \ \ \label{1}
\]

$x \to a$ iken $g(a)$ ve $f(a)$'nin sifira gittigini biliyoruz, tum bu
islere girmemizin sebebi oydu zaten, o zaman 

\[ =
\lim_{x \to a } \frac
{f(x) - 0 }{g(x)- 0} = 
\lim_{x \to a }  \frac{f(x)  }{g(x)}
\]

Bir diger hesaplanamayan form $\infty/\infty$ icin de L'Hospital Kurali
aynen gecerli. O formun ispati biraz daha cetrefil, ama kullanma baglaminda
aynen isliyor. 

Uyari: Eger $0/0$ ya da  $\infty/\infty$ durumu ortada yoksa L'Hospital
Kuralini kullanmayin. Ispat da zaten boyle bir durumun oldugu bilgisinden 
hareketle sonuca ulasiyor. 

$\infty/\infty$ Durumu

Bu ispat icin

$L=\lim_{x \to \infty} f'(x)/g'(x)$ kabul edelim ve oyle bir $a$ secelim ki

$\frac{f'(x)}{g'(x)} \stackrel{\approx}{\epsilon} L$, $x>a$. 

olsun. 

Not: Hocanin notasyonuna gore eger $a,b$ birbirlerine $\epsilon$ kadar
yakinlarsa $a \stackrel{\approx}{\epsilon} b$ kullanilir.

Simdi ispatin geri kalaninda su alttaki iki yaklasiksalligi ispat
etmek bir yontemdir, ($x\gg 1$ olmak uzere)

\[ 
\frac{f(x)}{g(x)} \stackrel{\approx}{\epsilon}
\frac{f(x)-f(a)}{g(x)-g(a)} \stackrel{\approx}{\epsilon}
L
 \]

Ortadaki ifade (1)'e benziyor. Simdi ilk yaklasiksallik icin 

\[ f(x) - f(a) = f(x) [ 1 - f(a)/f(x)]  \]

yazariz. Bu basit bir cebirsel manipulasyon. Ayni seyi $g(x)$'li bolen icin
de yapariz. Birada yazarsak

\[ \frac{f(x)}{g(x)} \stackrel{\approx}{\epsilon}
\frac{f(x) [ 1 - f(a)/f(x)] }{g(x) [ 1 - g(a)/g(x)] } 
\]

Her iki taraftan $f(x)/g(x)$ iptal olabilir. 

Sonra limit teorisini kullaniriz. Bu yaklasiksalligin varligi bariz, cunku,
herhangi bir $\epsilon$ icin $x_0$'i yeterince buyuk secebiliriz ki alttaki

\[ 1 - \epsilon < \frac{1 -  f(a)/f(x)}{1-g(a)/g(x)} < 1 + \epsilon \]

$x > x_0$ icin hep dogru olur. Unutmayalim, $f(x),g(x)$, $x\to \infty$ iken
sonsuzluga gidiyorlar. Sabit bir $f(a),g(a)$ degerini sonsuza giden bir
degerle bolunce ortaya sifir cikiyor, elde kalanlar yaklasiksal olarak $1/1$. 

Ikinci yaklasiksallik icin, Cauchy Ortalama Deger Teorisi (Cauchy
Mean-value Theorem) kullaniriz.  $a < c < x$ seklinde oyle bir $c$ vardir ki
$(f(x) - f(a))g'(c) = (g(x)-g(a))f'(c)$ dogrudur. Yani

\[ \frac{f(x)-f(a)}{g(x)-g(a)} = \frac{f'(c)}{g'(c)} \]

ki bu ifade $L$'e  $\epsilon$ kadar yakindir.

Kaynaklar

[1] Thomas Calculus 11th Edition, sf. 292

[2] Arthur Mattuck, Introduction to Analysis, sf. 220


\end{document}
