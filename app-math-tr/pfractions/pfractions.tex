\documentclass[12pt,fleqn]{article}\usepackage{../common}
\begin{document}
Kismi Kesirler Yontemi (Partial Fractions Method)

Bazen su sekildeki bir kesiri 

\[ \frac{8x + 22}{(x-1)(x+5)} \]

bolendeki her carpan ayri bir kesir parcasinda olacak sekilde o parcalarin
toplami olarak gostermek faydalidir. 

\[ \frac{...}{(x-1)} + \frac{...}{(x+5)} \]

Bu tur bir parcali kesirin entegralini almak cok daha kolaydir
mesela. Fakat bu parcalama islemini nasil yapacagiz?

Cebir kullanarak bu problemi cozebiliriz. Ustte nokta nokta olan yerlerin
ne oldugunu merak ediyoruz, o zaman onlara $A,B$ degiskenlerini atarsak

\[ \frac{A}{(x-1)} + \frac{B}{(x+5)} = \frac{8x + 22}{(x-1)(x+5)}\]

Eger bolum kisminda temiz bir esitlik elde etmek istiyorsak, o zaman
ustteki kesirlerin bolen kismini birbirinin aynisi haline getirmeliyiz. Ilk
terimin bolum, bolen kismini $(x+5)$, ikincisinin bolum, bolen kismini
$(x-1)$ ile carparsak, bu esitligi elde ederiz. 

\[ \frac{(x+5)A}{(x-1)(x+5)} + \frac{(x-1)B}{(x+5)(x-1)} = 
\frac{8x + 22}{(x-1)(x+5)}
\]


Bolen kismi birbirine esit olduguna gore, artik sadece kesirlerin ust
kismini kullanabiliriz, cunku aradigimiz bilinmeyenler orada. 

\[ (x+5)A + (x-1)B = 8x+22 \]

Esitligin sol tarafinin acilimini dusunursek, 

\[ xA + .. + Bx + .. = 8x + .. \]

\[ x(A + B) +  .. = 8x + .. \]


ve her iki tarafta $x$'in katsayilarinin ayni olmasi zorunlulugundan
hareketle

\[ A + B = 8 \]

olacaktir. Benzer sekilde geri kalan sabitleri esitlersek 

\[ 5A - B = 22 \]

O zaman elimizde iki bilinmeyen, iki denklem var, bu sistemi cozmek cok
kolay! 

\[ B = 5A - 22 \]

Oteki denkleme sokalim

\[ A + 5A - 22 = 8 \]

\[ 6A = 30 \]

\[ A = 5 \]

\[ B = 3 \]

Demek ki kismi kesirlerimiz soyle olacak 

\[ \frac{5}{(x-1)} + \frac{3}{(x+5)} = \frac{8x + 22}{(x-1)(x+5)}\]



\end{document}
