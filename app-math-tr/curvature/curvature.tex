\documentclass[12pt,fleqn]{article}\usepackage{../common}
\begin{document}
Egim (Curvature)

Kesit seviyeleri tekniginde bir egri normal formda degil, dolayli (implicit) bir
fonksiyon ile $F(x,y) = 0$ olarak gosterilir. Bu fonksiyonun tam diferansiyelini
alirsak,

\[ dF = F_x dx + F_y dy = 0  \]

\[ dy = \frac{-F_x}{F_y}dx  \]

\[ y' = \frac{dy}{dx} = \frac{-F_x}{F_y} = f'(x) = \frac{df}{dx} \]

Burada bir faraziye daha var, o da aslinda ilk verilen formulde olmasa bile
$y=f(x)$ olarak kabul etmemiz, yani $F(x,y)$ nasil bir formul olursa olsun,
$y$'nin $x$'leri icerecek sekilde tekrar duzenlenebilecegini farz etmemiz,
boylece $F(x,f(x))$ olabilecegini soylemis oluyoruz.

Simdi $y'$'in turevini bir daha alalim. Yukaridaki $y'$ formulunde en sag
taraf bir bolme islemi icerdigi icin burada Calculus'un Bolumler Kuralini
(Quotient Rule) uygulamamiz lazim (detaylar icin Bolum Kurali yazisina
bakiniz). Bu kural soyle gosterilir:

\[ \frac{d}{dx}\bigg(\frac{u}{v}\bigg) = 
\frac{\displaystyle \frac{v du}{dx} - \frac{u dv}{dx}}{v^2} \]

Bolumler Kurali icin $u$ ve $v$ tanimlari nedir? 

\[ u = -F_x(x,f(x))  \]

\[ v = F_y(x,f(x)) \]

O zaman

\begin{equation} v \frac{du}{dx} = F_y \frac{dF_x}{dx} \label{eq1} \end{equation}

\begin{equation} u \frac{dv}{dx} = -F_x \frac{dF_y }{dx} \label{eq2} \end{equation}

Bunlardan mesela $dF_x/dx$ uzerinde Zincirleme Kanunu (Chain Rule) uygulamak
lazim (bu kural tam integral kuralinin bir sonucu). 

\[ \frac{d F_x(x,f(x)) }{dx} = \frac{\partial F_x}{\partial  x}(x,f(x))+\frac{\partial F_x}{\partial y}\frac{df}{dx}\\ \]

\begin{equation} = F_{xx}(x,f(x))+F_{xy}(x,f(x))f'(x) \label{eq3}  \end{equation}

\begin{equation} \frac{d F_y(x,f(x)) }{dx} =  F_{xy}(x,f(x))+F_{yy}(x,f(x))f'(x) \label{eq4} \end{equation}

Zincirleme Kanunu niye ustteki sekilde acildi? Tam Diferansiyeli bir daha
hatirlayalim:

\[ df = \frac{\partial f}{\partial x} dx + \frac{\partial f}{\partial y} dy  \]

\[ \frac{df}{dx} = \frac{\partial f}{\partial x} \frac{dx}{dx} + \frac{\partial f}{\partial y} \frac{dy}{dx}  \]

\[ \frac{df}{dx} = \frac{\partial f}{\partial x} + \frac{\partial f}{\partial y} \frac{dy}{dx}  \]

O zaman formuller \ref{eq1}, \ref{eq2}, \ref{eq3} ve \ref{eq4} bir araya konulursa,

\[ y '' = - \frac{F_yF_{xx} - F_y F_{xy}\frac{F_x}{F_y} - F_xF_{xy} + F_xF_{yy}\frac{F_x}{F_y}}{F_y^2}\\ \]

\[ y '' = - \frac{F_yF_{xx} - F_{xy}F_x - F_xF_{xy} + \frac{F_x^2F_{yy}}{F_y}}{F_y^2} \]

Ustteki bolumun hem bolen, hem bolunen terimlerini $F_y$ ile carparsak, ve
sadelestirirsek

\[ y '' = - \frac{F_y^2F_{xx} - 2F_{xy}F_xF_y + F_x^2F_{yy}}{F_y^3} \]

Simdi Wolfram sitesinde turetimi gosterilen egim (curvature) formulune
bakalim [2]. Not: Eger 

\[ \kappa = \frac{x'y''-y'x''}{\bigg(x'^2 + y'^2 \bigg)^{3/2}} \]

formulunun alttaki formule nasil donustugu tam anlasilir degilse, hatirlayalim ki,
$y=f(x)$, ve $x'=1$, ve $x'' = 0$. 

Bu formulun Courant [1] sf. 231'de benzer bir formunu goruyoruz [3]. 

\[ \kappa = \frac{f''}{(1+f'^2)^{3/2}} \]

Bu formuldeki $f''$ yani $y''$ icin ustte buldugumuz sonucu, $f'$ yani $y'$ icin
bu yazinin basindaki formulu koyarsak,

\[ 
\kappa = \frac
{-\frac
{\displaystyle F_y^2F_{xx} - 2F_{xy}F_xF_y +  F_x^2F_{yy}}{\displaystyle F_y^3}}
{(1+f'^2)^{3/2}} 
\]  

Bolen kismi nedir?

\[ (1+f'^2)^{3/2} = \bigg( 1 + \bigg(\frac{-F_x}{F_y}\bigg)^2 \bigg)^{3/2}  \]

\[ = \bigg( 1 + \frac{F_x^2}{F_y^2} \bigg)^{3/2}  \]

\[ = \bigg( \frac{F_y^2 + F_x^2}{F_y^2} \bigg)^{3/2}  \]

\[ = (F_y^2 + F_x^2)^{3/2}(F_y^{-2})^{3/2}  \]

\[ = (F_y^2 + F_x^2)^{3/2}F_y^{-6/2}  \]

\[ = (F_y^2 + F_x^2)^{3/2}F_y^{-3} \]

Yerine koyarsak,

\[ 
\kappa = \frac{\displaystyle
- \frac{F_y^2F_{xx} - 2F_{xy}F_xF_y + F_x^2F_{yy}}{F_y^3}}
{(F_y^2 + F_x^2)^{3/2}F_y^{-3}}
 \]

$F_y^{-3}$ ve $F_y^{3}$ birbirlerini iptal ederler ve sonuc:

\begin{equation}\kappa = \frac{F_y^2F_{xx} - 2F_{xy}F_xF_y + F_x^2F_{yy}}{(F_y^2 + F_x^2)^{3/2}} \label{eq5}\end{equation}

Ustteki unlu egim (curvature) formuludur. 

Bu egim formulunun diger bir sekli soyledir ($F$ yerine $\phi$ kullanirsak)

\[ \kappa = \bigtriangledown \cdot \frac{\bigtriangledown \phi}{|\bigtriangledown \phi|} \]

Bunun okunus sekli ``birim normal gradyanin uzaklasim olcusu (divergence of the
unit normal gradient)'' seklindedir. Acaba bu formul, \ref{eq5}. formul ile
uyumlu mu?

\[ \kappa = \nabla \cdot \frac{\nabla \phi}{|\nabla \phi|}  \]

\[ = \nabla \cdot \frac{(\phi_x,\phi_y)}{\sqrt{\phi_x^2+\phi_y^2}} \]

\[ = \left(\partial_x \frac{\phi_x}{\sqrt{\phi_x^2+\phi_y^2}}\right)+ 
\left(\partial_y \frac{\phi_y}{\sqrt{\phi_x^2+\phi_y^2}}\right)  \]

\[ = \frac{\phi_{xx}}{\sqrt{\phi_x^2+\phi_y^2}} - \frac{\phi_x (\phi_x\phi_{xx}+\phi_y\phi_{xy})}
{(\phi_x^2+\phi_y^2)^{3/2}} +
\frac{\phi_{yy}}{\sqrt{\phi_x^2+\phi_y^2}} - \frac{\phi_y(\phi_x\phi_{xy}+\phi_y\phi_{yy})}
{(\phi_x^2+\phi_y^2)^{3/2}}  \]

\[ = \frac{\phi_{xx}(\phi_x^2+\phi_y^2) - \phi_x
  (\phi_x\phi_{xx}+\phi_y\phi_{xy}) +\phi_{yy}(\phi_x^2+\phi_y^2) -
  \phi_y(\phi_x\phi_{xy}+\phi_y\phi_{yy})}{(\phi_x^2+\phi_y^2)^{3/2}} \]

\[ = \frac{\phi_{xx}\phi_y^2 - 2\phi_x\phi_y\phi_{xy} + \phi_{yy}\phi_x^2}{(\phi_x^2+\phi_y^2)^{3/2}}  \]

Bu formul bizim \ref{eq5}. formul ile tipatip ayni.

Ustteki islemlerde uzaklasim olcusu (divergence) operatoru $\nabla \cdot$ ile
gradyan operatoru $\nabla$ arasindaki farki belirtelim:  $\nabla \cdot$
operatoru $F(x,y)$ uzerinde kismi turevlerin toplamini verir, yani bir skalar
tek sayi dondurur. Gradyan ise her bir elemani bir kismi tureve tekabul eden bir
{\em vektor} geri getirir. 

Python Numpy kodlamasi baglaminda, daha once {\em Kesit Seviyeleri} yazisinda
ayriksal olarak bir \verb!phi! degiskeni icindeki bir fonksiyon uzerinde
egimselligi (curvature) soyle hesaplamistik:

\begin{lstlisting}[language=Python]
gradPhiY, gradPhiX = np.gradient(phi)
absGradPhi=np.sqrt(gradPhiX**2+gradPhiY**2)                               

normGradPhiX=gradPhiX/(absGradPhi+(absGradPhi==0))
normGradPhiY=gradPhiY/(absGradPhi+(absGradPhi==0))

divYnormGradPhiX, divXnormGradPhiX=np.gradient(normGradPhiX)
divYnormGradPhiY, divXnormGradPhiY=np.gradient(normGradPhiY)
                       
K = divXnormGradPhiX + divYnormGradPhiY
\end{lstlisting}

Bu satirlarin $\nabla \cdot \frac{\nabla \phi}{|\nabla \phi|}$ ifadesiyle
birebir uyum gosterdigini herhalde gorebiliyoruz. Satir 1, $\nabla \phi$
ifadesidir. Satirlar 2-4 $\frac{\nabla \phi}{|\nabla \phi|}$ islemini
gerceklestiriyor, gradyani onun uzunluguna (magnitude) bolerek onu birim vektoru
haline getiriyor. Satirlar 6-7 tekrar sonucun gradyanini bir daha aliyor, ama bu
sefer hesapsal kismi turevleri birbiriyle topluyor, boylece uzaklasim olcusu
(divergence) hesaplanmis oluyor. Tum bu islemlerin sonucu egimsellik $\kappa$
oluyor.

Dikkat edilirse Python kodundaki K yani $\kappa$, N x N boyutlu bir matristir,
bu mantikli cunku $\kappa$ hesabi icin kullandigimiz $F_x$, $F_y$ gibi
turevler aslinda $F_x(x,y)$, $F_y(x,y)$ formullerine sahipler, yani her $x,y$
kombinasyonu icin farkli bir sonuc dondurebilirler. Bu sebeple K yani $\kappa$
$\phi$ fonksiyonunun her $x,y$ noktasi icin tanimlidir. 

Bazen literaturde $\nabla \cdot$ yerine $div(..)$ kullanildigini gorebilirsiniz,
bu operatorlerin ikisi de aynidir.

--

Kaynaklar

[1] Courant, Introduction to Calculus and Analysis Volume 2, sf. 223-232

[2] Wolfram - http://mathworld.wolfram.com/Curvature.html

[3] Bu arada o karmasik formul yerine yaklasiksal olarak hesaplama
sirasinda sadece $f''$ kullanmak ta mumkun [4]

[4] Strang, G. Computational Science and Engineering, sf. Introduction bolumu

\end{document}

