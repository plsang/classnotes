\documentclass[12pt,fleqn]{article}\usepackage{../common}
\begin{document}
Kosulsal Rasgele Alanlar (Conditional Random Fields - CRF)

Kosulsal Olurluk (Conditional Likelihood)

Diyelim ki elimizde egitim verisi olarak ikili $<x,y>$ veri noktalari
var. O zaman $y$'nin $x$'e kosulsal olarak bagli (conditional on) bir
dagilimi oldugunu soyleyebiliriz. 

$$ y \sim f(x;\theta) $$

Yani her $x$ icin farkli bir $y$ dagilimi ortaya cikabilir. Ve tum bu
farkli dagilimlarin ortak noktasi $\theta$ parametresidir. Kosulsal
olasilik yani soyle yazilabilir, 

$$ P(Y=y | X=x;\theta) $$

Usttekiler $Y$ icin bir model ortaya koydu, peki elimizde $X$'in dagilimi
icin bir olasilik modelimiz var mi? Cevap hayir. Niye? Dusunelim, $p(y,x)$
nedir ?

$$ p(x,y) = p(x)p(y|x) $$

Ustte $p(y|x)$'i tanimlayacak ($\theta$ uzerinden) bir olasilik demeti /
ailesi tanimladik, fakat elimizde $p(x)$ dagilimini verecek bir model
yok, o zaman $p(x,y)$'yi tanimlayacak bir model de yok.

Fakat bu dunyanin sonu degil. Belki de Makine Ogrenimi bransinin bir
slogani su olmali: ``Ogrenmen gerekmeyen seyi ogrenme''. Ustteki ornekte
$p(y|x)$'i ogrenebiliriz, ama $p(x)$'i illa ogrenmemiz gerekir mi?

Siniflayici (classifier) ve takip edilen (supervised) ogrenim durumunu
dusunursek, bize egitim amacli olarak $<x,y>$ ikili veri noktalari
saglanacak. $x$ kaynak veri, $y$ tahmin edilecek (ya da basta egitim hedefi
olan) etiket olacak. $y$ icin bir model ortaya cikartiyoruz, cunku test
zamaninda $y$ olmayacak, fakat $x$ hep olacak. Yani $y$'nin modellenmesi
mecburi, cunku ``genelleyerek'' onun ne oldugunu bulacagiz, ama $x$ hep
verili.




\end{document}
